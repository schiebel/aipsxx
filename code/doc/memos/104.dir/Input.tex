\clearpage
\section{Input.h}

\subsection*{HEADER FILE DESCRIPTION}
 header file for class Input
  
\subsection*{ENVIRONMENT}
\begin{verbatim}
#define A_INPUT_H

#include <iostream.h>
#include <stdarg.h>
#include "Param.h"
#include <K_DList.h>                // CIC

DECLARE_ONCE K_DList<Param>;        // Template kludge (CIC)
DECLARE_ONCE K_DLinkable<Param>;    // Template kludge (CIC)
\end{verbatim}

\subsection*{CLASS DESCRIPTION}
   Class Input is:
      A linked list of parameters (defined by the helper class ``Param'')
      with various user interface attributes.
    
  
   Part of an example of a traditional ``key=value + help'' command-line user
   interface.
  

\subsection*{CLASS SUMMARY}
\begin{verbatim}
class Input
{
public:
     Input();
    ~Input();
     void Create(String key, String value, String help);
     void StdCreate(String stdkey, String key, String value, String help);
     void Close();
     double GetDouble(String key);
     int GetInt(String key);
     String GetString(String key);
     bool GetBool(String key);
     int Count();
     bool Debug(int l);
     bool Put(String key, String value);
     bool Put(String keyval);
};
extern Input inputs;
void error(char *fmt ...);          // The aips stdarg family
void warning(char *fmt ...);
void debug(int l, char *fmt ...);
\end{verbatim}

\subsection*{MEMBER FUNCTIONS}

       The default constructor is the only one!
       It enables the creation of parameters. 
       It puts the program in no-prompt mode unless environment variable
       HELP is defined with value "prompt".
       The output debug level is set according to the value of the
       environment variable DEBUG.
       The maximum number of error messages to be outputed is set according
       to the value of the environment variable ERROR.
\begin{verbatim}
     Input();
\end{verbatim}

       Destructor.
\begin{verbatim}
    ~Input();

                                // parameter creation

\end{verbatim}

       Create a new parameter, either from scratch or looking it
       up from an internal list of templates.
      
       The function also checks whether parameters can still be created,
       and whether key is unique for the program.
      
       The value, help and remaining arguments are all optional.
\begin{verbatim}
     void Create(String key, String value, String help);
     void StdCreate(String stdkey, String key, String value, String help);
\end{verbatim}

       Disable the creation of parameters. Highly recommended, but
       not required?
\begin{verbatim}
    void Close();

                                // query functions
\end{verbatim}

       Get the double value of the parameter (or 0.0 if unknown key).
       If the program is in prompt mode, ask the user for the value.
\begin{verbatim}
    double GetDouble(String key);
\end{verbatim}

       Get the int value of the parameter (or 0 if unknown key).
       If the program is in prompt mode, ask the user for the value.
\begin{verbatim}
    int GetInt(String key);
\end{verbatim}

       Get the string-type value of the parameter (or "" if unknown key).
       If the program is in prompt mode, ask the user for the value.
\begin{verbatim}
    String GetString(String key);
\end{verbatim}

       Get the boolean value of the parameter (or FALSE if unknown key).
       If the program is in prompt mode, ask the user for the value.
\begin{verbatim}
    bool GetBool(String key);
\end{verbatim}

       Get the number of parameters of this program
\begin{verbatim}
    int Count();
\end{verbatim}

       See if the current debug level is thresholded
\begin{verbatim}
    bool Debug(int l);

                                // modify function
\end{verbatim}

       Set the value for a named parameter.
       Return FALSE if key is an unknown parameter name.
       The default value is "".
      
       The function can also be called with a single argument of the
       form `key=value', where key is a valid new parameter name, and
       where value may be empty (the `=' is required though). In this 
       case a new parameter will be created (provided that creation is 
       still allowed).
\begin{verbatim}
    bool Put(String key, String value);
    bool Put(String keyval);
\end{verbatim}

\subsection*{NON-MEMBER FUNCTIONS}

\begin{verbatim}
   void error(char *fmt ...);          // The aips stdarg family
   void warning(char *fmt ...);
   void debug(int l, char *fmt ...);
\end{verbatim}
