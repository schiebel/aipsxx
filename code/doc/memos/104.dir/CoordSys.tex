\clearpage
\section{CoordSys.h}

\subsection*{HEADER FILE DESCRIPTION}
 Header file for CoordSys class  

\subsection*{ENVIRONMENT}
\begin{verbatim}
#define COORDSYS_H

#include <iostream.h>
#include "ImPixelCoord.h"
#include "ImageCoord.h"
#include "CoordSysType.h"
\end{verbatim}

\subsection*{CLASS DESCRIPTION}
    Class: CoordSys
          This class defines the coordinate system associated with an image.
   The coordinate system specification consists of two parts. The first
   relates ImPixelCoords to a "native" image coordinate system. For example,
   the "native " coordinate system for a VLA observed image would be
   l and m in the SIN system. The second part of the specification defines
   the coordinate system which the user wants to use for ImageCoords.
   For example, the user may want to use Galactic coordinates as ImageCoords.
  
\subsection*{CLASS SUMMARY}
\begin{verbatim}
class CoordSys
{
public:
        void SetRefPix(ImPixelCoord);
        ImPixelCoord GetRefPix() const;
        void SetRefCoord(ImageCoord ic);
        ImageCoord GetRefCoord() const;
        void SetDeltaCoord(ImageCoord ic);
        ImageCoord GetDeltaCoord() const;
        void SetRotAng(double);
        double GetRotAng() const;
        void SetNtvCrd(CoordSysType ct);
        CoordSysType GetNtvCrd() const;
        void SetImageCrd(CoordSysType ct);
        CoordSysType GetImageCrd() const;
        ImageCoord GetImageCoord(ImPixelCoord) const;
        ImPixelCoord GetImPixelCoord(ImageCoord) const;
        CoordSys();
        CoordSys(ImPixelCoord rp, ImageCoord rc, ImageCoord d, double a,
                CoordSysType n, CoordSysType i);
        ~CoordSys();
};
ostream& operator << (ostream& os, const CoordSys& cs);
\end{verbatim}

\subsection*{MEMBER FUNCTIONS}

           Set the coordinates of the reference ImPixel for coordinate
           system transformations to native ImageCoords.
\begin{verbatim}
        void SetRefPix(ImPixelCoord);
\end{verbatim}

           Return the reference ImPixel for coordinate transformations
\begin{verbatim}
        ImPixelCoord GetRefPix() const;
\end{verbatim}

           Set the reference native ImageCoord value at the reference pixel
           (presumably a value in radians for sky coordinates)
\begin{verbatim}
        void SetRefCoord(ImageCoord ic);
\end{verbatim}

           Return the reference native ImageCoord value at the reference pixel
           (presumably a value in radians for sky coordinates)
\begin{verbatim}
        ImageCoord GetRefCoord() const;
\end{verbatim}

           Set the increment native ImageCoord value, per ImPixelCoord
           increment (presumably a value in radians for sky coordinates)
\begin{verbatim}
        void SetDeltaCoord(ImageCoord ic);
\end{verbatim}

           Return the increment native ImageCoord value, per ImPixelCoord
           increment (presumably a value in radians for sky coordinates)
\begin{verbatim}
        ImageCoord GetDeltaCoord() const;
\end{verbatim}

           Set the rotation angle (radians)
\begin{verbatim}
        void SetRotAng(double);
\end{verbatim}

           Get the rotation angle (radians)
\begin{verbatim}
        double GetRotAng() const;
\end{verbatim}

           Set the type of the native coordinate system (SIN, NCP, etc.)
\begin{verbatim}
        void SetNtvCrd(CoordSysType ct);
\end{verbatim}

           Return the type of the native coordinate system (SIN, NCP, etc.)
\begin{verbatim}
        CoordSysType GetNtvCrd() const;
\end{verbatim}

           Set the type of the desired Image Coordinate system (SIN, NCP, etc.)
\begin{verbatim}
        void SetImageCrd(CoordSysType ct);
\end{verbatim}

           Return the type of the Image Coordinate system (SIN, NCP, etc.)
\begin{verbatim}
        CoordSysType GetImageCrd() const;
\end{verbatim}

           Convert ImPixel Coordinates to Image Coordinates (converting from
           system "ntvcrd" to "imagecrd").
\begin{verbatim}
        ImageCoord GetImageCoord(ImPixelCoord) const;
\end{verbatim}

           Convert Image Coordinates to ImPixel Coordinates (converting from
           system "imagecrd" to "ntvcrd" in the process).
\begin{verbatim}
        ImPixelCoord GetImPixelCoord(ImageCoord) const;

        // constructors and destructor
        CoordSys();
        CoordSys(ImPixelCoord rp, ImageCoord rc, ImageCoord d, double a,
                CoordSysType n, CoordSysType i);
        ~CoordSys();
\end{verbatim}

\subsection*{NON-MEMBER FUNCTIONS}

\begin{verbatim}
ostream& operator << (ostream& os, const CoordSys& cs);
\end{verbatim}
