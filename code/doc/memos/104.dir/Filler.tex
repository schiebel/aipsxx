\clearpage
\section{Filler.h}

\subsection*{HEADER FILE DESCRIPTION}
   This file contains the defintions for the Filler class.
  
\subsection*{ENVIRONMENT}
\begin{verbatim}
#define FILLER_H

#include "String.h"
#include "XYZ.h"
#include "YegSet.h"
#include "Telescope.h"
\end{verbatim}

\subsection*{CLASS DESCRIPTION}
   The Filler class produces an observation, consisting of a Telescope
   object, and its associated YegSet.
   The constructor defines the observation parameters. The only other
   method, GetTelescope, returns the associated Telescope object.

\subsection*{CLASS SUMMARY}
\begin{verbatim}
class Filler
{
public:
    Filler(const String& mytel, const XYZ& myxyz, double myinttime=10.0,
        double myfreq=1.4, double myHAbegH=-2.0,  double myHAbegH= 2.0,
        double mylat=34.0, double mydecl=50.0);
    Telescope& Filler::GetTelescope();
};
\end{verbatim}

\subsection*{MEMBER FUNCTIONS}
  
   The one (and only) constructor. Give the various observation parameters.
   The default parameters are for an obseratory at the VLA's latitude,
   observing a source at dec of 50 degrees, at 1.4 GHz, with a 10 sec
   integration time.
\begin{verbatim}
    Filler(
        const String& mytel,    // Observatory name.
        const XYZ& myxyz,       // Antenna coordinates (nanosec).
        double myinttime=10.0,  // Integration time (seconds).
        double myfreq=1.4,      // Observing frequency (GHz).
        double myHAbegH=-2.0,   // Beginning Hour Angle, (hours).
        double myHAbegH= 2.0,   // Ending Hour Angle, (hours).
        double mylat=34.0,      // Observatory latitude (degrees).
        double mydecl=50.0);    // Source declination (degrees).
\end{verbatim}

   Return the Telescope object associated with this observation.
\begin{verbatim}
    Telescope& Filler::GetTelescope();
\end{verbatim}
