\clearpage
\section{Image.h}

\subsection*{HEADER FILE DESCRIPTION}
   aips++ header

\subsection*{CLASS DESCRIPTION}
   Abstract base class which stores an image.

\subsection*{ENVIRONMENT}
\begin{verbatim}
#define A_IMAGE_H

#include <assert.h>
#include <iostream.h>
#include "K_String.h"
#include "CoordSys.h"
#include "ImPixelCoord.h"
#include "PixelCoord.h"
#include "ImageUnits.h"
#include "ImPixStep.h"
#include "HistFile.h"
#include "ImageDim.h"
#include "Pixel.h"

\end{verbatim}
\subsection*{CLASS SUMMARY}
\begin{verbatim}
class Image
{
public:
   Image(int m =0, int n = 0);
   ImageDim GetDim() const;
   virtual int GetNumEl() const = 0;
   K_String GetImageType() const;
   void SetImageType(K_String s);
   void SetCenPix(PixelCoord p);
   PixelCoord GetCenPix() const;
   void SetImCornPix(ImPixelCoord p);
   ImPixelCoord GetImCornPix() const;
   void SetPixStep(ImPixStep r);
   ImPixStep GetPixStep() const;
   void SetDataType(ImageUnits iu);
   ImageUnits GetDataType() const;
   void SetCoordSys(CoordSys * pcs);
   CoordSys * GetCoordSys() const;
   void SetRegBLC(ImPixelCoord blc);
   ImPixelCoord GetRegBLC() const;
   void SetRegTRC(ImPixelCoord trc);
   ImPixelCoord GetRegTRC() const;
   HistFile * GetHistPointer() const;
   void CenterImage(ImPixelCoord);
   int ConformsWith(const Image&) const;
   ImPixelCoord GetImPixCoord(PixelCoord) const;
   ImageCoord GetImageCoord(ImPixelCoord) const;
   ImageCoord GetImageCoord(PixelCoord) const;
   ImPixelCoord GetImPixCoord(ImageCoord) const;
   PixelCoord GetPixelCoord(ImPixelCoord) const;
   PixelCoord GetPixelCoord(ImageCoord) const;
   void AddHistory(K_String);
   void AddHistory(const Image&);
   void ListHistory() const;
   virtual void Scale(float) = 0;
   virtual void Fill(float) = 0;
   virtual int Extrema(Pixel&, Pixel&) const = 0;
   Pixel Maximum() const;
   Pixel Minimum() const;
   virtual void SetPixel(PixelCoord, float) = 0;
   virtual float GetPixel(PixelCoord) const = 0;
   virtual int GetImPixelVal(const ImPixelCoord&, float&) const = 0;
   virtual int UCombine (float, const Image&, float, const Image&, float) = 0;
   virtual int XCombine (float, const Image&, float, const Image&, float) = 0;
   virtual int ScaledAdd(float, const Image&) = 0;
   virtual  ~Image();

protected:
   ImageDim dim;
   K_String pImtype;
   ImPixelCoord regblc, regtrc;
   PixelCoord cenpix;
   ImPixelCoord imcornpix;
   ImPixStep pixstep;
   ImageUnits  datatype;
   CoordSys *pCsys;
   HistFile *pHist;

   int CombTwoImages(int, const Image&, const Image&, float&, float&,
		float&, float&);
   int CombOneImage(const Image&, float&, float&, float&, float&);
   void UpdateCenter();
};
\end{verbatim}

\subsection*{MEMBER FUNCTIONS}

      Constructor ("fill with value" version must be in derived class).
      (Probably need more constructors.)
\begin{verbatim}
   Image(int m =0, int n = 0);

   // Accessors.
\end{verbatim}
      The following accessor returns the image dimensions. If the
      dimensions of an image are [m, n], then PixelCoords run from 0 to m-1
      in the X dimension and from 0 to n-1 in the Y direction. Image
      dimensions can be set only by the constructor, or modified by adding
      pixels to a list image, or by an X/UCombine method. The default
      dimensions from the constructor are [0,0].
\begin{verbatim}
   ImageDim GetDim() const;

\end{verbatim}

      Return the number of pixels in the image
\begin{verbatim}
   virtual int GetNumEl() const = 0;

\end{verbatim}

      Return/set the image descriptor (a string)
\begin{verbatim}
   K_String GetImageType() const;
   void SetImageType(K_String s);

\end{verbatim}

      Set/return the specification of the "image center" in PixelCoord.
      N.B. The default "center" position is (m/2, n/2-1) in Pixel
      coordinates, i.e. displaced to Top Right of geometric image
      center.
\begin{verbatim}
   void SetCenPix(PixelCoord p);
   PixelCoord GetCenPix() const;

\end{verbatim}

      Set/return the ImPixelCoords of the TLC of the image (the origin
      (0,0) of PixelCoord).
\begin{verbatim}
   void SetImCornPix(ImPixelCoord p);
   ImPixelCoord GetImCornPix() const;

\end{verbatim}

      Set/return the ImPixStep for the image == the number of ImPixels
      per Pixel. Since the TLC is the origin for PixelCoords (CIC
      convention), the Y value of ImPixStep is negative. Step values
      must be integral.
\begin{verbatim}
   void SetPixStep(ImPixStep r);
   ImPixStep GetPixStep() const;

\end{verbatim}

      Set/return the type of data units for the image
\begin{verbatim}
   void SetDataType(ImageUnits iu);
   ImageUnits GetDataType() const;

\end{verbatim}

      Set/retrieve the "coordinate system" for the image. This defines
      the relationship between ImPixelCoord and some selected system
      of ImageCoords. The definition of the precise system of ImageCoords
      to be used is defined within the CoordSys object. When images are
      cloned, sub-images are formed, assignments made to new image objects,
      copied, etc., the same CoordSys object is referenced.
\begin{verbatim}
   void SetCoordSys(CoordSys * pcs);
   CoordSys * GetCoordSys() const;

\end{verbatim}

      Set/retrieve a "region of interest" for the image. This feature
      is defined by a BLC in ImPixelCoord and a TRC in ImPixelCoord.
      Implementation of a feature like this is not currently in place.
      Probably one would need a list of regions of interest.
\begin{verbatim}
   void SetRegBLC(ImPixelCoord blc);
   ImPixelCoord GetRegBLC() const;

   void SetRegTRC(ImPixelCoord trc);
   ImPixelCoord GetRegTRC() const;

\end{verbatim}

      Get pointer to history file
\begin{verbatim}
   HistFile * GetHistPointer() const;

\end{verbatim}

      Method to shift "center" of image to a specified ImPixelCoord. The
      reference value (imcornpix) of the TLC of the image is modified.
\begin{verbatim}
   void CenterImage(ImPixelCoord);

\end{verbatim}
      Checks whether another Image conforms with the current Image, i.e. can 
      be used with it in a "combine" or "scaledAdd" operation. The two images  
      must have the same "pCsys", the same "pixstep", the same "dataunit",
      and their "imcornpix" must be consistent with"pixstep". Returns TRUE
      if they are conformant, FALSE otherwise.
\begin{verbatim}
   int ConformsWith(const Image&) const;

\end{verbatim}

       Coordinate conversion - forward (various coordinate systems)
      Note that the first may return non-integral values of ImPixelCoord.
\begin{verbatim}
   ImPixelCoord GetImPixCoord(PixelCoord) const;
   ImageCoord GetImageCoord(ImPixelCoord) const;
   ImageCoord GetImageCoord(PixelCoord) const;

\end{verbatim}

       Coordinate conversion - reverse (various coordinate systems)
      Note that these may return non-integral values of ImPixelCoords or
      PixelCoord (even though the latter must be integral to refer to
      image data elements). 
\begin{verbatim}
   ImPixelCoord GetImPixCoord(ImageCoord) const;
   PixelCoord GetPixelCoord(ImPixelCoord) const;
   PixelCoord GetPixelCoord(ImageCoord) const;

\end{verbatim}

      History maintenance. Add comments (string), another Image's history,
      or print out file
\begin{verbatim}
   void AddHistory(K_String);
   void AddHistory(const Image&);
   void ListHistory() const;

\end{verbatim}

      Scale all pixel values in the image by a scaling factor.
\begin{verbatim}
   virtual void Scale(float) = 0;

\end{verbatim}

      Replace all pixel values in the image by a new value.
\begin{verbatim}
   virtual void Fill(float) = 0;

\end{verbatim}

      Return the maximum and minimum Pixel value (by returning Pixels)
      in the image. Returns FALSE if image is empty, otherwise TRUE.
\begin{verbatim}
   virtual int Extrema(Pixel&, Pixel&) const = 0;

\end{verbatim}

      Return the maximum and minimum Pixel values (by returning Pixels)
      in the image. Exits if image is empty.
\begin{verbatim}
   Pixel Maximum() const;
   Pixel Minimum() const;

\end{verbatim}

      Inserts a pixel value into the image with given PixelCoord.
\begin{verbatim}
   virtual void SetPixel(PixelCoord, float) = 0;

\end{verbatim}

      Returns the data value of a pixel having a  given PixelCoord.
      For a ListImage, it will return the value of the FIRST matching
      pixel in the list. If no matching pixel exists in the image,
      the method exits. 
\begin{verbatim}
   virtual float GetPixel(PixelCoord) const = 0;

\end{verbatim}

      Returns the data value of a Pixel having a specified ImPixel coordinate. 
      This function may not always be able to find a matching pixel, in which 
      case it returns FALSE. (There is no guarantee that an integral ImPixel
      coordinate will be an integral Pixel coordinate). In the case of a 
      "ListImage", it will return the FIRST matching pixel found in the list. 
\begin{verbatim}
   virtual int GetImPixelVal(const ImPixelCoord&, float&) const = 0;
 
   // Union image operators.
\end{verbatim}

      General UNION combine: Image C (*this) = factor1*Image A + 
      factor2*Image B . Image C can be Filled or ListImage type, while
      Images A and B can be any image type. Images A and B must be 
      conformant: same ImageUnits, same CoordSys, same ImPixStep and
      their "imcornpix" reference corners must be separated by an
      integral number of ImPixSteps. Image C will be made to be conformant
      with images A and B, and its history file (if any) will be repaced
      by that of A followed by B. The dimensions of C, and its reference
      corner, will be set to represent the true UNION of images A and B.
      Output pixels in C will be set to a "fill" value (only if C is
      a FilledImage) if there are no corresponding pixels in A or B.
      Otherwise, the output pixels in C will be = factor1*A (if pixel
      exists only in A), = factor2*B (if pixel exists only in B), 
      = factor1*A + factor2*B (if pixel exists in both). The method returns
      FALSE if the images are non-conformant.
\begin{verbatim}
  virtual int UCombine (float, const Image&, float, const Image&, float) = 0;

   // Intersection image operators.
\end{verbatim}

      General INTERSECTION combine: Image C (*this) = factor1*Image A + 
      factor2*Image B . Image C can be Filled or ListImage type, while
      Images A and B can be any image type. Images A and B must be 
      conformant: same ImageUnits, same CoordSys, same ImPixStep and
      their "imcornpix" reference corners must be separated by an
      integral number of ImPixSteps. Image C will be made to be conformant
      with images A and B, and its history file (if any) will be repaced
      by that of A followed by B. The dimensions of C, and its reference
      corner, will be set to represent the true INTERSECTION of images A and B.
      Output pixels in C will be set to a "fill" value (only if C is
      a FilledImage) if there is no corresponding pixels in both A and B.
      (This can only happen if A or B is a ListImage.) Otherwise, the output 
      pixels in C will be = factor1*A + factor2*B. The method returns
      FALSE if the images are non-conformant, or if there is no INTERSECTION.
\begin{verbatim}
   virtual int XCombine (float, const Image&, float, const Image&, float) = 0;

\end{verbatim}

      General scaled-add: Image A (*this) = Image A + factor*Image B. Image A
      can be Filled or ListImage type, while B can be any image type. Images A 
      and B must be conformant: same ImageUnits, same CoordSys, same ImPixStep 
      and their "imcornpix" reference corners must be separated by an
      integral number of ImPixSteps. The history file of Image B will be
      added to that of Image A. The scaled-add operation will take place only
      for pixels in A that have corresponsding pixels in B. The method returns
      FALSE if the images are non-conformant.
\begin{verbatim}
   virtual int ScaledAdd(float, const Image&) = 0;


   // Destructor.
   virtual  ~Image();

\end{verbatim}

\subsection*{PROTECTED DATA MEMBERS}

      Dimensions of image [x,y]
\begin{verbatim}
   ImageDim dim;
\end{verbatim}

      Image type identifier.
\begin{verbatim}
   K_String pImtype;
\end{verbatim}

      The ImPixel coordinates of the BLC and TLC of a region of interest.
\begin{verbatim}
   ImPixelCoord regblc, regtrc;
\end{verbatim}

      Image "centre", in pixel coordinates.  Whenever the image dimensions are 
      updated (at construction time or otherwise) this is set to the
      default value of (m/2, n/2-1) where the dimensions are [m,n]. The 
      "CenterImage()" method uses this parameter.
\begin{verbatim}
   PixelCoord cenpix;
\end{verbatim}

      ImPixel coordinates of Pixel coordinate (0,0), the top-left corner. This 
      defines the position of the image with respect to ImPixel (and hence 
      Image) coordinates. The default value of this is (1,n) when the 
      dimensions are [m,n], i.e. the BLC is (1,1).
\begin{verbatim}
   ImPixelCoord imcornpix;
\end{verbatim}

      Number of ImPixels per Pixel. This defines the ratio of ImPixel    
      coordinates to Pixel coordinates, and the default value is (1,-1),
      corresponding to the CIC convention of having Pixel coordinate (0,0) 
      at the TLC. When a new image is created by taking a sub-image of an 
      existing image, this may change.
\begin{verbatim}
   ImPixStep pixstep;
\end{verbatim}

      Data units for the pixel values in the image.
\begin{verbatim}
   ImageUnits  datatype;
\end{verbatim}

      Associated coordinate system. Note that a coordinate system must be 
      explicitly assigned (currently no constructor does this!) by inserting
      a pointer to it here.
\begin{verbatim}
   CoordSys *pCsys;
\end{verbatim}

      Associated history (pointer to Hist File object). This is usually
      assigned automatically by the constructor, and the destructor
      removes the storage.
\begin{verbatim}
   HistFile *pHist;
\end{verbatim}

\subsection*{PROTECTED MEMBER FUNCTIONS}
      Methods for combining images. These are solely for the use of the
      derived classes in implementing dynamically bound combine and
      scaled add functions. Essentially they compute the ``union'' or
      ``intersection'' areas in terms of x and y values of ImPixel coordinates.
\begin{verbatim}
    int CombTwoImages(int, const Image&, const Image&, float&, float&,
		float&, float&);
    int CombOneImage(const Image&, float&, float&, float&, float&);

\end{verbatim}

      Update the "centre pixel". This method is for the use of the base
      and derived classes, to update the "center" definition whenever the
      dimensions are changed.
\begin{verbatim}
   void UpdateCenter();
\end{verbatim}
