\clearpage
\section{ModelImage.h}

\subsection*{HEADER FILE DESCRIPTION}
 aips++ header
  
\subsection*{ENVIRONMENT}
\begin{verbatim}
#define A_MODELIMAGE_H

#include <assert.h>
#include <iostream.h>
#include "Image.h"

typedef float (*pFunc)(float *,ImPixelCoord);
\end{verbatim}

\subsection*{CLASS DESCRIPTION}
   Class: ModelImage
  
   Image with pixel values defined by an analytical function.
  
   An example of how to use ModelImages follows:
  
   If one wished to have a 128 x 128 ModelImage of a Gaussian function 
   centered on ImPixel coordinates (64,64) and having width paramters
   (5,5) and unity amplitude, the following is the procedure.
  
\begin{verbatim}
      float Gaussian(float   parms, ImPixelCoord imp);
          {
              float xim = imp.GetImPixCoord(1);
              float yim = imp.GetImPixCoord(2);
              float dx = (xim - parms[1])/parms[3];
              float dy = (yim - parms[2])/parms[4];
              double val = parms[0]*exp(-dx*dx - dy*dy);
              return (float)val;
           }
  
       float gparms[5] = {1,64,64,5,5};
  
  
       ModelImage mim(128,128);
       // Attach a coordinate system here!
       mim.SetFunction(Gaussian);
       mim.SetParmList(gparms);
               etc.
\end{verbatim}  
   NOTE: If one wished to have the Gaussian centred at given Pixel coordinates,
   one must (before any image operation) set gparms[1:2] to the ImPixel
   coordinates corresponding to the Pixel coordinates. This requires a call to 
   the Image method that makes this conversion.
  
\subsection*{CLASS SUMMARY}
\begin{verbatim}
class ModelImage  : public Image
{
public:
   ModelImage ( int m = 0, int n = 0, pFunc f = 0, float *p = 0,
                float sc = 1.0);
   ModelImage(const ModelImage& src);
   ModelImage &operator=(const ModelImage& src);
   int GetNumEl() const;
   void SetParmList(float * pp);
   void SetFunction(pFunc fp);
   void Scale(float scl);
   void Fill(float);
   int Extrema(Pixel &maxpix, Pixel &minpix) const;
   void SetPixel(PixelCoord, float);
   float GetPixel(PixelCoord) const;
   int GetImPixelVal(const ImPixelCoord&, float&) const;
   int UCombine (float, const Image&, float, const Image&, float);
   int ScaledAdd(float, const Image&);
   int XCombine(float, const Image&, float, const Image&, float);
 friend ostream& operator << (ostream&, ModelImage&);
   ~ModelImage();
};
\end{verbatim}

\subsection*{MEMBER FUNCTIONS}

      Make a ModelImage.
\begin{verbatim}
   ModelImage ( int m = 0, int n = 0, pFunc f = 0, float *p = 0, 
        float sc = 1.0);
\end{verbatim}

      Copy constructor
\begin{verbatim}
   ModelImage(const ModelImage& src);
\end{verbatim}

      Asignment operator
\begin{verbatim}
   ModelImage &operator=(const ModelImage& src);
\end{verbatim}

      Accessors
\begin{verbatim}
   int GetNumEl() const;
   void SetParmList(float * pp);
   void SetFunction(pFunc fp);
\end{verbatim}

      Data operations.
\begin{verbatim}
   void Scale(float scl);
   void Fill(float);
\end{verbatim}

      Find extrema. Returns FALSE if empty image.
\begin{verbatim}
   int Extrema(Pixel &maxpix, Pixel &minpix) const;
   void SetPixel(PixelCoord, float);
\end{verbatim}

      Return "pixel" value. It will return a model image value, even for
      non-integral values of PixelCoord.
\begin{verbatim}
   float GetPixel(PixelCoord) const;
\end{verbatim}

      Get pixel value, given ImPixel coordinates. Returns FALSE if the 
      resulting PixelCoord is non-integral.
\begin{verbatim}
   int GetImPixelVal(const ImPixelCoord&, float&) const;
\end{verbatim}

      Union image operator.
\begin{verbatim}
   int UCombine (float, const Image&, float, const Image&, float);
\end{verbatim}

      Scaled Add and Intersection image operator.
\begin{verbatim}
   int ScaledAdd(float, const Image&);
   int XCombine(float, const Image&, float, const Image&, float);
\end{verbatim}

      Output a ModelImage object. Note that a CoordSys must be defined!
\begin{verbatim}
friend ostream& operator << (ostream&, ModelImage&);
\end{verbatim}

      Destructor.
\begin{verbatim}
   ~ModelImage();
\end{verbatim}


