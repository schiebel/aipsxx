\chapter{Image handling}

\section{Basic approach}

        The development of classes for image handling for the prototype \aipspp
exercise was based on a few basic postulates:

\begin{itemize}
\item
image data values would be restricted to ``float", although it was
recognized that many types of images would eventually be required (complex,
double, int, ....).
\item
dimensionality would be restricted to 2, although the code was
written so that it could, without a great deal of effort, be expanded to more
dimensions.
\item
rudimentary capability for keeping track of pixel coordinates in
an astronomical frame of reference should be built in, but not developed in
any depth.
\item
rudimentary capability for keeping a history attached to an image
should be built in, but not developed in depth.
\item
the prototype classes should demonstrate some capabilities of
dynamic binding.
\item
some attempt should be made to assess the usefulness of the
CIC image classes.
\end{itemize}

        In the following, the design of the set of prototype IMAGE classes 
will be outlined, and some of the member functions for the various classes
will be briefly described.


\section{Types of images}

        One of the first assumptions made in the image area was that the
concept of an Image class would be restricted to sets of data values in
which each value has coordinates that can be mapped onto a grid or n-cube,
i.e. the coordinates can be represented by integers. Data having random
coordinate values therefore do not fall into the class that is defined by
``Image". Within the bounds of this definition of ``Image", it was apparent that 
there were needed (at least) three sub-classes of ``Image": images in which 
a full matrix of data values exist, images for which a list of pixel values and
coordinates exist, and images that are defined by an analytical function.
In the following, these will be referred to as ``FilledImage", ``ListImage",
and ``ModelImage". These will be described in considerable detail after the
generic attributes of the base ``Image" class are presented.



\section{Coordinate systems}

        One of the fundamental attributes of an Image is a coordinate system
description that allows the coordinates of any given pixel to be specified 
(in some user defined representation). For the two-dimensional case
treated in the prototype, a basic system of grid coordinates, aligned with
the ``rows and columns" of an image, and which can be uniquely related to
some ``physical" system of coordinates, has been assumed. Grid positions
in this system are called ImPixel coordinates, with a coordinate specification
being given by an ImPixelCoord object. Coordinate values in this system
have been treated as ``float" values, even though image pixels are located
at integral coordinate values. (More will be said about this later).
ImPixel coordinates are defined to increase from left to right across an
image (first value) and from the bottom to the top of an image (second
value). 

        The class ``CoordSys" was devised to specify the relationship
between ImPixel coordinates and some physical coordinate system, and also
to specify the representation in which the user wishes to have image
coordinates expressed. For example, an image may have intrinsic physical
coordinates such as (l,m) in an interferometric image, but the user would
like to access the image in Galactic coordinates. The coordinate system
selected by the user is termed Image coordinates, and coordinates in this
system are given by an ImageCoord object.

        The attributes for a CoordSys objects, for this exercise, consisted
of a set of parameters that can convert ImPixel coordinates to some intrinsic
or ``native" physical coordinate system, assuming that the two components of 
the native coordinates are orthogonal and separable (the standard AIPS 
convention).  The characteristics of this native coordinate system are 
specified in an object of the ``CoordSysType" class. For this exercise, this
is defined by a name, an epoch, and a set of four parameters. Similarly,
the user-specified Image coordinate system is defined by a CoordSysType
object. Conversions between the native physical coordinate system and
the user's Image coordinate system have not been implemented in the
prototype, but the parameters provided in the CoordSysType objects should
suffice for such methods.

        Thus a CoordSys object allows the following coordinate conversion,
and its inverse:
\begin{verbatim}
        ImPixelCoord --> ``Native Coords" --> ImageCoord  
\end{verbatim}
where the ``native" coordinate system is just a useful intermediary.


        A third (as defined by separate classes) coordinate system is the
internal coordinate system used within a given image. At first glance, it would
seem logical to use ImPixel coordinates. However, there are reasons to 
introduce a separate system, in order to minimize changes required to
CoordSys objects as images are manipulated. For example, if an image is 
defined by a ``window" which moves in ``astronomical space" and ImPixel
coordinates are used for internal image coordinates, the parameters in the
attached CoordSys object must be changed for each window location. Similarly,
if a new image is created by taking a sub-image (perhaps every n'th
pixel within a window area), the derived image must have a new CoordSys
object different from that of the parent image. Although all of this is
possible, it seemed simpler to introduce a third coordinate system and one
image attribute that links this system to ImPixel coordinates. This limits
the proliferation of CoordSys objects. The new system is referred to as
Pixel coordinates, and a set of coordinate values is given by a PixelCoord
object. For the prototype exercise, Pixel coordinates have their origin
(0,0) at the top-left corner of the image and the first value increases
from left to right, and the second from top to bottom.

If a sub-image is extracted from an image by selecting every
m'th pixel in the x direction and every n'th in the y direction, in the
resulting image each increment in Pixel coordinates will no longer
correspond to a unit increment in ImPixel coordinates. An image attribute
(object of the ImPixStep class) records this relation. Conversions
between Pixel coordinates and ImPixel coordinates are the responsibility
of Images.


\section{Image history}

        Images must carry with them some record of processing history. For
the prototype, a class HistFile was implemented, consisting only of a linked
list of strings (using CIC classes). Simple methods for listing entries in
the files, and for inserting entries, are provided. This is an area that
would require much more development for a practical system.



\section{Image base class and derived classes}

        The abstract base class Image contains data members which provide
linkages to a history file and to a coordinate system, descriptors giving
the type of file and data units, and several parameters describing the
relationship between internal Pixel coordinates and ImPixel coordinates.
Regions of interest (regions within which, and only within which, certain
image operations are to be performed) have not been incorporated in detail in
the prototype. One region of interest has been included, but in a practical
system, a list of regions of interest is probably required. Aside from the
parameters that define the Pixel coordinate - ImPixel coordinate relationship,
another useful parameter that has been introduced is the image ``center". This
is user-definable but defaults to Pixel coordinates of (m/2, n/2-1) where the 
dimensions are [m,n]. 


        The major member functions in the Image class deal with the following
functional requirements:

\begin{itemize}
\item
setting a pixel in the image, and retrieving a given pixel from an image.
\item
conversions between Pixel coordinates and ImPixel (or Image) coordinates,
and vice-versa.
\item
checking whether one image is ``conformant" with another, i.e. an add operation
can be performed on them without a chance of adding ``apples and oranges".
\item
adding (with weighting) two images to produce a third, either in the UNION
or INTERSECTION sense.
\item
scaling an image and adding it to another.
\item
adding entries to the image history, either individual entries or by copying the
history file of another image.
\item
adjusting an image's reference (TLC) position in ImPixel coordinates so
that the center of the image falls on a given ImPixel coordinate.
\item
finding the maximum and minimum pixel values in an image.
\end{itemize}

     The three derived classes of images are ``FilledImage", ``ListImage"
and ``ModelImage". The data members of these classes are not generally of
interest to the users of the class, so nothing more will be said of them
here. (See the header files for details). It should be stated, however, that
the CIC class libraries and templates have been used in the prototype to
implement linked lists and arrays. The major member functions of these classes
implement the following functionality (in addition to that presented by the base Image class).

{\Large\it FilledImage Class}
\begin{itemize}
\item
display image as a gray-scale using PGPLOT.
\item
provide access to the data storage for efficient mathematical operations.
Although this violates encapsulation to some extent, it may well be required
in any realistic system.
\item
allow a sub-image to be extracted from an image, creating a new image,
where there is considerable flexibility in the selection of the sub-image,
i.e. every i'th pixel in x and every j'th pixel in y. It is this operation 
which demonstrates the utility of having both Pixel and ImPixel coordinates, 
since the choice of i and j completely restructures the relationship of
Pixel coordinates and ImPixel coordinates, but the extracted sub-image and 
its parent can and do still have the identical CoordSys object.
\end{itemize}


{\Large\it ListImage Class}
\begin{itemize}
\item
be able to ``clone" itself, copying all attributes but zeroing the list of
pixels. This functionality might be required for FilledImages also but
was only implemented as a test for ListImages.
\item
various methods of adding or retrieving pixels in the list (by matching
Pixel coordinates, by serial number in the list, etc.). Special methods were
introduced to make use of the list iterator provided by CIC.
\item
the ability for the ``dimensions" of a ListImage to grow as new pixels, with
new Pixel coordinates, are added to the list.
\item
the ability to merge pixels in the list which may have the same Pixel
coordinates (useful for a list of CLEAN components). In a real system, this
might be accompanied by a sort operation.
\end{itemize}


{\Large\it ModelImage Class}
\begin{itemize}
\item
to allow the flexible specification of an analytical model of an image.
\item
to provide a flexible means of updating the parameters of the model.
\end{itemize}

\section{Image operations using dynamic binding}

        Some experience in applications of dynamic binding has been obtained.
Aside from the general pixel access routines, which are implemented as virtual
base class functions, the combined and scaled-add methods have been the best
examples of this. A statement of the type:

\begin{verbatim}
        Image C = 5 * Image A - 7 * Image B  (logical code)
\end{verbatim}

works correctly regardless of the image types (except that C cannot be a
ModelImage). The prototype has given some insight into the practicality
(not always) of such methods, and the requirements for implementing them.


\section{What has been learned from the prototype}

        The prototype has provided a lot of experience in using C++ and
C++ tools, but has also, we feel, indicated that the general framework
adopted for image handling is probably not too different from what one would
like for a practical system. The use of Pixel and ImPixel coordinate systems
has raised some  questions of overhead (and possible user confusion) but has
provided great flexibility. Part of the overhead arises because Pixel 
coordinates have two aspects: as indexes in image data arrays (where they must
be integers), and as computed counterparts of Image (or ImPixel) coordinates
(where they must usually be floating numbers). A practical system may have
to introduce a better way of meeting both these requirements, if it can be
done with lower overhead. Certainly, several methods that now take Pixel
coordinate arguments or return Pixel coordinates will have to be overloaded
to also take/return ImPixel coordinates. It is possible that an ImPixel
class (data value plus ImPixel coordinates) will have to join the current
Pixel class. If the dual-coordinate system is to be used successfully, the
client must be able to perform all image operations without ever bothering
with Pixel coordinates!
     The prototype has not been overly successful in testing the usefulness
of the CIC image classes, but only because multi-dimensional CIC arrays
of floating numbers, and associated display methods, are unavailable at this
moment.


\section{What was missing from the prototype}

	The prototype lacks dimensionality greater than two, generality
with regard to types of pixels, and no capabilities in the areas of regions
of interest and image errror data, amongst other things. The latter of these
needs careful analysis before the optimum implementation can be designed.

