\clearpage
\section{XYZ.h}

\subsection*{HEADER FILE DESCRIPTION}
 This file contains the definitions for the XYZ position class.
  
\subsection*{ENVIRONMENT}
\begin{verbatim}
#define XYZ_H

#include "Vector.h"
#include <assert.h>
\end{verbatim}

\subsection*{CLASS DESCRIPTION}
   The XYZ position class is used to store the location of various
   antennas. Positions are assumed to be in an equatorial system. That
   is, Y axis points east, Z axis is parallel to the pole, and the X
   axis to defined to make a right-handed orthogonal system. Units are
   assumed to be nanoseconds.

\subsection*{CLASS SUMMARY}
\begin{verbatim}
class XYZ
{
public:
    XYZ(int n, const double* myx, const double* myy, const double* myz);
    int Dim() const;
    double X(int i) const;
    double Y(int i) const;
    double Z(int i) const;
};
\end{verbatim}

\subsection*{MEMBER FUNCTIONS}
  
   Construct the antenna locations from arrays of the positions.
   Input are the antenna locations and the number of antenna.
\begin{verbatim}
    XYZ(int n,                  // Number of antennas.
      const double* myx,        // Antenna locations in X.
      const double* myy,        // Antenna locations in Y.
      const double* myz)        // Antenna locations in Z.
\end{verbatim}
  
   Function to return the number of antenna.
\begin{verbatim}
    int Dim() const;
\end{verbatim}

   Functions to return the locations of the various antenna.
\begin{verbatim}
    double X(int i) const;
    double Y(int i) const;
    double Z(int i) const;
\end{verbatim}
