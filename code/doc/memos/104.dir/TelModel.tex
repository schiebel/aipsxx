\clearpage
\section{TelModel.h}

\subsection*{HEADER FILE DESCRIPTION}
 This file contains definitions for the TelModel class.
  
\subsection*{ENVIRONMENT}
\begin{verbatim}
#define TELMODEL_H

#include <assert.h>
#include <math.h>
#include "String.h"
#include <complex.h>
#include "GainTable.h"
#include "YegSet.h"
#include "Telescope.h"
\end{verbatim}

\subsection*{CLASS DESCRIPTION}
 TelModel class.

\subsection*{CLASS SUMMARY}
\begin{verbatim}
class TelModel
{
public:
    TelModel(int, float, float, float);
    ~TelModel();
    void Update(YegSet&, YegSet&);
    YegSet& Apply(const YegSet&);
    complex InterpGain(int, float);
};
\end{verbatim}

\subsection*{MEMBER FUNCTIONS}
      
       Constructor for TelescopeModel specifying:
         number of receptors;
         time of first entry;
         time of last entry;
         time interval between entries.
\begin{verbatim}
    TelModel(int, float, float, float);
\end{verbatim}

       Destructor.
\begin{verbatim}
    ~TelModel();
\end{verbatim}

       Update this model given a pair (measured and predicted) of YegSets.
       This model solves for and updates its internal state.  The YegSets
       remain unchanged.
\begin{verbatim}
    void Update(YegSet&, YegSet&);
\end{verbatim}

       Create a new, corrected, YegSet and associated Telescope from an 
       existing YegSet by applying this model.  The preexisting YegSet 
       remains unchanged.  A copy of this model is attached to the
       new telescope as its default model.
\begin{verbatim}
    YegSet& Apply(const YegSet&);
\end{verbatim}
       
        Return an interpolated gain given:
          a receptor number;
          time.
        This simple performs a crude linear interpolation 
        between the two nearest entries in the gain table.
\begin{verbatim}
    complex InterpGain(int, float);
\end{verbatim}
