\chapter{UV data and imaging}

\section {Class descriptions}

The problem specified for the UV data and imaging group is essentially
as specified for the Green Bank meeting; {\it i.e.} the manipulation
of (for the prototype) visibility data, with particular regard to
calibration, and the formation of images from data.  Thus the classes
which have been defined and implemented closely resemble those
described in the Green Bank report, but given the very limited amount
of time available for this prototype, the design and implementation of
these is often rather different from what we envisage for a full-scale
system based on the Green Bank analysis.  In particular, the classes
do not exhibit the full functionality specified in the Green Bank
report, and their relationships have been somewhat simplified.
Nevertheless, the overall scheme is basically the same as that
decribed in the Green Bank report, although it is by no means a
meaningul test of that proposal.

A number of issues have been ignored and/or circumvented.  In
particular, persistence is neither required nor implemented in the
prototype.

The fundamental classes are described below, and the way in which
these differ from the corresponding classes in the Green Bank report
are indicated.  More detailed descriptions of the classes are given in
the header files.

\begin{description}

  \item[YegSet and IntYegSet -] These are the representations of the
       data themselves.  The full functionality of YegSet as described in the
       Green Bank report is not implemented.  We have chosen to deal with
       data in bulk, as YegSets, rather than as individual Yegs.  In
       addition, the selection and sorting operations have not been
       implemented, although these are still regarded as essential to a real
       system.

  \item[Telescope -] As described in the Green bank report, this is essentially
       a crude means of associating various kinds of objects which are
       related in some way.  In our small prototype, this functionality
       is largely redundant, since we can maintain the relationships 
       in a ``hard-wired'' form, and the use of this is not really explored.

  \item[Telescope Model -] This is a model of all attributes of the
       observing telescope which are required in calibration/self-calibration
       of YegSets, together with methods for updating the model
       {\it e.g., } determining gain solutions and applying corrections
       to the data on the basis of the new attributes.
       In the prototype, this models a simple interferometer using
       only complex gains for each receptor.

       
  \item[ImagingModel/IntImagingModel -] This is an imaging model for 
       interferometer data.  It performs a Fourier transform 
       on a YegSet to produce an image which represents the
       sky brightness distribution.

\end{description}

In addition to the principal classes, a number of subsidiary classes
have also been developed from an analysis of the individual classes
described above.  Briefly, these are:

\begin{description}

  \item[RVector -] Real vector;

  \item[DVector -] Double vector;

  \item[CVector -] Complex vector;

  \item[AssArray -] Associative array;

  \item[GainTable -] Complex table;

\end{description}
Whilst these classes have been developed specifically to serve the
requirements of the application-related classes, they are clearly
likely to be generally useful.  The Vector classes used to implement
YegSets, could also be used to implement the GainTable in a way which
might be more efficient for operations such as applying complex gains
to a YegSet.


\section {Design issues, problems and lessons}

A number of simplifications to the Green Bank model have been
mentioned, and in most cases we do not envisage that these would be
present in the ultimate design.  A number of problems which arise if
these simplifications are not made will be discussed shortly, and
these will have to be addressed.  However, one change which is likely
to remain is that we may wish to work with YegSets (rather than
individual Yegs) in many most cases.  In the prototype, we have
assumed that all Yegs in a YegSet are associated with the same
Telescope, TelescopeModel and ImagingModel, and we believe it may be
more convenient (as well as efficient) to make this assumption.  We
should still be able to cope with data from multiple telescopes, but
we should deal with them as sets of YegSets.

One of the most important problems arose out of design exercise which
immediately preceeded the implementation of the prototype.  The Green
Bank scheme requires many kinds of objects to be associated together,
and proposed that this be implemented by using pointers or references
in one object to refer to another.  It is often the case that such
associations may involve classes derived from those for which the
pointers are defined, and may require a cast down to the derived type
in order to make use of methods which are not present in the base
class.  This problem appears to be quite common, and it is almost
certainly the case that any scheme other than the Green Bank proposal,
which has the degree of flexibility we require, will also suffer from
this problem.  Put simply, relying solely on the static typing of C++
may restrict the kind of dynamic association of different kinds of
objects which is likely to be essential.

This presents no problem in the simplified prototype, but must be
addressed in a generally extensible system.  Whilst it would be
inappropriate to discuss solutions to this problem in this report, we
should say that it is not insurmountable, and need not affect the
fundamental classes and their interrelationships.  The most obvious
difference might be the need to maintain associations between objects
using some entity external to the objects, rather than by using
pointers or references within the objects themselves.  This external
entity might be a simple table containing ``handles'' for the objects
to be associated, or might be a more complicated database-like system,
possibly part of the Project system proposed in the Green Bank report.

The prototyping exercise has confirmed that under some circumstances,
encapsulation may be an obstacle to efficiency.  For example, if we
wish to perform some arbitrary operation on all the elements of a
vector or multi-dimensional array, this is best performed with direct
access to the array itself.  It seems unlikely that it will always be
possible to implement such operations as methods of the class, and it
might be argued that allowing applications programmers to routinely
modify such fundamental classes as vectors and images is worse than
introducing methods which present an array of raw data to the
application.  The latter might be regarded as a violation of
encapsulation, but could be performed with some degree of control by
the class itself.  The appropriate solution to this problem is not
clear; there are a number of possibilities, most of which will have
important consequences for the ultimate design of our classes.

The analysis and design of high level application-oriented classes
leads to new requirements which are best implemented as a lower layer
of classes.  The Vector and GainTable classes are examples of these;
with a little modification, they could serve a variety of requirements
from many high-level classes.  It is clear that there is considerable
scope for reuse of objects at this level.  This will require a great
deal of interaction between designers of high level classes in order
to specify requirements for utility classes which can be used
commonly.  However, this is likely to result in a class library which
is better tailored to the needs of the system, as compared to one
which is built out of classes which have not been designed with a
particular need in mind, and thus often turn-out to be a poor fit to
the requirements.

