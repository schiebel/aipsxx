\clearpage
\section{IntYegSet.h}

\subsection*{HEADER FILE DESCRIPTION}
 This file contains the definitions for the IntYegSet class.
  
\subsection*{ENVIRONMENT}
\begin{verbatim}
#define INTYEGSET_H

#include "YegSet.h"
\end{verbatim}

\subsection*{CLASS DESCRIPTION}
   The IntYegSet class provides services to return telescope, data, weights
   and coordinates.

\subsection*{CLASS SUMMARY}
\begin{verbatim}
class IntYegSet:public YegSet
{
public:
    IntYegSet(const YegSet&);
    DVector u() const;
    DVector v() const;
    DVector w() const;
    DVector time() const;
    DVector r1() const;
    DVector r2() const;
    virtual void ShowYegSet (int i1, int i2);
};
\end{verbatim}

\subsection*{MEMBER FUNCTIONS}

    Construct an IntYegSet from an YegSet.
    Default copy constructor should also work OK.
\begin{verbatim}
    IntYegSet(const YegSet&);
\end{verbatim}

   Return u,v,w coordinates, time and receptor numbers.
\begin{verbatim}
    DVector u() const;     // u in wavelengths.
    DVector v() const;     // v in wavelengths.
    DVector w() const;     // w in wavelengths.
    DVector time() const;  // time in seconds.
    DVector r1() const;    // Number of ant1.
    DVector r2() const;    // Number of ant2.
\end{verbatim}

   Display
\begin{verbatim}
    virtual void ShowYegSet (int i1, int i2);
\end{verbatim}

