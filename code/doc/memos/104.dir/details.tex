\chapter{Details}
This chapter is just a bag of tricks. The items may not belong in this report,
but their inclusion at least ensures that they do not get lost.

\section{Object-state checking}
{\it Brian Glendenning}

It is useful, especially when the code is being debugged, to be able
to check the ``state'' of an object (in Eiffel this is called checking
the invariants). It is useful to write a member function {\tt int Ok()} that
performs such a check and returns {\tt 0} if not successful, {\tt 1} otherwise
(or boolean values). This function should be virtual for obvious reasons,
and a subclass that redefines {\tt Ok()} should probably within it also
refer to its parent classes {\tt Ok()}, e.g., refer to {\tt Parent::Ok()} to
ensure that the parent class is also consistent.

Conceptually you want to check the state of the object after
construction, and at entry to every member function. However this
state checking could be computationally expensive (e.g., checking all
the pointers in a complicated data structure) so it must be possible
to compile it out, preferably on a class by class basis, for the
production system. A way of doing this is to use the macro {\tt OK} rather
than the function {\tt Ok()} (obviously the latter is still available). Then
the macro {\tt OK} can be defined to be nothing, or can be defined to be
something like {\tt assert(Ok())} (the assert should be replaced with the
exception handling mechanism when available). Every {\tt .h} file for
classes which use the {\tt Ok()} mechanism should have OK default to
something, e.g. if you wanted to check state by default:

\begin{verbatim}
#ifndef OK
#define OK assert(Ok())    // Use exceptions when available
#endif /* OK */
\end{verbatim}

This can be overridden at compile time (probably in the makefile).


\section{Efficient indexing in vectors}
{\it Bob Sault}

The three vector classes RVector, DVector and CVector (for real, double
precision and complex) have methods defined on them to do all the normal
arithmetic operations (addition, subtraction, multiplcation, division),
and functions (sin, cos, etc), as well as interconversion. They use a 
reference counting scheme, and a copy-on-modify policy to reduce the
about of copying. 

The implementation of the indexing operator which returns a data element
of the array (i.e operator {\tt []}) is noteworthy. Convention dictates
that this should act as either an lvalue or an rvalue. 
Having this operator always return a reference to a data-element would have
disadvantages. As the operator would not know whether it is being used as an
lvalue or an rvalue, it would have to assume the worse -- an lvalue -- which
would significantly reduce the effectiveness of the reference counting and
copy-on-modify policy. The way used to avoid this loss is to have two
overloaded versions of the indexing operator, defined (for the real vector
class) as

\begin{verbatim}
  float operator[](int) const;

and

  float& operator[](int);
\end{verbatim}

The first version can only appear as an rvalue, and the compiler will
use this in preference whenever the vector is of const type. The second
version can appear as both an lvalue and an rvalue, and the compiler will
use this for non-const vectors. Provided the programmer uses the const
declaration wherever possible, the effectiveness of the reference
counting, etc is maintained. For a multiply-referenced, non-const vector,
then a copy will have to be made sooner or later anyway, so it does not
matter what this is initiated by the indexing operator (even in instances
where it is used as an rvalue).

Coplien discusses an alternate scheme (pp. 50-52) which would be applicable
here. Coplien's scheme is quite a bit more expensive -- a very undesirable
characteristic in such a common operation as indexing. It also has the
disadvantage that the indexing operator would always be a non-const method.



