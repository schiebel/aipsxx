\def\~{{\char'176}}

\chapter{System management}
\bigskip

   With the dissemination of two papers, {\it System management for \aipspp -
Part 1: organization and distribution} and {\it Part 2: activation,
generation, and verification} the design of \aipspp system management is now
well advanced.  The final part of the trilogy {\it Part 3: networking} is
scheduled for release by Jun/30, and will cover the area of network services.

   Implementation of the system design has been driven by necessity.  Creation
of the empty \aipspp directory tree was a trivial operation, belying a great
deal of thought which had been put into its design.

   Code management has, in the first instance, been implemented by using
{\it RCS}.  Each of the code directories has an {\it RCS} repository attached
to it.  Plain-text copies of the code are kept in the code area itself.
Programmers can create their own "shadow" representation of the \aipspp code
directory tree by using the {\it mktree} utility which creates symbolic links
to the {\tt \~aips++} {\it RCS} directories.  Programmers then appear to have
their own private workspace with a window into the master {\it RCS}
repositories, and can check code in and out of the {\it RCS} repository as
though it were their own.  This mechanism has served us extremely well.

   A generic GNU makefile works together with the {\it RCS} mechanism
described above to compile classes, class test programs, and the kernel
library, and also has several other functions.  With a dozen programmers
contributing 1000 lines of code per day on average, the system has grown in
complexity at an accelerating rate, and the makefile is now proving to be
indispensible.  The makefile allows programmers to compile code without
having to check it out of {\it RCS} and thereby minimizes the number of files
that need to be present in their private workspace, with the consequent
possibility that these may be "stale".  It uses the search path mechanisms
which are part of GNU {\it make}, searching for files first in the
programmer's own directory, then in the standard \aipspp directories.
However, although the makefile is logically correct, it is not particularly
efficient in shirking unnecessary work.  In particular, it recompiles a class
implementation file if any header file has changed.  A later generation should
be able to do better.

   \aipspp programmers may now define their \aipspp "environment" by means of
an {\tt aipsinit.[c]sh} script.  This redefines the {\tt PATH} (and
{\tt MANPATH}) environment variables, appending the \aipspp binary (and man
page) directories to it.  It also defines a single environment variable,
{\tt AIPSPATH}, which contains five space-separated character strings which
define the root of the \aipspp directory tree, the host architecture, the
\aipspp version, the local site name, and the local host name.  This
information is fundamental and must be known in order to access the
{\tt aipsrc} databases.

   The {\tt aipsrc} databases have been implemented via a {\tt C}-program
called {\it getrc}.  It looks for device and other definitions in a format
similar to that of the {\it .Xdefaults} database used by {\tt X-windows}.  In
resolving a reference it searches the following sequence of {\tt aipsrc}
files:

\begin{verbatim}
      ~/.aipsrc
      $AIPS/$ARCH/$VERS/$SITE/$HOST/aipsrc
      $AIPS/$ARCH/$VERS/$SITE/aipsrc
      $AIPS/$ARCH/$VERS//aipsrc
\end{verbatim}

The last of these files contains default values, and the other three allow
these to be overridden on a user-, host-, and site-specific basis.

   The first use to which the {\tt aipsrc} mechanism has been put is that of a
simple, and easily configurable set of printer utilities, including a utility
to print class header and implementation files in a compact and convenient
form. 


