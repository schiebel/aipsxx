\clearpage
\section{GridTool.h}

\subsection*{HEADER FILE DESCRIPTION}
  
\subsection*{ENVIRONMENT}
\begin{verbatim}
#define GRIDTOOL_H

#include <IntYegSet.h>
#include <DummyImage.h>

typedef double  DFunction(double);
double SincExp(double);
\end{verbatim}

\subsection*{CLASS DESCRIPTION}

\subsection*{CLASS SUMMARY}
\begin{verbatim}
class GridTool
{
public:
   GridTool(DFunction *CF=SincExp,int w=1,int n=6);
   ~GridTool();
   void SetConvFunc(DFunction *CF);
   void SetWeightingScheme(int w);
   void SetSupportSize(int i);
   virtual int Resample(const IntYegSet &, DummyImage &);
   virtual int Resample(const DummyImage &, IntYegSet &); 
   virtual int Resample(const IntYegSet &, DummyImage &, DummyImage &);
};
\end{verbatim}

\subsection*{MEMBER FUNCTIONS}
        Constructor
\begin{verbatim}
   GridTool(DFunction *CF=SincExp,int w=1,int n=6);
\end{verbatim}

        Destructor
\begin{verbatim}
   ~GridTool();
\end{verbatim}

        Services
\begin{verbatim}
   void SetConvFunc(DFunction *CF);
   void SetWeightingScheme(int w);
   void SetSupportSize(int i);
\end{verbatim}

 The Gridding and De-gridding services.  
 The approprate algorithm is invoked by the ordering of the 
 argumensts.

\begin{verbatim}
 Resample(IntYegSet, DummyImage) ==> griddiding from VIS -> Image
 Resample(DummyImage, IntYegSet) ==> de-gridding from Image -> VIS

   virtual int Resample(const IntYegSet &, DummyImage &);
   virtual int Resample(const DummyImage &, IntYegSet &); 
\end{verbatim}

 Gives grid for PSF as well
\begin{verbatim}
   virtual int Resample(const IntYegSet &, DummyImage &, DummyImage &);
\end{verbatim}
