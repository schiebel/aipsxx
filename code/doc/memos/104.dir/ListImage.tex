\clearpage
\section{ListImage.h}

\subsection*{HEADER FILE DESCRIPTION}
 Header file for ListImage class
  
\subsection*{ENVIRONMENT}
\begin{verbatim}
#define LISTIMAGE_H

#include <iostream.h>
#include "Image.h"
#include "K_DList.h"
#include "Pixel.h"
#include "ModelImage.h"
#include "FilledImage.h"

DECLARE_ONCE K_DList<Pixel>;

class FilledImage;        // Needed for proper declaration order
class ModelImage;
\end{verbatim}

\subsection*{CLASS DESCRIPTION}
   The ListImage class: pixels (values plus coordinates) are stored
   as a linked list.

\subsection*{CLASS SUMMARY}
\begin{verbatim}
class ListImage : public Image
{
public:
        ListImage(int m = 0, int n = 0);
        ~ListImage();
        ListImage(const ListImage& src);
        ListImage(const ModelImage& src);
        ListImage(const FilledImage& src);
        ListImage & operator = (const ListImage&);
        ListImage & operator = (const ModelImage&);
        ListImage & operator = (const FilledImage&);
        int GetNumEl() const;
        void Scale(float);
        void Fill(float);
        void FillPixel(int, float);
        int Extrema(Pixel &, Pixel &) const;
        void SetPixel(PixelCoord, float);
        int PixelExists(int) const;
        int PixelExists(PixelCoord) const;
        float GetPixel(PixelCoord) const;
        int GetPixel(PixelCoord, float&) const;
        void DeletePixel(int);
        Pixel GetPixel(int) const;
        void SetToPix(int) const;
        Pixel GetNxtPixel() const;
        int GetImPixelVal(const ImPixelCoord&, float&) const;
        int GetFirstNeg() const;
        ListImage CloneEmpty() const;
        void SortMerge(int);
 friend ostream& operator << (ostream&, ListImage&);
        int ScaledAdd(float, const Image&);
        int UCombine(float, const Image&, float, const Image&, float);
        int XCombine(float, const Image&, float, const Image&, float);
};
end{verbatim}

\subsection*{MEMBER FUNCTIONS}

           Constructor and destructor
\begin{verbatim}
        ListImage(int m = 0, int n = 0);
        ~ListImage();
\end{verbatim}

           Copy constructors
\begin{verbatim}
        ListImage(const ListImage& src);
        ListImage(const ModelImage& src);
        ListImage(const FilledImage& src);
\end{verbatim}

           Assignment operators
\begin{verbatim}
        ListImage & operator = (const ListImage&);
        ListImage & operator = (const ModelImage&);
        ListImage & operator = (const FilledImage&);
\end{verbatim}

           Return number of elements
\begin{verbatim}
        int GetNumEl() const;
\end{verbatim}

           Scale all pixel values by a scaling factor
\begin{verbatim}
        void Scale(float);
\end{verbatim}

           Replace values of all existing pixels by a new value.
\begin{verbatim}
        void Fill(float);
\end{verbatim}

           Fill the value of the n'th pixel with a new number.
           n must be a valid serial pixel number (0 < n  <= 
           number of elements)
\begin{verbatim}
        void FillPixel(int, float);
\end{verbatim}

           Return maximum and minimum pixels in image.
           Returns FALSE if empty image.
\begin{verbatim}
        int Extrema(Pixel &, Pixel &) const;
\end{verbatim}

           Set a pixel (add a pixel to the list)
\begin{verbatim}
        void SetPixel(PixelCoord, float);
\end{verbatim}

           Check that pixel of serial number i exists.
\begin{verbatim}
        int PixelExists(int) const;
\end{verbatim}

           Check that a pixel having given Pixel Coords exists.
\begin{verbatim}
        int PixelExists(PixelCoord) const;
\end{verbatim}

           Retrieve the value of a pixel (scan through list). The value
           of the FIRST pixel with matching coordinates is returned. If
           there is no matching pixel, or if the image is empty, the
           method exits.
\begin{verbatim}
        float GetPixel(PixelCoord) const;
\end{verbatim}

           Retrieve the value of a pixel (scan through list). The value
           of the FIRST pixel with matching coordinates is returned.
           The method returns FALSE if the image is empty or no
           matching pixel is found, otherwise TRUE.
\begin{verbatim}
        int GetPixel(PixelCoord, float&) const;
\end{verbatim}

           Delete the n'th pixel in the list. The serial pixel number
           n must be > 0 and <= the number of pixels.
\begin{verbatim}
        void DeletePixel(int);
\end{verbatim}

           Retrieve the n'th pixel in the list. The serial pixel number
           n must be > 0 and <= the number of pixels.
\begin{verbatim}
        Pixel GetPixel(int) const;
\end{verbatim}

           Prepare to extract pixels having serial number n (and
           greater). This is a faster access method than the
           preceding. The number n must be in range 0 < n <= number
           of pixels in image.
\begin{verbatim}
        void SetToPix(int) const;
\end{verbatim}

           Get next pixel. Must be preceded by invocation of itself,
           or SetToPix(). Exits if one has run off end of
           pixel list!
\begin{verbatim}
        Pixel GetNxtPixel() const;
\end{verbatim}

           Return value of FIRST pixel having ImPixelCoords. If there is no
           matching pixel, it returns FALSE, otherwise TRUE.
\begin{verbatim}
        int GetImPixelVal(const ImPixelCoord&, float&) const;
\end{verbatim}

           Get serial pixel number of first negative pixel. Returns 
           zero for empty image or no negative pixels.
\begin{verbatim}
        int GetFirstNeg() const;
\end{verbatim}

           "Clone" a new ListImage that has an empty list but all
           other attributes of the current image
\begin{verbatim}
        ListImage CloneEmpty() const;
\end{verbatim}

           Merge pixels with corresponding coordinates (i = 0), and
           merge and then sort into decreasing order of pixel value
           (i = 1). The latter is not yet implemented.
\begin{verbatim}
        void SortMerge(int);
\end{verbatim}

           Print out a ListImage object. Note that a CoordSys must be defined!
\begin{verbatim}
 friend ostream& operator << (ostream&, ListImage&);
\end{verbatim}

           Union and intersection combine and scaled-add.
           All methods used to implement these must be virtual methods!
\begin{verbatim}
        int ScaledAdd(float, const Image&);
        int UCombine(float, const Image&, float, const Image&, float);
        int XCombine(float, const Image&, float, const Image&, float);
\end{verbatim}

