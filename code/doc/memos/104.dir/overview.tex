\chapter{Overview}

\section{Why a prototype?}

In early February a group met in Green Bank. This group consisted of
six of the \aipspp development group from Charlottesville and six
others (from NRAO, NFRA and NRAL). The goal of this group was to
analyse the process of calibration and imaging astronomical data, in
particular radio-astronomical data. Most participants were satisfied
with the results from that meeting, and their report is summarized in
\cite{key1}. One of the participants has very strong disagreements with
\cite{key1}, his disagreements and a proposal to start afresh are outlined
in \cite{key2}.

After the Green Bank meeting and followup discussion in
Charlottesville (including Andrew Klein, our OO/C++ consultant) it was
decided to write a small prototype for the following reasons:
\begin{itemize}
\item
To attempt to see if there was anything grossly wrong with the ``Green
Bank'' model (it is fair to point out that a small prototype may
inadequately push the boundaries of the model).
\item
For the implementation group to get some hands-on experience with
programming radio-astronomical applications in C++. Although the group
was supposed to arrive in Charlottesville familiar with C++, the
experience was largely with small personal ``toy'' programs.
\item
To see how well the individual groups could communicate and provide
services for one another. This is critical if a distributed project
like \aipspp is to succeed.
\end{itemize}

\section{What prototype?}

The intention of the prototype was to build a system that was ``broad
and shallow'' rather than one that was ``narrow but deep''. The
consensus was that it was most appropriate to attempt in some small
way many parts of the ultimate system (also, as a practical matter it
is easier to get a lot of people working on a wide problem).

The chosen problem was to apply an antenna based calibration to a UV
dataset, image it, and display it using a simple ``keyword=value''
based command line interface.  Details of the prototype are explained
more fully in the following chapters and appendices.

\section{How the prototype was built}

The prototype was finished in just under three weeks and consisted of
about 15,000 lines of code and comments. We built our prototype on top
of the CIC container classes and with the GNU String class.

The team was split up into groups to implement the prototype.
The {\em Raw Data} group\footnote
{Dave Shone, Sanjay Bhatnagar, Bob Hjellming, Mark Holdaway, and Bob Sault}
was responsible for forming/calibrating and imaging UV data,
the {\em Image} group\footnote
{Lloyd Higgs, Mark Calabretta, and Brian Glendenning}
was responsible for creating image classes and simplified coordinate systems,
the {\em User Interface} 
group\footnote{Friso Olnon and Peter Teuben}
was responsible for creating a parameter passing mechanism,
the {\em Fundamental Libraries} group\footnote
{Bob Payne and Mark Stupar}
provided some support with support classes,
and the {\em Organization} group\footnote{Mark Calabretta}
was responsible for creating
appropriate makesfiles and directory structures.

The group was fairly enthusiastic about the prototype. In fact, doing
it was enough fun so that it was hard to stop work on it when it
reached its predefined limits. The initial few days of design were
extremely useful, especially the exercises with the CRC (Class -
Responsibility - Collaborator) cards. The implementation went pretty
well; there was some time wasted getting the ObjectCenter (Saber)
programming environment to work properly with the preprocessor that is
used by CIC to emulate C++ parametric types. A somewhat surprising
result was that the software environment --- makefiles, source code
checkin systems, etc.  --- was extremely important, even for this
small prototype with people in only two offices.

\section{Some preliminary conclusions}

The user interface group should probably have been given a larger
problem, perhaps a simple image arithmetic program. The groups were
only moderately successful at communicating with each other. While it
is desirable to keep communications costs down, we likely erred too
strongly on the side of working independently.

Some flaws in the Green Bank model were found and discussed during the
implementation of the prototype. The consensus of some internal email
\cite{key3} was that some decoupling of entities was required,
undoubtedly by associating objects explicitly inside a database. This
should also be attractive for end-users of the system.
