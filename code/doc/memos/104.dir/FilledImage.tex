\clearpage
\section{FilledImage.h}

\subsection*{HEADER FILE DESCRIPTION}
   This file contains the interface to the FilledImage class.
  
\subsection*{ENVIRONMENT}
\begin{verbatim}
#define FILLED_IMAGE_H

#include "Array2d.h"
#include "Image.h"
#include <assert.h>
#include <iostream.h>
#include "ListImage.h"
#include "ModelImage.h"

DECLARE_ONCE Array2d<float>;
DECLARE_ONCE Array2d<double>;

class ModelImage;
class ListImage;
\end{verbatim}

\subsection*{CLASS DESCRIPTION}
   The FilledImage class is a two dimensional image (of float) class with
   access via the various coordinate type.

\subsection*{CLASS SUMMARY}
\begin{verbatim}
class FilledImage  : public Image
{
public:
    FilledImage();
    FilledImage(int m, int n, float v=0.0);
    FilledImage(const FilledImage& src);
    FilledImage(const ListImage& src, float);
    FilledImage(const ModelImage& src);
    FilledImage &operator=(const FilledImage& src);
    FilledImage &operator=(const ModelImage& src); 
    int GetNumEl() const;
    Array2d<double> GetStorage();
    void SetStorage(const Array2d<double> &);
    void Scale(float);
    void Fill(float);
    int Extrema(Pixel &maxpix, Pixel &minpix) const;
    void SetPixel(PixelCoord, float);
    float &operator() (int,int);
    float GetPixel(PixelCoord) const;
    int GetImPixelVal(const ImPixelCoord&, float&) const;
    virtual int UCombine (float, const Image&, float, const Image&, float);
    virtual int XCombine (float, const Image&, float, const Image&, float);
    virtual int ScaledAdd (float, const Image&);
    FilledImage SubImage(PixelCoord &, ImPixStep &, ImageDim &) const;
    void Display();
    void Write(Char *File);
    ~FilledImage();
    friend ostream &operator<<(ostream &os, const FilledImage &im);
};
\end{verbatim}

\subsection*{MEMBER FUNCTIONS}

       Default constructor makes 0 sized image.
\begin{verbatim}
    FilledImage();
\end{verbatim}

       Make a FilledImage of a given size (possibly with a set value)
\begin{verbatim}
    FilledImage(int m, int n, float v=0.0);
\end{verbatim}

       Copy constructors 
\begin{verbatim}
    FilledImage(const FilledImage& src);
\end{verbatim}

       Copy constructor (from ListImage) - needs "fill" value
\begin{verbatim}
    FilledImage(const ListImage& src, float);
\end{verbatim}

       Copy constructor (from ModelImage)
\begin{verbatim}
    FilledImage(const ModelImage& src);
\end{verbatim}

       Assignment operators
\begin{verbatim}
    FilledImage &operator=(const FilledImage& src);
\end{verbatim}

       The following will fail if the existing FilledImage does not
       have enough storage allocated!!
\begin{verbatim}
    FilledImage &operator=(const ModelImage& src); 
\end{verbatim}

       Accessors
\begin{verbatim}
    int GetNumEl() const;
\end{verbatim}

       Put all the values in memory, which this function allocates
\begin{verbatim}
    Array2d<double> GetStorage();
\end{verbatim}

       Fill the image with the values in memory, optionally delete that memory
\begin{verbatim}
    void SetStorage(const Array2d<double> &);
\end{verbatim}

      Data operations.
\begin{verbatim}
   void Scale(float);
   void Fill(float);
   int Extrema(Pixel &maxpix, Pixel &minpix) const;
   void SetPixel(PixelCoord, float);
\end{verbatim}

      Set/get pixel values by direct indexing
\begin{verbatim}
   float &operator() (int,int);
\end{verbatim}

      If PixelCoord is non-integral or lies outside image range,
      method exits.
\begin{verbatim}
   float GetPixel(PixelCoord) const;
\end{verbatim}

      The following returns FALSE if corresponding PixelCoord is non-
      integral or lies outside image.
\begin{verbatim}
   int GetImPixelVal(const ImPixelCoord&, float&) const;
\end{verbatim}

      Union/Intersection operators
\begin{verbatim}
  virtual int UCombine (float, const Image&, float, const Image&, float);
  virtual int XCombine (float, const Image&, float, const Image&, float);
  virtual int ScaledAdd (float, const Image&);
\end{verbatim}

      Extract sub-image with the TLC specified (as a Pixel coordinate), the 
      step in pixels specified (by an ImPixStep object), and desired dimension 
      specified. Exits if no image is possible (e.g. TLC doesn't lie within
      the parent image). Desired dimension will shrink to get maximum sub-image
      allowed.
\begin{verbatim}
   FilledImage SubImage(PixelCoord &, ImPixStep &, ImageDim &) const;
\end{verbatim}

      Display Function (uses PGPLOT to present a grey-scale representation
      of the image)
\begin{verbatim}
   void Display();
\end{verbatim}

      Write Function (ASCII)
\begin{verbatim}
   void Write(Char *File);
\end{verbatim}

      Destructor.
\begin{verbatim}
   ~FilledImage();
\end{verbatim}

      Output an image's attributes and contents. Note that a CoordSys must
      be defined.
\begin{verbatim}
   friend ostream &operator<<(ostream &os, const FilledImage &im);
\end{verbatim}
