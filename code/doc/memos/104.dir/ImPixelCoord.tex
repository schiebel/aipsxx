\clearpage
\section{ImPixelCoord.h}

\subsection*{HEADER FILE DESCRIPTION}
 Header file for ImPixelCoord class
  
\subsection*{ENVIRONMENT}
\begin{verbatim}
#define A_IMPIXELCOORD_H

#include <assert.h>
#include <iostream.h>
#include "ImageCoord.h"
\end{verbatim}

\subsection*{CLASS DESCRIPTION}
          This class defines the n-dimensional (currently 2-dimensional)
   coordinates of an image, in the pixel frame of reference used by a
   CoordSys object. The x axis increases left to right and the y axis
   from bottom to top. Although usually only integral coordinate values
   are used with reference to an image, they are stored as floats for
   generality. (Coordinate conversions from astronomical coordinates can
   return non-integral values!) 

\subsection*{CLASS SUMMARY}
\begin{verbatim}
class ImPixelCoord
{
public:
        void SetImPixCoord(int i, float val);
        void SetImPixCoord(float*);
        float GetImPixCoord(int i) const;
        float* GetImPixCoord() const;
        ImPixelCoord operator += (const ImPixelCoord &);
        ImPixelCoord operator -= (const ImPixelCoord &);
        ImPixelCoord();
        ImPixelCoord(float i, float j);
        ~ImPixelCoord();
};
ostream& operator << (ostream& os, const ImPixelCoord ic);
ImPixelCoord operator + (const ImPixelCoord&, const ImPixelCoord&);
ImPixelCoord operator - (const ImPixelCoord&, const ImPixelCoord&);
ImPixelCoord operator * (const ImPixelCoord&, const float);
ImPixelCoord operator * (const float, const ImPixelCoord&);
ImageCoord operator * (const ImPixelCoord&, const ImageCoord&);
ImageCoord operator * (const ImageCoord&, const ImPixelCoord&);
ImPixelCoord operator / (const ImageCoord&, const ImageCoord&);
\end{verbatim}

\subsection*{MEMBER FUNCTIONS}
      
           Set the i-axis coordinate to value: val. The value of i must
           be greater than 0 and less than 3 (currently)
\small\begin{verbatim}
       Preconditions:
                  valid axis: 0 < i  < 3
          
       Postconditions:
                  invalid input parameters: assertion exit
\end{verbatim}\normalsize
\begin{verbatim}
        void SetImPixCoord(int i, float val);
\end{verbatim}

           Set the coordinate values to the array of float values given by
           the float pointer.
\small\begin{verbatim}
       Preconditions:
                  valid number of values: 2
          
       Postconditions:
                  invalid input parameters: assertion exit
\end{verbatim}\normalsize
\begin{verbatim}
        void SetImPixCoord(float*);
\end{verbatim}

           Get the value of the i'th axis coordinate. The value i must be
           greater than 0 and less than 2.
\small\begin{verbatim}
       Preconditions:<
                  valid axis: 0 < i  < 3
          
       Postconditions:
                  true
\end{verbatim}\normalsize
\begin{verbatim}
        float GetImPixCoord(int i) const;
\end{verbatim}

           Return the coordinate values by a pointer
\small\begin{verbatim}
       Preconditions:
                  true
          
       Postconditions:
                  true
\end{verbatim}\normalsize
\begin{verbatim}
        float* GetImPixCoord() const;
\end{verbatim}

         Add an ImPixelCoord
\begin{verbatim}
        ImPixelCoord operator += (const ImPixelCoord &);
\end{verbatim}

         Subtract an ImPixelCoord
\begin{verbatim}
        ImPixelCoord operator -= (const ImPixelCoord &);


        ImPixelCoord();
        ImPixelCoord(float i, float j);
        ~ImPixelCoord();
\end{verbatim}

\subsection*{NON-MEMBER FUNCTIONS}

 Print an ImPixelCoord
\begin{verbatim}
   ostream& operator << (ostream& os, const ImPixelCoord ic);
\end{verbatim}

 Add two ImPixelCoords
\begin{verbatim}
   ImPixelCoord operator + (const ImPixelCoord&, const ImPixelCoord&);
\end{verbatim}

 Subtract two ImPixelCoords
\begin{verbatim}
   ImPixelCoord operator - (const ImPixelCoord&, const ImPixelCoord&);
\end{verbatim}

 Multiply an ImPixelCoord by a float
\begin{verbatim}
   ImPixelCoord operator * (const ImPixelCoord&, const float);
\end{verbatim}

 Multiply a float by an ImPixelCoord
\begin{verbatim}
   ImPixelCoord operator * (const float, const ImPixelCoord&);
\end{verbatim}

 Multiply an ImPixelCoord by a ImageCoord
\begin{verbatim}
   ImageCoord operator * (const ImPixelCoord&, const ImageCoord&);
\end{verbatim}

 Multiply a ImageCoord by an ImPixelCoord
\begin{verbatim}
   ImageCoord operator * (const ImageCoord&, const ImPixelCoord&);
\end{verbatim}
 Divide two ImageCoords (a coordinate value by a coordinate increment)
\begin{verbatim}
   ImPixelCoord operator / (const ImageCoord&, const ImageCoord&);
\end{verbatim}
