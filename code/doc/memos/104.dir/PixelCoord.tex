\clearpage
\section{PixelCoord.h}

\subsection*{HEADER FILE DESCRIPTION}
   The PixelCoord class is used for holding positions in the "storage" 
   coordinates. They may have fractional values, although in general use
   they are restricted to integers (indexes along the coordinate
   axes. If the dimensions of an image are m,n, the PixelCoords range 
   from 0 to m-1, and 0 to n-1.  The valid operations
   essentially consist of setting and getting values. Dimensionality
   is implicitly 2. PixelCoords must always be positive when being used
   within an image but can have negative values otherwise. Therefore there is
   trap for negative values.
  
   It would probably be very useful to define coordinate arithmetic and
   comparisons on this type. (<,>,-,= etc).
  
\subsection*{ENVIRONMENT}
\begin{verbatim}
#define PIXELCOORD_H

#include <iostream.h>
#include <assert.h>

\end{verbatim}

\subsection*{CLASS DESCRIPTION}

\subsection*{CLASS SUMMARY}
\begin{verbatim}
class PixelCoord
{
public:
    PixelCoord();
    PixelCoord(float ri, float rj);
    virtual ~PixelCoord();
    virtual int Ok() const;
    void SetPixelCoord(int i, float v);
    void SetPixelCoord(float *fp);
    float GetPixelCoord(int i) const;
    float* GetPixelCoord() const;
};
ostream& operator<< (ostream&, PixelCoord);
PixelCoord operator + (const PixelCoord&, const PixelCoord&);
PixelCoord operator * (const PixelCoord&, const float);
PixelCoord operator * (const float, const PixelCoord&);
PixelCoord operator - (const PixelCoord&, const PixelCoord&);
int operator == (const PixelCoord&, const PixelCoord&);
\end{verbatim}

\subsection*{MEMBER FUNCTIONS}
     The default constructor sets the positions to 0.0
\begin{verbatim}
    PixelCoord();
\end{verbatim}

     Construct at a given position
\begin{verbatim}
    PixelCoord(float ri, float rj);
\end{verbatim}

     Destructor currently does nothing
\begin{verbatim}
    virtual ~PixelCoord() {}
\end{verbatim}

     Checks the state of this object
\begin{verbatim}
    virtual int Ok() const;             // This object always OK
\end{verbatim}

     Set the location on the i'th axis. Range checks i.
\begin{verbatim}
    void SetPixelCoord(int i, float v);
\end{verbatim}

     Sets the positions on all axes. Should be a vector type so
     we could check the length of the vector.
\begin{verbatim}
    void SetPixelCoord(float *fp);
\end{verbatim}    

     Gets the value on the i'th axis
\begin{verbatim}
    float GetPixelCoord(int i) const;
\end{verbatim}

     Returns the value on all axes (should be vector)
\begin{verbatim}
    float* GetPixelCoord() const;
\end{verbatim}

\subsection*{NON-MEMBER FUNCTIONS}

 Print a PixelCoord
\begin{verbatim}
   ostream& operator<< (ostream&, PixelCoord);
\end{verbatim}

 Add two PixelCoords
\begin{verbatim}
   PixelCoord operator + (const PixelCoord&, const PixelCoord&);
\end{verbatim}

 Multiply a PixelCoord by a float
\begin{verbatim}
   PixelCoord operator * (const PixelCoord&, const float);
\end{verbatim}

 Multiply a float by a PixelCoord 
\begin{verbatim}
   PixelCoord operator * (const float, const PixelCoord&);
\end{verbatim}

 Subtract two PixelCoords
\begin{verbatim}
   PixelCoord operator - (const PixelCoord&, const PixelCoord&);
\end{verbatim}

 Compare two PixelCoords for equality
\begin{verbatim}
   int operator == (const PixelCoord&, const PixelCoord&);
\end{verbatim}
