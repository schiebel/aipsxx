\clearpage
\section{ImPixStep.h}

\subsection*{HEADER FILE DESCRIPTION}

\subsection*{ENVIRONMENT}
\begin{verbatim}
#define IMPIXSTEP_H

#include <iostream.h>
#include <K_Array.h>
#include <assert.h>
#include "PixelCoord.h"
#include "ImPixelCoord.h"

DECLARE_ONCE K_Array<int>;    // For the TI preprocessor
\end{verbatim}

\subsection*{CLASS DESCRIPTION}
   Number of ImPixel's per pixel. Important for derivation of sub-images from
   an image. For a system where ImPixels increase left to right, and
   bottom to top; and a system such as CIC where image pixels increase
   left to right and top to bottom, ImPixStep is + in the x coordinate and
   - in the y coordinate. ImPixStep is also a useful object for integer
   vector operations involving Pixel or ImPixel coordinates.
  
   INVARIANT
  
   Pixel steps must be non zero, although both positive and negative are
   acceptable.
   
\subsection*{CLASS SUMMARY}
\begin{verbatim}
class ImPixStep
{
public:
    ImPixStep();
    ImPixStep(int dx,int dy);
    ImPixStep(const K_Array<int> &);
    virtual ~ImPixStep();
    virtual int Ok() const;
    void SetImPixStep(int i, int m);
    void SetImPixStep(const K_Array<int> &);
    int GetImPixStep(int i) const;
    K_Array<int> GetImPixSet() const;
};
ImPixelCoord operator * (const PixelCoord &, const ImPixStep &);
ImPixelCoord operator * (const ImPixStep &, const PixelCoord &);
PixelCoord operator / (const ImPixelCoord&, const ImPixStep&);
ImPixStep operator * (const ImPixStep&, const ImPixStep&);
int operator == (const ImPixStep&, const ImPixStep& );
ostream& operator<< (ostream&, ImPixStep);
\end{verbatim}

\subsection*{MEMBER FUNCTIONS}
       Default constructor sets to steps of 1
\begin{verbatim}
    ImPixStep();
\end{verbatim}

       Construct with given steps (non-zero)
\begin{verbatim}
    ImPixStep(int dx,int dy);
\end{verbatim}

       Construct with given steps given in an array (non-zero)
\begin{verbatim}
    ImPixStep(const K_Array<int> &);
\end{verbatim}
    
       Destructor does nothing presently
\begin{verbatim}
    virtual ~ImPixStep();
\end{verbatim}

       Returns one if the state is acceptable (non-zero steps)
\begin{verbatim}
    virtual int Ok() const;
\end{verbatim}

       Set's the ith step to m (must be non-zero)
\begin{verbatim}
    void SetImPixStep(int i, int m);
\end{verbatim}

       Sets the steps to the vector elements (must be non zero and the
    correct size.
\begin{verbatim}
    void SetImPixStep(const K_Array<int> &);
\end{verbatim}

       Get the ith step
\begin{verbatim}
    int GetImPixStep(int i) const;
\end{verbatim}

       Get all steps into a vector;
\begin{verbatim}
    K_Array<int> GetImPixSet() const;
\end{verbatim}

\subsection*{NON-MEMBER FUNCTIONS}

   Arithmetic operators
\begin{verbatim}
ImPixelCoord operator * (const PixelCoord &, const ImPixStep &);
ImPixelCoord operator * (const ImPixStep &, const PixelCoord &);
PixelCoord operator / (const ImPixelCoord&, const ImPixStep&);
ImPixStep operator * (const ImPixStep&, const ImPixStep&);
\end{verbatim}

   Compare two ImPixSteps for equality
\begin{verbatim}
int operator == (const ImPixStep&, const ImPixStep& );
\end{verbatim}

   Print a PixelCoord
\begin{verbatim}
ostream& operator<< (ostream&, ImPixStep);
\end{verbatim}
