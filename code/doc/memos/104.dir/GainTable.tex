\clearpage
\section{GainTable.h}

\subsection*{HEADER FILE DESCRIPTION}
 This file contains definitions for the GainTable class.
  
\subsection*{ENVIRONMENT}
\begin{verbatim}
#define GAINTABLE_H

#include <assert.h>
#include <math.h>
#include <complex.h>
\end{verbatim}

\subsection*{CLASS DESCRIPTION}
GainTable class

\subsection*{CLASS SUMMARY}
\begin{verbatim}
class GainTable
{
public:
    GainTable(int, int);
    ~GainTable();
    complex& Gain(int, int);
    void SetGain(int, int, complex);
    int Nants();
    int Nentries();
};
\end{verbatim}

\subsection*{MEMBER FUNCTIONS}
      
       Constructor for GainTable specifying:
         number of receptors;
         number of entries;
\begin{verbatim}
    GainTable(int, int);
\end{verbatim}

       Destructor.
\begin{verbatim}
    ~GainTable();
\end{verbatim}
      
       Return a reference to a gain field in the table specifying:
         entry/tuple number;
         receptor number;
\begin{verbatim}
    complex& Gain(int, int);
\end{verbatim}
      
       Set a Gain value specifying
         entry/tuple number;
         receptor number;
         value
\begin{verbatim}
    void SetGain(int, int, complex);
\end{verbatim}

       Return number of collumns (NANTS)
\begin{verbatim}
    int Nants();
\end{verbatim}

       Return number of rows (Entries)
\begin{verbatim}
    int Nentries();
\end{verbatim}
