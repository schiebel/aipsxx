% $Header: /home/cvs/casa/code/doc/reference/Glish.dir/editline.tex,v 19.0 2003/07/16 04:18:44 aips2adm Exp $

\chapter{Command Line Editing}
\label{cli-editing}
\index{command line editing|(}
\index{{\em editline}}

This chapter describes the optional command line editing feature
of {\em Glish}. Command line editing is implemented using a {\em modified}
version of the original {\em editline} library.
\footnote{Copyright 1992,1993 Simmule Turner and Rich Salz, All rights reserved.}
(See the {\em editline/README} and {\em editline/COPYING}
distribution files full information.) This library was written
by Simmule Turner and Rich Salz, and was modified to work within 
{\em Glish}'s asynchronous event constraints.

Command line editing allows you to re-execute previous commands, fix
problems with the current command, or search for a 
previous command matching
a particular substring. These operations are performed based on 
{\em emacs-like} control characters and can be used any time you
enter a command. Table~\ref{editline-table} summarizes
the editing commands. The notation {\tt C-\verb-<CHAR>-} means that the character 
{\tt \verb-<CHAR>-} is typed while the {\em CONTROL} key is held down. So
{\tt C-b} is known as {\em control-b}. The notation {\tt M-\verb-<CHAR>-} 
means that the {\em META} key is held down while {\tt \verb-<CHAR>-} is typed {\em or}
the {\em ESCAPE} key is pressed {\em before} pressing {\tt \verb-<CHAR>-}. So
{\tt M-b} is known as {\em meta-b} or {\em escape-b}. In addition, {\em BACKSPACE}
represents the ``backspace key'', {\em DELETE} represents  the ``delete key'',
{\em RETURN} analogous to the ``return key'', and {\em UP-ARROW}, {\em DOWN-ARROW},
{\em LEFT-ARROW}, and {\em RIGHT-ARROW} analogous to the up, down, left, and right
keys common on most keyboards.

These commands become relatively natural with use, and are a great aid
when working with the CLI. If you are not interested in all of the
control characters, backspace, delete, and the arrow keys perform as expected,
and should provide most of what is  needed.

\begin{table}[tbh]
\begin{center}
\begin{tabular}{|c|p{3.0in}|}
\hline
Command & Action   \\
\hline
\hline
{\tt C-p} {\em or} {\em UP-ARROW} 	& get the previous command \\ \hline
{\tt C-n} {\em or} {\em DOWN-ARROW}	& get the next command \\ \hline
{\tt C-a}	& go to beginning of line \\ \hline
{\tt C-e}	& go to end of line \\ \hline
{\tt C-b} {\em or} {\em LEFT-ARROW}	& move back (to the left) one character \\ \hline
{\tt C-f} {\em or} {\em RIGHT-ARROW}	& move forward (to the right) one character \\ \hline
{\tt C-d}	& delete the current character. {\em Note} {\tt C-d} on an
			empty line exits Glish.\\ \hline
{\tt C-h} {\em or} {\em BACKSPACE} {\em or} {\em DELETE}	& delete the 
			previous (to the left) character \\ \hline
{\tt C-k}	& {\em kill} from the current character to
			the end of the line \\ \hline
{\tt C-y}	& {\em yank} back any previously {\em kill}ed character
			before the current character \\ \hline
{\tt C-l}	& redisplay the current line \\ \hline
{\tt C-]{\em \verb-<CHAR>-}}	& move to the single
			character {\tt{\em \verb-<CHAR>-}} \\ \hline
{\tt C-t}	& transpose current and previous character \\ \hline
{\tt C-r}	& search for a string. Enter the string followed by 
			{\em RETURN} when prompted with {\tt Search:}.
			Just pressing return when prompted uses the last
			search string.\\ \hline
{\tt M-b}	& move back (to the left) one word \\ \hline
{\tt M-f}	& move forward (to the right) one word \\ \hline
{\tt M-d}	& delete from the current position to the end 
			of the word \\ \hline
{\tt M-l}	& toggle to lower case from the current position
			to the end of the current word \\ \hline
{\tt M-u}	& toggle to upper case from the current position
			to the end of the current word \\ \hline
{\tt M-\verb-<-}	& get the first history line \\ \hline
{\tt M-\verb->-}	& get the last history line. This is 
			often used to restart a search.\\ \hline
{\tt M-C-h} {\em or} {\tt M-{\em BACKSPACE}} {\em or} {\tt M-{\em DELETE}}, 	& 
			delete from the current position to the beginning
			of the current word \\ \hline
{\tt M-w}	& copy from the current position to the beginning 
			of the line to the {\em kill} buffer for a later {\em yank} \\
\hline
\end{tabular}
\end{center}
\caption{ Command Line Editing Commands }
\label{editline-table}
\end{table}

\index{command line editing|)}
