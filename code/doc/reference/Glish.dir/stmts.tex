% $Header: /home/cvs/casa/code/doc/reference/Glish.dir/stmts.tex,v 19.0 2003/07/16 04:18:50 aips2adm Exp $

\chapter{Statements}
\label{statements}

\index{scripts!made up of statements}
\index{statements|(}
{\em Glish} scripts are made up of a series of {\em statements\/}, which are
first compiled and then executed sequentially.  Enclosing a series of
\index{statements!block}
\label{statement-block}
\indoptwokey{\{\}}{grouping statements}{?9grouping}
statements inside of braces (``{\tt \{ $\ldots$ \}}") groups them together
into a block that is treated syntactically as a single statement. These
statement blocks can be used to introduce local scope. (See \xref{local-stmt}.)
As in many languages, groups of statements can be collected into functions
to provide subroutines. (See  \cxref{functions}, and \xref{subseq}.)
This section describes the various types of statements available
in {\em Glish}.

\index{statements!semi-colon termination}
\index{semi-colon!as statement terminator}
Strictly speaking, all {\em Glish} statements are terminated with semi-colons
(``{\tt ;}").  For the most part, though, the {\tt ;} needn't be explicitly
present, because {\em Glish} can figure out when inserting a {\tt ;} makes sense
and does so automatically.  (See \xref{semicolons}.)  In the
examples that follow, we omit the final {\tt ;} from statements since, in
general, they are not necessary.

\section{Expressions as Statements}
\label{expr-statement}

\index{statements!expressions as}
\index{expressions!as statements}
Any expression is also a legal statement.  The expression is evaluated
and the result discard; presumably the expression has some interesting
side-effects.  (See \cxref{expressions}, for a discussion of the
different types of expressions.)

\section{Empty Statement}
\label{empty-statement}

\index{statements!empty}
\index{empty statement}
\index{semi-colon!as empty statement}
A lone ``{\tt ;}" is treated as an empty, do-nothing statement.  For
example,
\begin{verbatim}
    if ( x )
        ;
    else
        print "not x"
\end{verbatim}
is equivalent to
\begin{verbatim}
    if ( ! x )
        print "not x"
\end{verbatim}
(See \xref{conditionals}.)

\section{Printing}

\index{printing|(}
\indtt{print}{statement}
The {\tt print} statement provides a simple way of displaying (to
{\em Glish}'s {\em stdout}) values.  Its syntax is:
\begin{quote}
    {\tt print {\em value$_1$}, {\em value$_2$}, $\ldots$}
\end{quote}
where any number of values may be listed (including none, which produces
a blank line).

\label{paged-output}
By default, large amounts of output are sent through a pager, e.g. {\em less}
or {\em more}. This prevents the information from being overwhelming.
The pager to be run is determined by \verb+system.output.pager.exec+.
The line limit where paging starts is determined by
\verb+system.output.pager.limit+. Setting the limit to {\tt 0} causes the
pager to always be used, and setting the limit to {\tt -1} causes it to never be
used.

At the moment printing of values is crude. Values are printed with a single
blank between them and a final newline added at the end.
\index{output!control|(}
\index{print!control|(}
There are only two ways of affecting the printing of {\em Glish}
values. 
These can be used either  as an attribute of a given value or 
specified for the whole system.

\subsection{Print Precision}
\index{{\tt system} global variable!{\tt print.precision}}
\index{attributes!system defined!{\tt print.precision}}
\label{system-print-precision}
The first way to affect the printing of {\em Glish} values 
is \verb+print.precision+. It is sometimes important to increase
the number of decimal precision used to print floating point
values. This method  sets
the number of significant digits that are used to 
display floating point numbers.
If this is set in the {\tt system} record, then the default output
behavior changes for the entire system:
\begin{verbatim}
    a := 142.8767901343
    print a
    system.print.precision := 10
    print a
\end{verbatim}
In this example, the precision is initially set to {\tt 6} but is changed to {\tt 10}
before the second {\tt print} statement, therefore the output looks
something like:
\begin{verbatim}
    142.877
    142.8767901
\end{verbatim}
Note that this sets the number of {\em significant digits} not the number
of {\em decimal places}. The print precision can also be set for individual
values by setting attributes of the value. In this case, it only affects how
this single value is printed. Continuing the example above:
\begin{verbatim}
    Pi := 3.141592653589793238462643
    print Pi
    Pi::print.precision := 15
    print Pi
    print a
\end{verbatim}
The output this time will look like:
\begin{verbatim}
    3.141592654
    3.14159265358979
    142.8767901
\end{verbatim}
This provides a very basic way of controlling the output precision of
floating point numbers. Setting {\tt precision} to a negative integer
resets the default printing behavior for both {\tt system} and attribute.

\subsection{Print Limit}
\index{{\tt system} global variable!{\tt print.limit}}
\index{attributes!system defined!{\tt print.limit}}
\label{system-print-limit}
The second way to affect the printing {\em Glish} values 
is \verb+print.limit+. This is used to avoid inadvertently
printing {\em very} large values to the screen. For example:
\begin{verbatim}
    a := 1:1e7
    # print a              # this would be a mistake!
    a::print.limit := 10
    print a
\end{verbatim}
By setting the print limit for {\tt a} you get this output:
\begin{verbatim}
    [1 2 3 4 5 6 7 8 9 10 ... ]
\end{verbatim}
instead of {\em many} pages of integers. As with
\verb+print.precision+, this can be set in {\tt system} record to
change the default print limit for the whole system, or it can be
specified as an attribute of any value to change the limit for that
value only. Setting {\tt limit} to be {\tt 0}
or a negative integer resets things to the default limit, i.e. no print
limit.
\index{output!control|)}
\index{print!control|)}

\index{possible future changes!better printing of values}
In the future {\tt print} must be extended to allow more sophisticated
formatting.
\index{printing|)}

\section{Conditionals}
\label{conditionals}

\index{conditionals|(}
\indtt{if}{statement}
{\em Glish} provides C-style {\tt if} and {\tt if $\ldots$ else}
conditionals:
\begin{quote}
    {\tt if ( {\em expression} ) {\em statement}}

    {\tt if ( {\em expression} ) {\em statement$_1$} else {\em statement$_2$} }
\end{quote}
An {\tt if} statement evaluates {\em expression}, converts the result
to a {\tt boolean} value, and if true executes {\em statement}.
The \index{if statement!... else@{\tt $\ldots$ else}}
\index{{\tt else}}
{\tt if
$\ldots$ else} statement is similar, executing {\em statement$_1$} 
if the value is true
and {\em statement$_2$} if false.  The {\em expression}
should evaluate to a
\index{conditionals!evaluation of condition}
scalar value; if it is a vector then its first element is tested, though
\index{possible misfeatures!first element of vector tested in conditionals}
in the future an error may be generated instead.

\index{else!dangling}
\index{dangling {\tt else}}
As in most languages, a ``{\em dangling-else}'' is associated with the nearest
previous {\tt if}, so
\begin{verbatim}
    if ( x )
        if ( y )
            print "x and y"
        else
            print "either not x or not y"
\end{verbatim}
is interpreted as:
\begin{verbatim}
    if ( x )
        {
        if ( y )
            print "x and y"
        else
            print "either not x or not y"
        }
\end{verbatim}
and not as:
\begin{verbatim}
    if ( x )
        {
        if ( y )
            print "x and y"
        }
    else
        print "either not x or not y"
\end{verbatim}
\index{conditionals|)}

\section{Loops}

\index{loops|(}
{\em Glish} supports two looping constructs, {\tt while} and {\tt for}.

\subsection{While Loops}

\indtt{while}{statement}
A {\tt while} loop looks like:
\begin{quote}
    {\tt while ( {\em expression} ) {\em statement}}
\end{quote}
where {\tt statement} can be a statement block (enclosed in braces) or a
single statement. As in C, when encountering a {\tt while} statement the
{\em expression} is evaluated in the same way as in an {\tt if} statement.
(See \xref{conditionals}).  If true, {\em statement} is executed and
{\em expression} is then evaluated again and if true the process repeats.

\subsection{For Loops}
\label{for-loops}

\indtt{for}{statement}
{\em Glish} supports a different style of {\tt for} loop than~C.
A {\em Glish} {\tt for} loop looks like:
\begin{quote}
    {\tt for ( {\em variable} in {\em expression} ) {\em statement}}
\end{quote}
When the {\tt for} is executed, {\em expression} is evaluated to produce
a vector value.  Then {\em variable} is assigned to each of the values in
the vector, beginning with the first and continuing to the last.
For each assignment, {\em statement} is executed.  Upon exit from the
loop {\em variable} keeps the last value assigned to it.

\index{example!{\tt for} loops|(}
Here, for example, is a {\tt for} loop that prints the numbers from
{\tt 1} to {\tt 10} one at a time:
\begin{verbatim}
    for ( n in 1:10 )
        print n
\end{verbatim}

Here's another example, this time looping over all the even elements of
the vector {\tt x}:
\begin{verbatim}
    for ( even in x[x % 2 == 0] )
        print even
\end{verbatim}

Here's a related example that loops over the {\em indices} of the
even elements of {\tt x}:
\begin{verbatim}
    for ( even in seq(x)[x % 2 == 0] )
        print "Element", even, "is even:", x[even]
\end{verbatim}

And one final example, looping over each of the fields in a record {\tt r}:
\begin{verbatim}
    for ( f in field_names(r) )
        print "The", f, "field of r =", r[f]
\end{verbatim}
\index{example!{\tt for} loops|)}

\index{for statement!philosophy}
The philosophy behind providing only this style of {\tt for} loop is
rooted in the fact that {\em Glish} is most efficient when performing operations
on vectors.  I believe that this {\tt for} loop (which was
taken from the {\em S} language) encourages the programmer to think
about problems in terms of vectors, while~C's {\tt for} loop
does not.

\subsection{Controlling Loop Execution}

\index{loops!controlling execution|(}
{\em Glish} provides two ways to control the execution of a loop,
the {\tt next}\indtt{next}{statement}
and {\tt break}\indtt{break}{statement}
statements, which are directly analogous to~C's
{\tt continue} and {\tt break} (indeed,
\index{{\tt continue}!as synonym for {\tt next}}
{\tt continue} is allowed as
a synonym for {\tt next}).  The syntax of these is simply:
\begin{quote}
    {\tt next}

    {\tt break}
\end{quote}

The {\tt next} statement ends the current iteration of 
the surrounding {\tt while}
or {\tt for} loop and begins the next iteration, or exits the loop
if there are no more iterations.  The {\tt break} statement 
immediately exits the
loop regardless of whether or not there are normally  more iterations.
\index{loops|)}
\index{loops!controlling execution|)}

\section{{\tt return} Statement}
\label{return-stmt}

\indtt{return}{statement}
Normally a function's execution
proceeds until the last statement of the function, as discussed in 
\cxref{functions}.   If that statement
is an expression then the value of the expression becomes the result
of the function call; otherwise the result is {\tt F}.  A function
can also prematurely terminate using the {\tt return} statement,
which has two forms:
\begin{quote}
    {\tt return}

    {\tt return {\em expression}}
\end{quote}
The first form results in a returned value of {\tt F}; the second
form returns the value of {\em expression}. (See \cxref{functions}, for examples.)

\section{{\tt fail} Statement}
\label{fail-stmt-1}

\index{{\tt fail}!statement}
The {\tt fail} statement is used to indicate an error and provides
an alternate way to return from a function.  Typically a function returns 
by either executing the last statement of the function or an
explicit {\tt return} statement. There are two forms of {\tt fail} you can
use to return from a function:

\begin{quote}
    {\tt fail {\em expression}}

    {\tt fail}
\end{quote}
Typically, the {\em expression} is a string describing the error
condition, and when no expression is provided, the last {\tt fail}
value generated is used, if available.

\index{{\tt fail}!automatic propagation}
An additional feature of {\tt fail} values, i.e. the values produced
and returned by {\tt fail} statements, are automatically propagated.
This is true at two different levels. When {\tt fail} values are used
in expressions (including function calls), the result is a {\tt fail}
value which is immediately generated without evaluation of the expression.
Additionally if the {\tt fail} value returned from a function call to
an executing function, the result of the function which received the
{\tt fail} value will in turn be a {\tt fail} value if the {\tt fail}
is not handled. A {\tt fail} value is considered ``handled'' if its
type has been checked, e.g. via {\tt type\_name()} or {\tt is\_fail()}.
(See \cxref{fail-stmt-2}, for a complete description of {\tt fail}.)

\section{{\tt exit} Statement}
\label{exit-stmt}

\indtt{exit}{statement}

Normally a {\em Glish} program ends when the last statement of the main
program has been executed and all tasks have terminated, as discussed
in \xref{program-execution}.   To prematurely
end the program, you use {\tt exit}. 
As you can see, it  has a syntax similar to that
of {\tt return}:
\begin{quote}
    {\tt exit}

    {\tt exit {\em expression}}
\end{quote}
The first statement exits the program with a status of {\tt 0}. The second
statement evaluates {\em expression} and converts it to an {\tt integer} scalar
(by ignoring all but the first element), which is then used as the
exit status.

\section{{\tt local} ``Statement''}
\label{local-stmt}
\index{variables!declaring local}
\index{variables!local scope|(}
\index{scope!modifier!local}
You declare variables as local using the {\tt local} statement:

\begin{quote}
    {\tt local {\em id$_1$}, {\em id$_2$}, $\ldots$}
\end{quote}
Here each {\em id} has one of the following two forms:
\begin{quote}
    {\em name}

    {\em name} {\tt :=} {\em expression}
\end{quote}
\index{variables!initialization}
The second form specifies an initial value to assign to the local
variable.  You can use any valid expression (See \cxref{expressions}).
The assignment is done each time the {\tt local} statement is
executed.

If {\tt local} is used outside of statement blocks and functions,
it creates a global variable. So in this example,
\begin{verbatim}
    if ( x )
        local a := 3
\end{verbatim}
{\tt local} is neither in a statement block nor  inside
a function, so the variable that is created, {\tt a}, is a {\em global}
variable.  Global variables are accessible everywhere in {\em Glish}. 
This usage is
equivalent to:
\begin{verbatim}
    if ( x )
        a := 3
\end{verbatim}
If the {\tt local} declaration does not include any initializations
then it is equivalent to an empty statement:
\begin{verbatim}
    if ( x )
        local a
\end{verbatim}
is the same as
\begin{verbatim}
    if ( x )
        ;
\end{verbatim}
If {\tt local} is used inside of statement blocks, then the variable that is
created is local to that block. So in this example,
\begin{verbatim}
    if ( x )
        {
        local a := 89
        }
\end{verbatim}
because {\tt local} is used inside of a statement block, the variable {\tt a} is
local to that block. Subsequent uses of the variable {\tt a} in this block will
refer to this {\em local} variable rather than one that might be defined in the
{\em global} scope. If no initialization is provided, {\tt local} still
introduces a variable that is local to the statement block. So in this example:
\begin{verbatim}
    a := b := c := 0
    if ( T )
        {
        local a := 90, c := 90
        local b
        b := -90
        {
            local b, a := a
            a := a - 10
            b := 10
            print a,b,c
        }
        print a,b,c
        }
    print a,b,c
\end{verbatim}
the variables {\tt a}, {\tt b}, and {\tt c} start out in the {\em global} scope
with the value {\tt 0}. {\em Local} variables are introduced as part
of the {\tt if}
statement block, and finally another statement block is introduced inside the
{\tt if} block. Variables local to this block are again declared. The variable
{\tt a} in this case is initialized to the value that it had in a wider scope,
the {\tt if} block in this case, and it is then modified. Because it is declared
to be local to this block, the variable {\tt a} in 
the {\tt if} block is not modified.
The variable {\tt c}, on the other hand, is {\em not} local to the inner block so
{\tt c} in the inner block is the same as the {\tt c} in the {\tt if} block.
The final output of this code snippet is:
\begin{verbatim}
    80 10 90
    90 -90 90
    0 0 0
\end{verbatim}
(See  \S~\ref{scoping} where this is also discussed.)
\index{variables!local scope|)}

\section{Sending and Receiving Events}
Sending and receiving events forms the heart of {\em Glish}, and both
are discussed in \cxref{events}.  Here we briefly cover
the syntax of the related statements.

\subsection{Sending Events}
\label{send-event-stmt}

\index{events!sending|(}
The event-sending statement looks like:
\begin{quote}
    {\tt {\em expression} -> {\em name} ( {\em arg$_1$}, {\em arg$_2$}, $\ldots$ )}

    {\tt {\em expr$_1$} -> [ {\em expr$_2$ } ] ( {\em arg$_1$}, {\em arg$_2$}, $\ldots$ )}
\end{quote}
The {\tt expression} (or {\tt expr$_1$}) must resolve to one {\tt agent}.
(See \xref{event-syntax} for more information.)  Each
\index{events!sending!naming values}
{\em arg} argument (there needn't be any, in which case an event with
the value {\tt F} is sent) has one of two forms:
\begin{quote}
    {\em expression}

    {\em name} {\tt =} {\em expression}
\end{quote}
analogous to the syntax of a function call
(See \cxref{functions}).  If only one argument is specified and the first 
form is used
then {\em Glish} evaluates {\em expression} and uses the
result as the event value.  If more than one argument is specified or
the second form used for a lone argument then {\em Glish} constructs a record
in a manner similar to that described in \xref{record-constants}, and
uses that as the event value.  (See \cxref{events}, for a full
discussion.)

If an event is sent in the context of an expression, the interpreter
waits for a result from the client.  When this result is received  
the evaluation of
the expression is completed. The syntax is the same, 
but the context is different:
\begin{verbatim}
    my_agent->reset()
    if ( my_agent->ready() ) print "agent is configured"
\end{verbatim}
In the first case, no result is required, but in the second case
because  a result is needed
the interpreter waits.  (See \xref{send-event-expr}.)
\index{events!sending|)}

\subsection{Receiving Events}

\index{events!receiving|(}
There are two types of statements for receiving events, {\tt whenever}
and {\tt await}.  Both are discussed in full in \xref{whenever}, and
\xref{await}; here is a brief overview of the related syntax.

\subsubsection{Whenever Statements}
\label{whenever-stmt}

\indtt{whenever}{statement}
A {\tt whenever} statement looks like:
\begin{quote}
    {\tt whenever {\em event$_1$}, {\em event$_2$}, $\ldots$ do {\em statement}}
\end{quote}
At least
one {\em event} must be specified.  When any of the given events 
are generated, execute {\em
statement} with {\tt \$agent}, {\tt \$name}, and {\tt \$value} equal
to the {\tt agent} that generated the event, the name of the event,
and the event's value. (See \xref{event-syntax}, for a description of {\em event} syntax.)  

\subsubsection{Await Statements}
\label{await-statement}

\indtt{await}{statement}
{\tt await} statements have three forms:
\begin{quote}
    {\tt await {\em event$_1$}, {\em event$_2$}, $\ldots$}

    {\tt await only {\em event$_1$}, {\em event$_2$}, $\ldots$}

    {\tt await only {\em event$_1$}, {\em event$_2$}, $\ldots$ except {\em event$_1$}, {\em event$_2$}, $\ldots$}
\end{quote}
The first form waits for any one of the specified {\em event\/}'s to be
received before proceeding with execution.
If other events arrive during the interim, they are processed normally.
The second form does not process interim events but instead drops
them with a warning.  The third form only processes those interim events
listed after the {\tt except} keyword.

After completion of any {\tt await}, the variables {\tt \$agent}, {\tt \$name},
and {\tt \$value} correspond to the event that caused the {\tt await} to complete.
(See \xref{await}, for a full description.)
\index{events!receiving|)}

\subsection{{\tt activate} and {\tt deactive} Statements}

\indtt{activate}{statement}
\indtt{deactivate}{statement}
The {\tt activate} and {\tt deactivate} statements provide a mechanism
for turning {\tt whenever} statements ``on'' and ``off''.

These  statements have the following forms:
\begin{quote}
    {\tt activate }

    {\tt deactivate }

    {\tt activate {\em expr}}

    {\tt deactivate {\em expr}}
\end{quote}

The builtin function {\tt whenever\_active()} is used to see if
a {\tt whenever} statement is active or not. (See \xref{whenever_active-func}, for
information about {\tt whenever\_active()}, and \xref{activate-stmt}, for
information about {\tt activate} and {\tt deactivate}.)

\subsection{{\tt link} and {\tt unlink} Statements}

\indtt{link}{statement}
\indtt{unlink}{statement}
The {\tt link} and {\tt unlink} statements provide a mechanism for
establishing and suspending point-to-point connections between {\em Glish}
clients.  These connections sacrifice flexibility (being able to inspect
and modify event values) for performance.

These  statements have the following form:
\begin{quote}
    {\tt link {\em event$_1$} to {\em event$_2$}}

    {\tt unlink {\em event$_1$} to {\em event$_2$}}
\end{quote}

(See \xref{point-to-point}, for a full description.)

\section{Leaving Out the Statement Terminator}
\label{semicolons}

\index{statements!semi-colon termination|(}
\index{semi-colon!omitting|(}
{\em Glish} has a simple rule for when the {\tt ;} terminating
a statement can be left out.  In general, if a line ends with a
token that suggests continuation (such as a comma or a binary operator)
the statement continues onto the next line.  A semi-colon 
is inserted if it ends with
something that could come at the end of a statement.
Those tokens that can end a statement are:
\begin{itemize}
\item the \verb+)+ character, unless it's part of the test in a {\tt if},
{\tt for}, or {\tt while} statement, or the argument list in a {\tt
function} definition;
\item the \verb+]+ character;
\item the {\tt break}, {\tt exit}, {\tt next} (and its alias {\tt continue}),
and {\tt return} keywords;
\item identifiers and constants;
\item and the special event variables {\tt \$agent}, {\tt \$name},
and {\tt \$value}.
\end{itemize}

{\em Glish} inserts {\tt ;}'s only at the end of a line or just before a ``{\tt
\{}".  You {\em cannot} use these
rules to jam two statements onto one line:
\begin{verbatim}
    print a b := 3
\end{verbatim}
is illegal, though both
\begin{verbatim}
    print a; b := 3
\end{verbatim}
and
\begin{verbatim}
    { print a } b := 3
\end{verbatim}
are perfectly okay.

You can prevent {\em Glish} from inserting a {\tt ;} 
by using an escape (\verb+\+)
as the last character of a line.  For example,
\begin{verbatim}
    print a \
        , b
\end{verbatim}
is okay, and equivalent to
\begin{verbatim}
    print a,
        b
\end{verbatim}
or
\begin{verbatim}
    print a, b
\end{verbatim}
A  final \verb+\+ doesn't work after a comment.  For example:
\begin{verbatim}
    print a   # oops, syntax error next line \
        , b
\end{verbatim}
is interpreted as two separate statements, the second statement produces
a syntax error.
\index{semi-colon!omitting|)}
\index{statements!semi-colon termination|)}

\index{statements|)}
