% $Header: /home/cvs/casa/code/doc/reference/Glish.dir/internals.tex,v 19.0 2003/07/16 04:18:49 aips2adm Exp $

\chapter{Internals}
\label{internals}
\begin{sloppy}

\index{Glish!internals|(}
The {\em Glish} interpreter is written in C++ and consists of about
33,000 lines.
It presently runs on SunOS, Solaris, AIX, Irix, OSF/1, Linux, HP/UX,
and Ultrix. This chapter discusses those internals of the {\em Glish} system
relevant to understanding the system's strengths and weaknesses, and for
assessing the difficulty of porting the system to another platform.

\section{Encoding Event Values}
\label{encoding-events}

Communication between {\em Glish} and clients is handled by a data transport layer
called {\em SOS}. {\em SOS} is a library that provides platform independent transport
of binary data between different machines, e.g. data can be moved from a SPARC machine
to an ALPHA machine transparently. Any necessary conversion is done at the receiving end
of the communication. This avoids unnecessary conversion on some architectures. {\em SOS}
is a reimplementation in C++ of a portion of another transport library called {\em SDS}
(contact Chris Saltmarsh at 
\htmladdnormallink{salty@farpoint.co.uk}{mailto:salty@farpoint.co.uk}
for more information about {\em SDS}).

The header portion of each event encodes information such as the architecture of the
machine that sent the event, a time stamp, length of the event, etc. The interface
to the event header is defined in the C++ header file \verb+<sos/header.h>+. The
header is followed by the actual event data.

The header information and binary values are written to and read from {\em SOS}
{\em sinks} and {\em sources}. These are C++ objects which provide an I/O abstraction
that allows you to read and write values from and to file descriptors, e.g. the
pipes or sockets which connect {\em Glish} to clients. The header file \verb+<sos/io.h>+
defines the interface for {\em SOS} sources and sinks.

For the most part, {\em Glish}'s dependence on the transport layer is isolated in two
functions \verb+send_event()+ and \verb+recv_event()+. Both are found
in \verb+Client.cc+. These routines are parameterized with {\em SOS} source and
sink objects. This is necessary to allow events being sent by the interpreter
to be non-blocking. When used by the interpreter, \verb+send_event()+ sends what
it can and the returns under the assumption that it will resume later. This
eliminates some deadlock situations.

\section{Creating and Controlling Remote Clients}
\label{glishd}

\index{clients!creating!internals|(}
\index{clients!controlling internals|(}
\index{{\em glishd} remote daemon|(}
Clients on a remote host are controlled by a daemon called {\em glishd}.
This process takes care of starting clients, checking on them as they
run, and terminating them. It is {\em Glish}'s intermediary on the remote
machine. An important point, though, is that while {\em glishd}
will create clients, all event communication between those clients and the
interpreter is still done directly, via a socket connection, and not using
{\em glishd} as an intermediary.

{\em glishd} is designed to be run either by {\em root} or by individual
users. {\em glishd} is only started by the interpreter (on behalf of a user)
if a {\em glishd} has not already been started by {\em root} on the remote
machine. The behavior of {\em glishd} is different depending upon whether it
is running as root or not.

\subsection{{\em glishd} Started by Root}
\label{glishd-as-root}
\index{{\em glishd} remote daemon!root mode}

When {\em glishd} is started by {\em root}, it handles all client
requests for {\em all} users on a given machine. It does this by listening
to a published port (TCP port 2833) to which interpreters connect.
\index{socket connections!daemon port (2833)}
\index{ports!daemon port (2833)}
\index{{\em glishd} remote daemon!accept port (2833)}
When an interpreter connects to this port, {\em glishd} forks a copy of
itself. This forked copy of {\em glishd} immediately sets its userid to the
user which connected. From that point on, the forked process handles all
requests from that user. In general when all of that user's interpreters 
exit, the forked process exits too. The only time when this isn't true is if
the forked {\em glishd} started shared clients which are persistent; in this
case, it sticks around to manage these remaining clients which will be used
again.

{\em glishd} must be started by root for the shared client capabilities
(see \S~\ref{shared-clients}) to work properly.

To identify users, {\em glishd} uses key files. Each host is assigned
a unique key when {\em glishd} is set up. A key file is created for each
user by {\em Glish} (if one doesn't already exist) the first time the user starts
a client. It is assumed that the key directory is either network accessible,
e.g. via NFS, or that the keys are distributed to all machines on which
clients will be started. Note that if keys must be distributed {\em manually},
keys need to be created for users as well as hosts because the keys
automatically created by {\em Glish} won't be seen on other hosts. Key files must
be owned by the user to whom the key belongs and only that user should have
read/write permission. Here's an example of what the keys directory
for {\em Glish}
might look like:
\begin{verbatim}
    bash$ ls -lR keys
    total 8
    drwxr-xr-x   2 jdoe          512 Sep  3 10:16 hosts/
    drwxrwxrwx   2 jdoe          512 Sep  3 10:18 users/

    keys/hosts:
    total 16
    lrwxrwxrwx   1 jdoe           17 Sep  3 10:21 milisant -> milisant.nrao.edu
    -rw-r--r--   1 jdoe          132 Sep  3 10:21 milisant.nrao.edu

    keys/users:
    total 8
    -rw-------   1 jdoe          132 Sep 14 11:48 jdoe
\end{verbatim}
This is the directory hierarcy {\em Glish} expects to find for its keys. The important
thing to note is that the ``hosts'' directory must be readable by all
{\em Glish}
users, and the ``users'' directory must be writable by all {\em Glish} users.

Key files are created by running {\em glishkey} which should be built along
with {\em Glish}. Here's an example:
\begin{verbatim}
    bash$ glishkey 
    5971b940c8eb94edc374aa0389c46a409fbcf1c91eb3b27e0bae444ee0259f57 \
    619c5f151f059ba7186c24c8031b05a8f5f94187439cff947a0a5b1499db3ea7
    bash$ glishkey 
    5966343f14f643ae3371e7b7fbfd183253c1834e7c386e15ee164aca8eb8b11a \
    d2977ec140fbe3cd17a68c6be9f4e0a70f61b34d2040b68e82c940d77e84fcc7
    bash$ glishkey > milisant.nrao.edu
\end{verbatim}
Each time {\em glishkey} is run it generates another key. A key file is
created by simply piping the output of {\em glishkey} to a file.

The full path to the directory containing the ``{\tt users}'' and ``{\tt hosts}''
directory is built into {\em Glish}, but it can be changed from within
{\em Glish}
by setting \verb+system.path.key+
\label{system-path-key}
\index{{\tt system} global variable!{\tt path.key}}
value to a specific path. The path which {\em glishd} uses can't currently
be changed, though.

If you need to kill a {\em root glishd}, you should use the {\tt HANGUP}
signal, i.e. \verb+kill -HUP+.

\subsection{{\em glishd} Started by the Interpreter }
\index{{\em glishd} remote daemon!user mode}

If the {\em Glish} interpreter fails to connect to the {\em glishd} port, it
assumes that there is no {\em root glishd} running and starts a
{\em glishd}. The interpreter starts {\em glishd} using
\index{{\em rsh} command}
the {\em rsh} command
\index{{\em remsh} command}
(called {\em remsh} on some systems).  Thus the user that invoked
the interpreter must have an account on the remote host, and must
have transparent access to that account enabled via the user's
\index{rhosts file@{\em .rhosts} file}
{\em .rhosts} file.  Furthermore, {\em glishd} runs with that user's
permissions.  This {\em glishd} which was started by an interpreter
does all of the things that the {\em root glishd} will do, but it
only handles request from that specific interpreter. So each
interpreter running remote clients on a given machine would have a
{\em glishd} running. When the interpreter exits, {\em glishd} exits.

\subsection{{\em glishd} Events }
\index{{\em glishd} remote daemon!events}

In addition to creating and controlling clients, {\em glishd} provides
\index{network outages!detecting}
a mechanism for detecting network outages.  Every five seconds the
{\em Glish} interpreter sends a ``probe'' event to {\em glishd\/}.  If it
receives no response within the next five seconds, the interpreter
deems network connectivity lost, generates a warning message to this
effect, and creates a ``{\tt connection\_lost}'' event for the
{\tt system} agent (\xref{system-var}).  If {\em glishd} subsequently
responds to another probe then the interpreter deems connectivity
regained, reports this fact, and generates a ``{\tt connection\_restored}''
event for {\tt system}.  If {\em glishd} exits for any reason (e.g. 
it crashes or is killed), then the interpreter generates a
``{\tt daemon\_terminated}'' event for {\tt system}.

{\em glishd} is itself a {\em Glish} client and responds to the following events:
\begin{list}{}{}

\item[{\tt setwd}] specifies
\index{{\tt setwd} event sent to {\em glishd}}
the working directory {\em glishd} should use when executing programs
on the interpreter's behalf.

\item[{\tt setbinpath}] specifies
\index{{\tt setbinpath} event sent to {\em glishd}}
the path {\em glishd} should search for starting clients.

\item[{\tt setldpath}] specifies
\index{{\tt setldpath} event sent to {\em glishd}}
the dynamic loader library path {\em glishd} should use when starting clients.

\item[{\tt client}] creates
\index{{\tt client} event sent to {\em glishd}}
a new client.  The event has a single
{\tt string} value, the first part of which gives an internal
identifier for later use in manipulating the client, the remainder a
full argument list (i.e., including executable name) for invoking the
client.  {\em glishd} searches for the executable using whatever
{\tt \$PATH} environment variable it inherited via being invoked
by {\em rsh\/}.  If it cannot invoke the client it presently just
generates an error message to {\em stderr} and continues.  It probably
should generate an event instead.

\item[{\tt client-up}] checks to see 
\index{{\tt client} event sent to {\em glishd}}
if a particular client or {\em Glish} script is already up and running, i.e.
shared.

\item[{\tt kill}] terminates
\index{{\tt kill} event sent to {\em glishd}}
a client by sending it a {\em SIGTERM}
signal.  The {\tt string} value of the {\tt kill} event identifies
the client to kill.

\item[{\tt ping}] pings
\index{{\tt ping} event sent to {\em glishd}}
a client by sending it a {\em SIGIO}
signal.  The {\tt string} value of the {\tt ping} event identifies
the client to kill.  {\tt ping} supports the {\tt ping=} argument
of the {\tt client} function.  (See \xref{client-func-long}.)

\item[{\tt shell}] executes
\index{{\tt shell} event sent to {\em glishd}}
a synchronous shell command and returns
the resulting output.  The value of the {\tt shell} event is a 
record containing at least a {\tt command} field giving the command
to execute, and possibly a {\tt input} field giving the input to
be used (see the {\tt input=} argument of the {\tt shell} function,
\xref{shell-func}).

\index{shell function!running on remote hosts}
If {\em glishd} is unable to run the shell command it generates a
{\tt fail} event with a value of {\tt F}.  If successful then it
first generates an {\tt okay} event with a value of {\tt T}, then
\index{{\tt shell\_out} event sent by {\em glishd}}
a {\tt shell\_out} event for each line of output generated by
the shell command (it would be better to buffer all of the output lines
together into a single event),
and finally a {\tt status} event containing the exit status of the
shell command.  All of these events are handled directly by the
{\em Glish} interpreter; they are not ``visible" in a {\em Glish} script.

\item[{\tt probe}] requests
\index{{\tt probe} event sent to {\em glishd}}
that {\em glishd} acknowledge that it is still receiving messages
from the interpreter (i.e., that network connectivity holds).  Ordinarily
{\em glishd} immediately responds with a {\tt probe-reply} event.

\item[{\tt *terminate-daemon*}] tells
\index{terminate-daemon event@{\tt *terminate-daemon*} event sent to {\em glishd}}
{\em glishd} to exit. This cannot be used to force a {\em root glishd} to exit.
If you need to kill a {\em root glishd}, you should use the {\tt HANGUP}
signal, i.e. \verb+kill -HUP+.

\end{list}

\index{{\em glishd} remote daemon|)}
\index{clients!controlling internals|)}
\index{clients!creating!internals|)}

\section{Transmitting Events}

\index{interprocess communication}
{\em Glish} uses three different forms of interprocess communication (IPC) for
transmitting events.  When the {\em Glish} interpreter creates a client it passes it 
special arguments telling it how to make its connection with the interpreter.

\index{socket connections}
The most general form of IPC used by {\em Glish} is a socket connection.
In this case, the client's arguments tell it to which host and port number
\index{socket connections!default port (2000)}
to connect\footnote{The interpreter picks the first free port available
on its host, starting with port 2000.}.  The client then opens a socket
to that host and port, sends a message identifying itself, and uses the
socket for its subsequence communication with the interpreter. This same
mechanism is used when {\em glishd} must be started by the interpreter.

\index{pipes!client connection}
As an optimization, however, if a client is running on the same host
as the interpreter then the interpreter will use pipes to communicate
with the client instead.  Experience has shown that using pipes
locally can result in a substantial improvement in performance (a
factor of~2 on SunOS).  In this case, prior to creating the client
the interpreter creates two pipes which the client inherits
when the interpreter {\em exec()\/}'s it.

\index{link statement!implementation}
The {\tt link} statement requires the creation of a separate connection
between two clients.  The interpreter sends the sending end of the link
a special
\index{internal event!{\tt *link-sink*}}
\index{link-sink event@{\tt *link-sink*} event}
{\tt *link-sink*} event.  The Client Library of the sender intercepts
\index{sockets!Unix domain}
\index{sockets!Internet domain}
this event, creates either a UNIX- or Internet-domain socket endpoint
(the former if the sender and receiving reside on the same host),
\index{internal event!{\tt *rendezvous*}}
\index{rendezvous event@{\tt *rendezvous*} event}
creates a {\tt *rendezvous*}
event describing how to connect to endpoint (i.e., which host and port),
and returns that event to the interpreter.
When the interpreter receives a {\tt *rendezvous*} event it sends a
corresponding
\index{internal event!{\tt *rendezvous-resp*}}
\index{rendezvous-resp event@{\tt *rendezvous-resp*} event}
{\tt *rendezvous-resp*} event
to the link receiver, and reflects back a
\index{internal event!{\tt *rendezvous-orig*}}
\index{rendezvous-orig event@{\tt *rendezvous-orig*} event}
{\tt *rendezvous-orig*} event
to the sender (this second event isn't strictly necessary, but used to
be necessary to avoid deadlock).  The sender and receiver then rendezvous using the
given socket, establishing the separate connection between them.

\index{unlink statement!implementation}
The {\tt unlink} statement suspends the separate connection between two
{\em Glish} clients.  It is implemented by sending a
\index{internal event!{\tt *unlink-sink*}}
\index{unlink-sink event@{\tt *unlink-sink*} event}
{\tt *unlink-sink*} event to the sender-side of the link.  The sender
then marks the link as inactive; it does not destroy the link, however,
since it might later be resurrected via another {\tt link} statement.

\index{shared clients!implementation}
\label{shared-client-implementation}
With shared clients, {\em glishd} must be started by {\em root}. This is
because {\em glishd} maintains a lot of information about the client that
is shared by all {\em Glish} users. If {\em glishd} were not running as root,
it could not control the ownership of processes it creates on behalf
of many users. A shared client starts up like any normal client, but as
part of its initialization process it connects to the published {\em glishd}
port and registers itself as shared. Before any client is started, the
{\em Glish} interpreter first attempts to check with {\em glishd} to see if the
client is already running.  {\em glishd} uses the information it collects
as clients register themselves as shared to answer these requests. If
the client is already running, the interpreter sends a \verb+"client"+ event
to {\em glishd} which then forwards the event the preexisting client.

\section{Suppressing Stand-Alone Client Behavior}
\label{suppressing-standalone}

\index{clients!stand-alone!suppressing}
If a {\em Glish} client is run without
being given the special arguments telling it how to connect to
the {\em Glish} interpreter, and if it is given the {\tt -glish} argument,
then it runs in a ``stand-alone" mode (as described 
in \xref{standalone}).   
In this mode any text appearing on {\em stdin} is interpreted as an incoming
event, and any events generated are written in text form on {\em stdout\/}.

This behavior can be annoying when the client uses {\em stdin} or {\em stdout}
for a different purpose, or generates large events that you don't want to
look at in text form, or is to be placed in the background (which can
result in the client being ``stopped" by the terminal driver because
its {\em stdin} disappears).  For this reason,
by default ``stand-alone'' is not enabled, and the
\index{noglish client argument@{\tt -noglish} client argument}
{\tt -noglish} argument confirms this default.  The client will
not see any inbound events or create any output when it generates
events.  To enable the stand-alone reading of events from {\em stdin}
and writing events to {\em stdout}, you must invoke the client with the
\index{glish client argument@{\tt -glish} client argument}
{\tt -glish} flag.

\section{The ``Shell" Client}

\index{asynchronous shell clients|(}
{\em Glish} creates and manages asynchronous shell clients (i.e., created
using the {\tt shell} function's {\tt async=T} option; see \xref{shell-func})
\indtt{shell\_client}{}
using a special client called {\tt shell\_client}.  {\tt shell\_client}
is invoked with an optional {\tt -ping} argument (to implement {\tt ping=T})
and then a list of arguments corresponding to the shell command.

Prior to executing the shell command, {\tt shell\_client} attempts
\index{pseudo-ttys}
to create a ``pseudo tty" master/slave pair.  If successful then
it uses the pseudo-tty for the shell command's {\em stdin} and {\em stdout\/};
this causes the command to believe it is communicating directly with
a user, so it will generate prompts, perhaps use terminal escape sequences
where appropriate, and, most importantly, line-buffer its output.

If {\tt shell\_client} fails to create a pseudo-tty then it uses a
pair of pipes to communicate with the command.  In this case, the commands'
output will be block-buffered, meaning that it may not appear at all
until the command has either generated a lot of output, or terminates.
This behavior makes the shell command much more difficult to use as
a {\em Glish} client, because its output appears unpredictably.

Each line of output generated by
the shell command results in a {\tt string}-valued 
{\tt stdout} event
(as discussed in \xref{async-shell}.
{\tt shell\_client} itself responds to the following events:
\begin{list}{}{}

\item[{\tt stdin}] instructs
\indtt{stdin}{event to shell client}
{\tt shell\_client} to make a {\tt string}
representation of the {\tt stdin} event's value appear on the command's
{\em stdin} input stream.

\item[{\tt EOF}] causes
\indtt{EOF}{event to shell client}
{\tt shell\_client} to close the command's {\em stdin\/}.

\item[{\tt terminate}] results
\indtt{terminate}{event to shell client}
in {\tt shell\_client} killing the
shell command by sending it a {\em SIGTERM} signal.

\end{list}
\index{asynchronous shell clients|)}

\section{Initializing the Interpreter}
\label{glish-init}

\index{interpreter!initialization}
A number of the predefined functions discussed in \cxref{predefineds},
are actually written in the {\em Glish} language rather than built into the
{\em Glish} interpreter.  These functions are compiled into the interpreter
from a file called
\index{{\em glish.init} initialization file}
{\em glish.init\/}, which resides in the {\em Glish} source directory.

You can further customize your {\em Glish} run-time environment via the
{\tt .glishrc} customization file(s) (\xref{glishrc}).

\section{Installing and Porting Glish}
\label{installation}

\index{Glish!installation}
Directions for installing {\em Glish} can be found in the file
\indtt{README}{}
{\em README} at the top level of the {\em Glish} distribution
(i.e., at the some level as the {\em glish/} source directory).

\index{Glish!porting requirements}
The basic requirements for installing and/or porting {\em Glish} are a C++
\index{socket connections}
compiler and access to sockets as provided by the {\em socket()\/},
{\em bind()\/}, {\em accept()\/}, and {\em connect()\/} system
calls.  Those system dependencies of which we are aware have been
isolated in the (C, not C++) source file {\em system.c\/}; its
companion header file, {\em system.h\/}, provides brief documentation
as to what each function is expected to do.

\end{sloppy}
\index{Glish!internals|)}
