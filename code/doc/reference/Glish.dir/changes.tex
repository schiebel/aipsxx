% $Header: /home/cvs/casa/code/doc/reference/Glish.dir/changes.tex,v 19.0 2003/07/16 04:18:44 aips2adm Exp $

\chapter{Changes Between Glish Releases}
\label{changes}
\index{changes!between Glish releases|(}

This chapter documents the changes between the various \emph{Glish} releases.

%%%%%%% Change the release number on the title page!

\section{Release 2.7 (September 1998)}

The following changes were made between release 2.6 and 2.7.

%\begin{sloppy}
\begin{itemize}

\item Regular expressions were added with code taken from Perl version 5.004\_04
(\xref{regular-exprs}).

\item ASCII file I/O was added (\xref{io}).

\item The Tk widgets were removed from the {\em Glish} interpreter, and moved
to a client, {\tt gtk} (\xref{glishtk}).

\item Proxy client capability was added to the {\em Glish} client library to
permit moving the Tk widgets to a client (\xref{proxy-class}).

\item Removed warning about disgarded results.

\item Removed warning about {\em not} passing functions and agents to clients.

\item Added new command line options, \verb+-vi+, \verb+-vf+, \verb+-info+,
\verb+-help+, and \verb+-version+ (\xref{interpreter}).

\item \verb+$agent+, \verb+$name+, and \verb+$value+ were made local to {\tt whenever}
statements. This means that they are {\em not} corrupted during the processing
of one {\tt whenever} statement by other {\tt await} or {\tt whenever} statements.

\item \verb+request+ statements now work more like {\tt await} statements.
They do not block all events, in their basic form, and they wait for a specific event
from the client instead of just accepting the first event from the client.

\item Several new functions were added:
\begin{itemize}
\item {\tt tr( )}, \xref{tr-func}
\item {\tt to\_upper( )}, \xref{to_upper-func}
\item {\tt to\_lower( )}, \xref{to_lower-func}
\item {\tt readline( )}, \xref{readline-func}
\item {\tt unique( )}, \xref{unique-func}
\item {\tt shape( )}, \xref{shape-func}
\item {\tt strlen( )}, \xref{strlen-func}
\item {\tt sizeof( )}, \xref{sizeof-func}
\item {\tt alloc\_info( )}, \xref{alloc_info-func}
\item {\tt bundle\_events( )}, \xref{bundle_events-func}
\item {\tt flush\_events( )}, \xref{flush_events-func}
\item {\tt is\_regex( )}, \xref{is_regex-func}
\item {\tt is\_file( )}, \xref{is_file-func}
\item {\tt stat( )}, \xref{stat-func}
\item {\tt is\_asciifile( )}, \xref{is_asciifile-func}
\item {\tt write( )}, \xref{write-func}
\item {\tt read( )}, \xref{read-func}
\item {\tt fprintf( )}, \xref{fprintf-func}
\item {\tt printf( )}, \xref{printf-func}
\item {\tt sprintf( )}, \xref{sprintf-func}
\end{itemize}

\item A few additions were made to the Tk interface:
\begin{itemize}
\item popup frames were added (\xref{tkframe});
\item an event to raize a frame was added (\xref{tkframe});
\item the {\tt start}, {\tt end}, and {\tt resolution} parameters
      to {\tt scale} were changed from integer to float (\xref{tkscale});
\item an {\tt anchor} argument (and event) was added to {\tt button},
      {\tt label}, and {\tt message};
\item a {\tt bitmap} argument (and event) was added to {\tt button}
      (\xref{tkbutton});
\end{itemize}

\end{itemize}

\section{Release 2.6 (November 1997)}

The following changes were made between release 2.5 and 2.6:

%\begin{sloppy}
\begin{itemize}

\item The Tk widgets were bound to \emph{Glish} in an integrated way (\xref{glishtk}),
and the necessary functions were added:
\begin{itemize}
\item \texttt{have\_gui( )}, \xref{have_gui-func}
\item \texttt{tk\_hold( )}, \xref{tk_hold-func}
\item \texttt{tk\_release( )}, \xref{tk_release-func}
\item \texttt{frame(...)}, \xref{tkframe}
\item \texttt{button(...)}, \xref{tkbutton}
\item \texttt{scale(...)}, \xref{tkscale}
\item \texttt{text(...)}, \xref{tktext}
\item \texttt{scrollbar(...)}, \xref{tkscrollbar}
\item \texttt{label(...)}, \xref{tklabel}
\item \texttt{entry(...)}, \xref{tkentry}
\item \texttt{message(...)}, \xref{tkmessage}
\item \texttt{listbox(...)}, \xref{tklistbox}
\item \texttt{canvas(...)}, \xref{tkcanvas}
\end{itemize}

\item The \emph{Glish} library \texttt{libglish.a} was split into \texttt{libglishp.a} and
\texttt{libglish.a} so clients no longer must link in all of the interpreter's
symbols (now in \texttt{libglishp.a}).

\item Changed function scope rules. Now variables are local to the function
by default (\xref{scoping}).

\item The \emph{Glish} interpreter now does non-blocking writes for sending
events. This allows the interpreter to send as much of an event as it
can and then do other things until it can continue sending the event.
This avoids some cases of deadlock, e.g. :
\begin{verbatim}
        echo_client->h(1:3000); echo_client->h(1:3000)
\end{verbatim} 
This required changes in the transport layer.

\item The transport layer (\emph{SDS}) was replaced. The new library is called
\emph{SOS} (\texttt{libsos.a}). It is a minimal C++ reimplementation of the portion
of \emph{SDS} which \emph{Glish} used. This new transport layer overcomes some of \emph{SDS}'
short comings:
\begin{itemize}
\item coercion of type boolean to type integer in boolean values
sent to clients
\item loss of ``embedded'' whitespace in strings sent to clients
\item lack of non-blocking write
\end{itemize}
\emph{SOS} does this without sacrificing speed. \emph{SOS} uses \texttt{writev( )}
to minimize the number of system calls in sending data.

\emph{This means that all clients compiled with earlier versions of the
\emph{Glish}
libraries must be recompiled and relinked with these newer libraries.}

\item \texttt{opaque} values are no longer supported.

\item Files written to disk with \texttt{write\_value()} from an earlier version
of \emph{Glish} will \emph{not} be readable via \texttt{read\_value()} with this version of
\emph{Glish}. This is due to the change in transport layers.

\item Changed string to boolean conversion; now a string is \texttt{T} if
it has a non-zero length (\xref{predefineds-conversion}).

\item Added \verb+<fail>+ values as a way to deal with error conditions
(\xref{fail-stmt-2}).

\item \texttt{const} now means not modifiable rather than ``constant reference''
(\xref{constant-values}).

\item All value in \emph{Glish} are now ``copy-on-write''.  This means that
\texttt{ref} function parameters typically  should not be needed; \texttt{val}
is now the default type for function parameters (\xref{copy-on-write}).

\item Added numeric constants \texttt{pi} and \texttt{e} (\xref{pi-var}).

\item Changed function parameter evaluation to permit things like:
\begin{verbatim}
        func foo( x, y=2*x ) { print y }
\end{verbatim}

\item Dropped the keywords ``send'' and ``request''. Currently these
keywords are simply ignored, but with release 2.6.1 they will result in
syntax errors (\xref{events}).

\item Many memory leaks were fixed. Currently the most notable outstanding
leaks are the result of \texttt{strdup( )} calls; the current \texttt{String} class
needs to be beefed up and improved so it can be used throughout \emph{Glish} instead
of \texttt{char*}.

\item Changed \texttt{include} from directive to a statement
(\xref{include-directive}).

\item Added \texttt{eval( )} to allow strings to be interpreted as \emph{Glish}
programs (\xref{eval-func})

\item Added several new predefined functions:
\begin{itemize}
\item general purpose:
\begin{itemize}
\item \tt{time( )}, \xref{time-func}
\item \tt{is\_nan(x)}, \xref{is_nan-func}
\item \tt{is\_modifiable(x)}, \xref{is_modifiable-func}
\item \tt{is\_const(x)}, \xref{is_const-func}
\item \tt{ceil(x)}, \xref{ceiling-func}
\item \tt{floor(x)}, \xref{floor-func}
\item \tt{is\_fail(x)}, \xref{is_fail-func}
\end{itemize}
\item array manipulation:
\begin{itemize}
\item \tt{rbind(...)}, \xref{rbind-func}
\item \tt{cbind(...)}, \xref{cbind-func}
\end{itemize}
\item symbol table manipulation:
\begin{itemize}
\item \tt{is\_defined(x)}, \xref{is_defined-func}
\item \tt{symbol\_value(x)}, \xref{symbol_value-func}
\item \tt{symbol\_names(...)}, \xref{symbol_names0-func}
\item \tt{symbol\_set(...)}, \xref{symbol_set1-func}
\item \tt{symbol\_delete(x)}, \xref{symbol_delete-func}
\end{itemize}
\item event related:
\begin{itemize}
\item \tt{whenever\_active(x)}, \xref{whenever_active-func}
\end{itemize}
\end{itemize}

\item Some functions were changed:
\begin{itemize}
\item \texttt{log} is now log base 10, and a new function \texttt{ln} is
the natural log (\xref{log-func}).
\item \texttt{split} can now be used to split a string into individual
characters by passing an empty separator string, e.g. \verb+split('this','')+
returns \verb+t h i s+ (\xref{split-func}).
\item \texttt{whenever\_stmts} added an extra field to the record returned. This
field indicates which of the statements are active.
\end{itemize}

\item Added \texttt{wider} keyword to access a scope which is wider than the 
current scope, but not \emph{global} (\xref{wider}). The main reason this was
added was to support \emph{function closures} (\xref{func-closure}).

\item With nested function definitions, the multiple scopes are preserved and
used when the function is invoked. This along with \texttt{wider} allows for
useful \emph{function closures} (\xref{func-closure}).

\item A couple of important changes were made to \texttt{whenever} statements:
\begin{itemize}
\item ``stacked'' \texttt{whenever} statements (as a result of intervening
\texttt{await}s) are now handled properly. The event is matched with the
\emph{oldest} queued \texttt{whenever} statement (\xref{await-and-whenever}).
\item Multiple function scope are now preserved and used if necessary. This
is an issue when a \texttt{whenever} statement is set up within nested function
definitions.
\end{itemize}

\item \verb+^C+, \verb+^\ +, and \verb+^Z+ now work properly, at least when
command line editing is enabled. \verb+^C+ prompts before exiting; \verb+^\ +
exits directly.

\item Floating point exceptions are now handled properly on most platforms.
This was especially a problem on the DEC Alpha machines.

\item Added ``-l \emph{file}'' command line parameter (\xref{interpreter}).

\item Rewrote the \texttt{glishd} client (see \S~\ref{glishd}). It now
can be either
started by \emph{root} and run as a daemon accepting connections on a published
port, or started by a user and only handle commands from that particular
user. When \texttt{glishd} is started by \emph{root}, users are authenticated
using keys. Each user has a key, and each host has a key. \texttt{glishd}
(running as \emph{root}) authenticates each user using a combination of the
host and user key. This authentication is done thanks to Vern's \emph{NPD}
library. Authentication is not necessary when \texttt{glishd} is started by a
user other than \emph{root}.

\item Improved remote client start up. Now the binary search path and the
dynamic loader search path are set properly before starting either the
\texttt{glishd} or a client.

\item The configuration and make system can now handle creating shared
\emph{Glish}
libraries for most architectures.

\item Now whenever \emph{Glish} dies, it attempts to write out all of the non-function,
non-agent values to a file called \texttt{glish.core}.

\item Now \verb+pragma include once+ can be used to prevent a script from being
included multiple times.

\item Clients can now be shared by multiple \emph{Glish} interpreters.  (See 
\S~\ref{shared-clients} and \S~\ref{shared-client-implementation}.) This allows
interpreters to communicate with each other. A \texttt{pragma} directive was added
to allow script clients to specify that they should be shared. The following are
the only valid pragmas for share clients are:
\begin{itemize}
\item \verb+pragma shared user+
\item \verb+pragma shared group+
\item \verb+pragma shared world+
\end{itemize}
Much of the initial work on shared clients was done by Todd Satogata
(\htmladdnormallink{satogata@bnl.gov}{mailto:satogata@bnl.gov}) and
Chris Saltmarsh
(\htmladdnormallink{salty@farpoint.co.uk}{mailto:salty@farpoint.co.uk}).

\item Several new fields were added to the \texttt{system} record:
\begin{itemize}
\item \verb+system.pid+ and \verb+system.ppid+, \xref{system-pid}
\item \verb+system.tk+, \xref{system-tk}
\item \verb+system.limits.max+ and \verb+system.limits.min+, \xref{system-limits-max}
\item \verb+system.path.include+, \xref{system-path-include}
\item \verb+system.path.bin.+\emph{hostname}, \xref{system-path-bin}
\item \verb+system.path.bin.default+, \xref{system-path-bin}
\item \verb+system.path.key+, \xref{system-path-key}
\item \verb+system.print.limit+, \xref{system-print-limit}
\item \verb+system.print.precision+, \xref{system-print-precision}
\item \verb+system.output.trace+, \xref{system-output-trace}
\item \verb+system.output.log+, \verb+system.output.ilog+, and \verb+system.output.olog+, \xref{command-logging}
\item \verb+system.output.pager.exec+, \verb+system.output.pager.limit+, \xref{paged-output}
\end{itemize}
setting some of these affect things like how clients or include files
are found, how variables are displayed, or if debug output is generated
or not.

\item Added paged output so that large values are sent through a pager, e.g.
\emph{less} or \emph{more}, for display.

\item Added handling for the escape sequences \verb+'\a'+, \verb+'\e'+ and \verb+'\v'+
in strings.

\end{itemize}
%\end{sloppy}


\section{Release 2.5 (December, 1994)}

Release 2.5 comprised the following changes to Release 2.4:

%\begin{sloppy}
\begin{itemize}

\item Complex numbers, \texttt{complex} and \texttt{dcomplex}, were added to
\emph{Glish} along with the relevant functions:
\begin{itemize}
\item \texttt{complex(x,y)}, \xref{complex-func}.
\item \texttt{imag(x)}, \xref{imag-func}.
\item \texttt{real(x)}, \xref{real-func}.
\item \texttt{conj(x)}, \xref{conj-func}.
\item \texttt{arg(x)}, \xref{arg-func}.
\item \texttt{is\_complex(x)}, \xref{is_complex-func}.
\item \texttt{is\_dcomplex(x)}, \xref{is_dcomplex-func}.
\item \texttt{as\_complex(x)}, \xref{as_complex-func}.
\item \texttt{as\_dcomplex(x)}, \xref{as_dcomplex-func}.
\end{itemize}
See \S~\ref{complex-types}, \S~\ref{predefined-vectors}, and 
\S~\ref{numeric-constants} for details.

\item Attributes were added to \emph{Glish}. These allow extra information
to be attached to \emph{Glish} values. (See \xref{attributes} for more
information.)

\item Multi-dimensional arrays were added to \emph{Glish}. These arrays 
were implemented using attributes. Users can access single elements,
slices, or a set of unrelated array elements using the
subscript operator. (See \ref{arrays} for more information.)

\item Value subreferences were added for vector types. These
allow users to create references to only a portion of
another vector or array. Once created, subreferences can
be used much like whole-value references. 
(See \xref{partial-value-references} for more information.)

\item There were several changes to default arguments for functions.
In particular, now defaulted arguments can be omitted as follows:
\begin{verbatim}
    func one(a=1, b=2, c=3, d=4) [a, b, c, d]
    one(5,,4)
\end{verbatim}
This invocation invokes \texttt{one} with the default values for the 
second and fourth arguments.  (See \xref{purposefully-omitted-arguments}.)

The ellipsis can also have default values:
\begin{verbatim}
    func two(...=0) [...]
    two(5,,6)
\end{verbatim}
Here, the second argument to \texttt{two} gets the default value. (See
\xref{defaulted-ellipsis} for more information.) This example also
shows how the ellipsis are now used to construct an array, when all
of the arguments are \emph{numeric} types. (See 
\xref{ellipsis-vector-construction} for more information.)

Finally, there is a new \texttt{missing()} function which returns
a boolean vector indicating which parameters received default
values. (See \xref{missing-func-example}.)

\item Some predefined functions were added:
\begin{itemize}
\item \texttt{asin(x)}, \xref{asin-func}.
\item \texttt{acos(x)}, \xref{acos-func}.
\item \texttt{atan(x)}, \xref{atan-func}.
\item \texttt{prod(...)}, \xref{prod-func}.
\end{itemize}
\texttt{asin(x)}, \texttt{acos(x)}, and \texttt{atan(x)} do not yet work for
\texttt{complex} types.

\item Some old functions were improved, mostly to operate on an arbitrary
number of arguments.
\begin{itemize}
\item \texttt{sum(...)}, \xref{sum-func}.
\item \texttt{range(...)}, \xref{range-func}.
\item \texttt{max(...)}, \xref{max-func}.
\item \texttt{min(...)}, \xref{min-func}.
\item \texttt{rep(x,y)}, \xref{rep-func}.
\item \texttt{length(...)}, \xref{length-func}.
\end{itemize}

\item Command line editing was added using the \emph{editline} library
by Simmule Turner and Rich Salz (see the \emph{editline/README} distribution
file for copyright and other information). (See \xref{cli-editing} for more
information about command line editing in \emph{Glish}.)

\item All of the above changes were contributed by Darrell Schiebel.

\item \emph{Glish} now has \texttt{byte} and \texttt{short} types (\xref{first-type}),
corresponding to the C types of \texttt{unsigned char} and \texttt{short}.  These
types include the usual \texttt{as\_byte}, \texttt{is\_short}, etc. functions.
They were contributed by Chris Saltmarsh.

\item The \texttt{sync(c)} function can be called to synchronize a \emph{Glish}
script's execution with that of client \texttt{c} (\xref{sync-func}).

\item The functions \texttt{sort(x)} (\xref{sort-func}),
\texttt{sort\_pair(x,y)} (\xref{sort_pair-func}),
and \texttt{order(x)} (\xref{order-func})
provide support for sorting \emph{numeric} and \texttt{string} types.

\item The precedence of the \texttt{val}, \texttt{const}, and \texttt{ref}
operators has been increased (\xref{precedence-table}).

\item The \emph{glish.init} file is now compiled into the interpreter
rather than located at run-time (\xref{glish-init}).

\item A new \texttt{Client} member function, \texttt{Error}, is used to
report an error (particularly in a recently received event)
(\xref{client-class}).

\item A new \texttt{Client} member function, \texttt{ReplyPending}, returns
true if the client has a \emph{request/reply} pending and false otherwise
(\xref{client-class}).

\item The \texttt{Client} member function \texttt{AddInputMask} now returns
the number of new \texttt{fd}'s it added to the mask.

\item The \texttt{Client} member function \texttt{HasSequencerConnection} is
now called \texttt{HasInterpreterConnection}.

\item A \texttt{Client::PostEvent} variant has been added which does
\emph{printf-style} formatting but with two string arguments instead
of just one (\xref{client-class}).

\item The \texttt{GlishEvent} class has been spruced up a bit, and now
supports a member function \texttt{int IsRequest() const}, which returns
true if the event was a \texttt{request} and false otherwise
(\xref{glishevent-class}).

\item The \texttt{Value} associated with an event in \texttt{Client::PostEvent}
can now be nil, if you know it won't be used (\xref{client-class}).

\item Clients executed ``stand-alone'' (not from within a \emph{Glish} script)
now default to \texttt{-noglish} behavior, in which a call to \texttt{NextEvent}
returns nil (``no more events'').  To enable (for debugging purposes) reading
events from \emph{stdin} and writing them to \emph{stdout}, you must explicitly
specify the \texttt{-glish} flag (\xref{suppressing-standalone}).

\item Use of the \texttt{bool} type in the \emph{Client} and \emph{Value}
class libraries has now in many cases been replaced by \texttt{int}.  In
the remaining cases, \texttt{glish\_bool} must be used.  The abbreviations
of \texttt{true} and \texttt{false} for \texttt{glish\_true} and \texttt{glish\_false}
are no longer available (\xref{value-constructors}).

\end{itemize}
%\end{sloppy}


\section{Release 2.4}

Release 2.4 comprised the following changes to Release 2.3:
%\begin{sloppy}
\begin{itemize}

\item \emph{Glish} now has a mechanism for synchronous request/reply events:
\begin{verbatim}
    result := request a->b( 1:10 )
\end{verbatim}
sends a \texttt{b} event to \texttt{a} with value \texttt{1:10} and then waits for
\texttt{a} to reply.  The value of \texttt{a}'s reply is stored in \texttt{result}.
Note that \texttt{request} is a new keyword, which may cause incompatibilities
(syntax errors) with existing scripts that have variables with that name.
(See \xref{request-reply} for details.)

\item The ``event-send'' statement now takes an optional \texttt{send} keyword.
That is, you can write
\begin{verbatim}
    foo->bar( args )
\end{verbatim}
instead as
\begin{verbatim}
    send foo->bar( args )
\end{verbatim}
The belief is that using \texttt{send} will lead to more readable scripts,
and the plan is to gradually phase in \texttt{send} as a mandatory keyword.

\item The \texttt{Client} library now includes a member function:
\begin{verbatim}
    int Client::HasEventSource()
\end{verbatim}
which returns true if a \emph{Glish} client has \emph{any} input source (either
a connection to the \emph{Glish} interpreter, or by reading from \emph{stdin\/}),
and false if it has no input source (due to using \texttt{-noglish})
(\cxref{client-lib}).

\item The \texttt{[]} expression now returns a truly empty array
(\xref{empty-vector}). 

\end{itemize}
%\end{sloppy}

\section{Release 2.3}

Release 2.3 comprised the following changes to Release 2.2:
%\begin{sloppy}
\begin{itemize}

\item The new \texttt{activate} and \texttt{deactivate} statements allow
control of executing \texttt{whenever} bodies (\xref{activate-stmt}).

\item The \texttt{whenever\_stmts(agent)}, \texttt{active\_agents()},
\texttt{current\_whenever()}, and \texttt{last\_whenever\_executed()}
built-in functions provide information regarding which agents
generate what events, to be used in conjunction with \texttt{activate}
and \texttt{deactivate} (\xref{last_whenever_executed-func}).

\item Each host now runs at most one copy of the \emph{Glish} daemon \emph{glishd\/}.
The \emph{Glish} interpreter periodically probes the daemon and generates events
if connectivity is lost or regained, or if the daemon terminates
(\xref{glishd}).

\item A new agent record, ``\texttt{system}'', manages information about the
general environment where  a script runs.  It also generates events
indicating that the environment has changed.  (See \xref{system-var}.)

\item The ``\texttt{version}'' global has been removed, as it's now
subsumed by ``\texttt{system.version}''.

\item An ``\texttt{include}'' directive supports including the contents
of one \emph{Glish} source file inside another (\xref{include-directive}).

\item When running \emph{Glish} interactively, you can now create clients
and set up ``\texttt{whenever}'' statements to respond to their events.

\item You can use ``\verb+==+'' and ``\verb+!=+'' operators to compare
\texttt{record}, \texttt{function}, \texttt{agent}, and \texttt{opaque} values
(\xref{rel-exprs}).

\item The \emph{Glish} interpreter now allows only one filename on the command
line (since the ``\texttt{include}'' directive can be used to access multiple
sources).  Because of this change, you no longer need the special "--"
argument to delimit the end of source filenames and the beginning of script
arguments.  See \xref{interpreter}.

\item A new program (not a \emph{Glish} client), \texttt{tell\_glishd}, is available
for controlling the \emph{Glish} daemon on a given host (\emph{tell\_glishd was removed}).

\end{itemize}
%\end{sloppy}

\section{Release 2.2}

Release 2.2 comprised the following changes to Release 2.1 (the
original \emph{Glish} release):
%\begin{sloppy}
\begin{itemize}

\item Assignment (the ``\texttt{:=}'' operator) changed from being
a statement to being an expression, allowing ``cascaded'' assignments
(\xref{assignment}).

\item \emph{Glish} now supports ``compound'' assignment such as \texttt{x +:= 1}
(\xref{assignment}).

\item You can include an optional initialization assignment in
\texttt{local} statements (\xref{scoping}).

\item You can use a \emph{Glish} script as a client in another \emph{Glish} script
(\xref{client-scripts}).

\item The \texttt{opaque} type is available for client data uninterpreted
by \emph{Glish}.

\item The division operator (``\texttt{/}'') now always converts its
operands to \texttt{double} and yields a \texttt{double} value.

\item The \emph{Client} class now includes a virtual member function
\texttt{FD\_Change} you can use to be notified when the Client's
input sources changes (\xref{multiplexing}).

\end{itemize}
%\end{sloppy}
\index{changes!between Glish releases|)}
