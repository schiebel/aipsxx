% $Header: /home/cvs/casa/code/doc/reference/Glish.dir/funcs.tex,v 19.0 2003/07/16 04:18:46 aips2adm Exp $

\chapter{Functions}
\label{functions}

\index{functions|(}
{\em Glish} provides a flexible mechanism for defining and calling functions.
\index{functions!as data type}
These functions are a data type that  can be assigned to variables or
record fields, passed as arguments to other functions, and returned as results
of functions.

\section{Simple Examples}

\index{example!of using functions|(}
Before delving into the details of functions, first look at some
simple examples.  Here's a function that returns the difference
of its arguments:
\begin{verbatim}
    function diff(a, b) a-b
\end{verbatim}
It can also be written:
\begin{verbatim}
    function diff(a, b)
        {
        return a - b
        }
\end{verbatim}
Here's a version that prints its arguments before returning their difference:
\begin{verbatim}
    function diff(a, b)
        {
        print "a =", a
        print "b =", b
        return a - b
        }
\end{verbatim}
In the following version the second parameter is optional and if not
present is set to~{\tt 1}, so the function becomes a ``decrementer":
\begin{verbatim}
    function diff(a, b=1) a-b
\end{verbatim}

Suppose you defined {\em diff} using this last definition.  If
you call it using:
\begin{verbatim}
    diff(3, 7)
\end{verbatim}
it returns {\tt -4}.  If you call it using:
\begin{verbatim}
    diff(3)
\end{verbatim}
it returns {\tt 2}.  If you call it using:
\begin{verbatim}
    diff(b=4, a=7)
\end{verbatim}
it returns {\tt 3}, because  $7-4 = 3$.

Every function definition is an expression (See \cxref{expressions}).  When
the definition is executed, it returns a value whose type is {\tt function}.
You can then assign the value to a variable or record field.  For example,
\begin{verbatim}
    my_diff := function diff(a, b=1) a-b
\end{verbatim}
assigns a {\tt function} value representing the given function to
{\tt my\_diff}.  Later you can make the call:
\begin{verbatim}
    my_diff(b=4, a=7)
\end{verbatim}
and the result will be {\tt 3}, just as it will be if you call
{\tt diff} instead of {\tt my\_diff}.  With this sort of assignment you
can also leave out the function name:
\begin{verbatim}
    my_diff := function(a, b=1) a-b
\end{verbatim}
Now {\tt my\_diff} is the only name of this function.
\index{example!of using functions|)}

\section{Function Definitions}

\index{functions!definition syntax}
\index{functions!defined with {\tt function} keyword}
\index{{\tt function}}
A function definition looks like:
\begin{quote}
    {\tt function {\em name} ( {\em formal$_1$}, {\em formal$_2$}, $\ldots$ ) {\em body}}
\end{quote}
\index{{\tt func}!as abbreviation for {\tt function}}
The keyword {\tt function} can be abbreviated {\tt func}.
Each part of this definition is discussed below.

\index{expressions!functions}
\index{functions!as expressions}
Function definitions are {\em expressions\/}; they can occur anywhere
an expression can.  In particular because expressions are also
{\em statements\/}, a function definition can also occur anywhere
a statement occurs.

\section{Function Names}

\index{functions!names!optional}
\index{functions!names!global}
In a function definition, {\em name} is the name associated with the function.
As indicated in the examples above, {\em name} is optional.  If it is present
while compiling the function definition {\em Glish} creates a variable
with that name whose value is the resulting {\tt function} value.  This name
can then be used to call the function.

If the name is missing then presumably the function definition is being
used in an expression. The resulting {\tt function} value is then assigned
to a variable or passed as an argument to another function.  To illustrate
the latter, here is a function that takes two parameters, a vector and
another function.  It prints out the result of applying the function
to each element in the vector:
\begin{verbatim}
    func apply(array, f)
        {
        for ( a in array )
            print "f(", a, ") =", f(a)
        }
\end{verbatim}
You then call this function as follows:
\begin{verbatim}
    square := func(x) x^2
    apply( 1:10, square )
\end{verbatim}
to print out the squares of the first ten positive integers.
You can also  call it using:
\begin{verbatim}
    apply( 1:10, func(x) x^2 )
\end{verbatim}

\section{Function Parameters}

\index{functions!parameters|(}
\index{functions!parameters!formal}
Each function definition includes zero or more formal parameters,
enclosed within {\tt ()}'s.  Each {\em formal} looks like:
\begin{quote}
    {\em type} {\em name} {\tt =} {\em expression}
\end{quote}
{\em type} and {\tt =} {\em expression} are optional.  ({\em formal\/}'s have
one other form, ``{\tt ...}", discussed in \xref{ellipsis}.)

\subsection{Parameter Names}
\label{param-names}

\index{functions!parameters!names}
{\em name} serves
as the name of a local variable that during a function call is initialized
with the corresponding actual argument.  (See \xref{scoping}, for a discussion
of local variables.)
As in most programming languages,
\index{functions!parameter matching!left-to-right}
actual arguments match with formal parameters left-to-right:
\begin{quote}
    {\tt function diff(a, b) a-b}

    $\ldots$

    {\tt diff(3, 7)}
\end{quote}
matches {\tt 3} with {\tt a} and {\tt 7} with {\tt b}.  Argument
\index{functions!parameter matching!by name}
matching can also be done ``{\em by name}'':
\begin{verbatim}
    diff(b=1, a=2)
\end{verbatim}
matches {\tt 1} with {\tt b} and {\tt 2} with {\tt a}.

\subsection{Parameter Defaults}

\index{functions!parameters!defaults}
If the function definition {\em formal} includes an 
\indopone{=}{default parameter value}
{\tt =} {\em expression},
an actual argument for that formal parameter can
be left out when calling the function.  The actual argument is instead
initialized using {\em expression}
({\em expression} is referred to as the formal's {\em default}).
As you saw above, you can define {\tt diff} as:
\begin{verbatim}
    function diff(a, b=1) a-b
\end{verbatim}
in which case a call with only one argument will match that argument
with {\tt a} and initialize {\tt b} to {\tt 1}.  A call using by-name
\index{functions!parameter matching!by name}
argument matching, though, can not specify {\tt b} and not {\tt a},
because {\tt a} has no {\em default}:
\begin{verbatim}
    diff(b = 3)
\end{verbatim}
is illegal.

You can instead  define {\tt diff} with:
\begin{verbatim}
    function diff(a=0, b) a-b
\end{verbatim}
so that only {\tt b} is required in a call. This makes {\tt diff}
a ``negation'' function.  A call like:
\begin{verbatim}
    diff(6)
\end{verbatim}
is now illegal, because {\tt 6} matches {\tt a} and not {\tt b}; but
the call
\begin{verbatim}
    diff(b = 6)
\end{verbatim}
is legal and returns {\tt -6}. Arguments 
\label{purposefully-omitted-arguments}
which have defaults can 
simply be left out, as long as their absence is denoted. The previous
invocation can also be written as:
\begin{verbatim}
    diff(,6)
\end{verbatim}
The first argument is purposefully left out, therefore  it is  assigned
the default value.   You can test for missing arguments using the
{\tt missing()} function. (See \xref{missing-func}.)

Note that while match-by-position
and match-by-name arguments can be intermixed, a parameter must
\index{functions!parameters!only specified once in a call}
be specified only once.  For example,
\begin{verbatim}
    diff(3, 4, a=2)
\end{verbatim}
is illegal because {\tt a} is matched twice, first to {\tt 3} and
then to {\tt 2}.  Furthermore, once a match-by-name argument is given
no more match-by-position arguments can be given, because  their
position is indeterminate:
\begin{verbatim}
    diff(a = 3, 2)
\end{verbatim}
is illegal, since it's unclear what parameter {\tt 2} is meant to match.

\subsection{Parameter Types}

\index{functions!parameters!types|(}
A formal parameter definition can also include a type.  Presently,
the type is one of {\tt ref}, {\tt const}, or {\tt val}.
The type indicates the relationship between the actual argument and
the formal parameter.

\index{functions!parameters!{\tt ref}}
If the formal parameter's type is {\tt ref}
then the formal is initialized as a reference to the actual
argument.  In this case, the actual argument (outside of the function)
can be modified with the {\tt ref} parameter. (See \xref{assignment} regarding {\tt val}
assignment and \xref{references}, for a full discussion of references.)

\index{functions!parameters!{\tt val}}
\index{functions!parameters!default type as {\tt val}}
The {\tt val} parameter type indicates that the parameter can be modified,
but changes won't be reflected in the actual parameter which is passed into
the the function. The default type for parameters is {\tt val}.

\index{functions!parameters!{\tt const}}
If the type is {\tt const}, then it's initialized as
a {\tt const} value. A {\tt const} parameter cannot be modified in the
course of executing the function. Attempts to modify {\tt const} parameters
result in errors (See \xref{constant-values} for a discussion of {\tt const}
values).

\subsubsection{Using {\tt ref} Parameters} 
\index{example!function with {\tt ref} parameter}
Here is an example of a function with a {\tt ref} parameter that
increments its argument:
\begin{verbatim}
    function bump(ref x)
        {
        val x +:= 1
        }
\end{verbatim}
After executing:
\begin{verbatim}
    y := 3
    bump(y)
\end{verbatim}
{\tt y}'s value is {\tt 4}.  Note though that the following call:
\begin{verbatim}
    bump(3)
\end{verbatim}
is perfectly legal and does {\em not} change the value of the
constant {\tt 3} to {\tt 4}!

Here's another example of using a {\tt ref} parameter:
\begin{verbatim}
    # sets any elements of x > a to 0.
    func remove_outliers(ref x, a)
        {
        x[x > a] := 0
        }
\end{verbatim}
\index{functions!parameters!types|)}

\subsubsection{Parameter and Return Value Efficiency} 
\label{passing-efficiency}
\index{functions!parameters!passing efficiency}
\index{functions!return!efficiency}
For the most part, when  writing a function you do not  have to worry
about parameter passing efficiency. This is because all values in {\em
Glish} are stored and then accessed using {\em copy-on-write}.
(See \xref{copy-on-write}.)  Copy-on-write means that if two 
values are assigned to be equal then they  share the 
same underlying storage until one is modified.

This same mechanism is used when passing parameters or returning values.
The default parameter type is {\tt val}, but the parameter is only really
copied if the parameter is modified in the function. So large parameters are
only copied if necessary. Using a {\tt const} parameter type provides a way
to ensure that a given parameter does not change and, as a result, is not
copied. With return values, the value being returned is never modified 
after the {\tt return} statement, and as a result, return values are
not copied in the process of returning from the function.
\subsubsection{Future Directions} 
\index{possible future changes!parameter typing}
In the future {\em Glish} will support more explicit typing of parameters.
For example, it will be possible to define a function like:
\begin{verbatim}
    function abs(val numeric x)
\end{verbatim}
In this case if {\tt abs} is called with a non-{\em numeric} value
{\em Glish} detects the type clash and generates an error.

\subsection{Extra Arguments}
\label{ellipsis}

You can write functions that take a variable number of parameters
by including the special parameter
\index{ellipsis|(}
``{\tt ...}" (called {\em ellipsis})
in the function definition.
For example, here's a function that returns the sum of all its arguments,
\index{example!function with variable arguments}
regardless how many there~are:
\begin{verbatim}
    func total(...)
        {
        local result := 0
        for ( i in 1:num_args(...) )
            result +:= nth_arg(i, ...)
        return result
        }
\end{verbatim}

Two functions are available for dealing with variable argument lists.\indfunc{num\_args}
The function {\tt num\_args} \label{num_args-func}
returns the number of arguments with which it is called.
\index{functions!{\tt nth\_arg()} function|(}
\indfunc{nth\_arg}
The {\tt nth\_arg} \label{nth_arg-func} function 
returns a copy of the argument specified by its first
argument, with the first argument numbered as {\tt 0}.  For example,
\begin{verbatim}
    num_args(6,2,7)
\end{verbatim}
returns {\tt 3} and
\begin{verbatim}
    nth_arg(3, "hi", 1.023, 42, "and more")
\end{verbatim}
returns {\tt 42}.
\index{functions!{\tt nth\_arg()} function|)}

\index{possible misfeatures!{\tt num\_args()} function}
\index{possible misfeatures!{\tt nth\_arg()} function}
There's a temptation to expect {\tt num\_args} and {\tt nth\_arg} to
return information about ``{\tt ...}" if they're not given an argument
list, but presently they do not.  Probably they will be changed to
do so in the future.

\index{ellipsis!allowed operations}
Note that the only operation allowed with ``{\tt ...}" is to pass it
as an argument to another function.  It cannot otherwise appear in
an expression.  When passing it to a function, it is expanded into
\index{ellipsis!expanded as list of {\tt const} references}
a list of {\tt const} references to the actual arguments matched
by the ellipsis.  For example,
\index{example!function with variable arguments}
\begin{verbatim}
    func many_min(x, ...)
        {
        if ( num_args(...) == 0 )
            return x
        else
            {
            ellipsis_min := many_min(...)

            if ( ellipsis_min < x )
                return ellipsis_min
            else
                return x
            }
        }
\end{verbatim}
returns the minimum of an arbitrary number of arguments.

\index{ellipsis!matching following parameters}
When an ellipsis is used in a function definition then any parameters
listed after it must be matched by name (or by default).  Furthermore,
the corresponding arguments must come after those to be matched by
the ellipsis.  For example, given:
\begin{verbatim}
    func dump_ellipsis(x, ..., y)
        {
        for ( i in num_args(...) )
            print i, nth_arg(i,...)
        }
\end{verbatim}
both of the following calls are illegal:
\begin{verbatim}
    dump_ellipsis(1, 2, 3)
    dump_ellipsis(1, y=2, 3)
\end{verbatim}
In the first {\tt y} is not matched, and in the second the actual
argument {\tt 3} is not matched (in particular, it is not matched
by the ellipsis).  The following, though, is legal:
\begin{verbatim}
    dump_ellipsis(1, 2, y=3)
\end{verbatim}
and results in the ellipsis matching the single argument {\tt 2}.

\label{defaulted-ellipsis}
\index{ellipsis!default values}
An ellipsis can also have a default value specified. This value is 
used as the value for any arguments which are purposefully left
out. In the following,
\begin{verbatim}
    func add(...=0)
        {
        local ret := 0;
        for ( i in num_args(...) )
            ret +:= nth_arg(i,...)
        }
\end{verbatim}
{\tt add} is defined so that any arguments purposefully left
out will be set to zero. So the following invocation,
\begin{verbatim}
    print add(1,2,,,5)
\end{verbatim}
prints {\tt 8}. The two arguments between the {\tt 2} and the {\tt 5}
default to {\tt 0}.

\label{ellipsis-vector-construction}
\index{ellipsis!vector construction}
An ellipsis can be used to construct a vector. This allows all of the
parameters to be captured as a vector:
\begin{verbatim}
    func args(...) { return [...] }
\end{verbatim}
this returns the parameters as a vector. So given the following
invocations,
\begin{verbatim}
    args(1,5,8)
    args(4,3:5,1)
\end{verbatim}
the result of the first is {\tt [1, 5, 8]}, and the result of the second
is {\tt [4, 3, 4, 5, 1]}. It is important to note, however, that if one
of the arguments is not an array value, e.g. a {\tt record}, an error
results.

\index{ellipsis|)}
\index{functions!parameters|)}

\section{Missing Parameters}
\index{functions!{\tt missing()} function|(}
\label{missing-func}

The {\tt missing} function \indfunc{missing}
\label{missing-func-example}
is used to check the status of the function arguments. It returns
a boolean vector that has one element for each argument. If an element
of the vector has the value {\tt T}, it indicates that the corresponding
argument is {\em missing} and, as a result,  
is filled in by a default value. 
The following two functions,
\begin{verbatim}
    func m1(...=0) missing()
    func m2(a=1,b=2,c=3,...) missing()
\end{verbatim}
simply return the result of the call to {\tt missing}. The result of the
following invocation,
\begin{verbatim}
    m1(2,3,,5,)
\end{verbatim}
is {\tt [F, F, T, F, T]}. This indicates that the third and fifth 
parameters are  missing. The results of the following two invocations,
\begin{verbatim}
    m2(2,,3,4,5)
    m2(2)
\end{verbatim}
are {\tt [F, T, F, F, F]} and {\tt [F, T, T]} respectively. The first
result indicates that only the second parameter is missing, while the
second indicates that the second and third parameters are  missing.
\index{functions!{\tt missing()} function|)}

\section{The Function Body}

\index{functions!body|(}
\index{functions!body!as expression}
\index{functions!body!as statement block}
The body of a Glish function has one of two forms:
\begin{quote}
    {\em expression}

    {\tt \{ {\em statement$_1$ statement$_2$ $\ldots$} \} }
\end{quote}
When a function using the first form is called, it evaluates {\em expression}
and returns the result as the value of the function call.  With the second
form, the statements are collected in a statement block (See 
\S~\ref{statement-block} and \S~\ref{local-stmt}), and then the
statements are executed sequentially. The value of the last statement executed
is returned.  Most statements
\index{statements!return values}
do not have a value associated with them.  If the last executed statement
is one of these, the function call returns {\tt F}.  If the last executed
statement is an expression (See \xref{expr-statement}) or a {\tt return}
statement (See \xref{return-stmt}) then the call returns
the value of the expression.

\index{recursion!in function calls}
Functions may call themselves either directly or indirectly; there is
no limit on the recursive invocation depth other than the available
memory.

\subsection{Scoping}
\label{scoping}

\index{scope|(}
{\em Glish} supports three levels of scoping: {\em function}, {\em global}, and
{\em local\/}.

\subsubsection{{\em function} Scope}
\index{variables!function scope|(}
By default, all variables which are assigned within a
function are local to that function; they have {\em function scope}. So for
example:
\begin{verbatim}
    function my_sin(y)
        { 
        x := sin(y)
        return x
        }
\end{verbatim}
In this example, the {\tt x} that is  assigned to in the function
is local to this function. It will not modify a {\em global} {\tt x} if one
is defined. Note however that another {\em Glish} 
value is used in this function,
{\tt sin}. This is a {\em global} function, and because  global variables are
accessible in functions, the {\em global} {\tt sin} variable can be used. If {\tt sin}
is modified:
\begin{verbatim}
    function my_func(x)
        {
        sin := 2^x
        return sin
        }
\end{verbatim}
then {\tt sin} will be a variable that  is local to this function, and the
{\em global} {\tt sin} will not be modified.
\index{variables!function scope|)}
    
\subsubsection{{\em global} Scope}
\index{variables!global scope|(}
\index{scope!modifier!global|(}
A {\em global} variable persists throughout the execution of the
{\em Glish} program, and sometimes functions must be able to 
modify these variables
as well as access their values. For example, the following:
\begin{verbatim}
    x := 1
    function bump_x() { global x +:= 1 }
    bump_x()
    print x
\end{verbatim}
will print the value {\tt 2}. Here the {\tt global} keyword is used to specify
that a function variable should correspond to a {\em global} variable.
If the {\tt global} keyword is left out, {\tt x} will be local
to the function. When assignment is made to these function variables which
have been tagged as {\tt global}, the global variable of the same name is
modified. The {\tt global} keyword can also be used without the assignment:
\begin{verbatim}
    function bump_x_2()
        {
        global x
        x +:= 1
        }
\end{verbatim}
Here \verb+bump_x_2+ and \verb+bump_x+ are functionally equivalent. 
In either case,
subsequent use of the variable within the function will actually operate on
the global variable of the same name. The syntax of the {\tt global} statement
mirrors that of {\tt local} (See \xref{local-stmt}), but {\tt global}'s
semantics are the inverse of {\tt local}'s.
\index{scope!modifier!global|)}
\index{variables!global scope|)}

\subsubsection{{\em local} Scope}
\index{variables!local scope|(}
\index{scope!modifier!local|(}
The {\tt local} scope variable specifies that a particular variable
is local to the current statement block.  This variable was introduced
in \S~\ref{local-stmt}.  In this example:
\begin{verbatim}
    function trix(x)
        {
        y := x^2
        local z := x^3
        {
            local y := x^4, w
            w := 100
            print x,y,z,w
        }
        print x,y,z
        }
\end{verbatim}
the {\tt local} in front of the assignment of {\tt z} is not needed because
by default assigned variables in a function default to {\em function}
scope. 
Here {\tt x}, {\tt y}, and {\tt z} are local
to the function block. Another {\tt y} is introduced in the inner statement
block along with {\tt w}; these are local to the inner block. The invocation
of {\tt trix(2)} results in this output:
\begin{verbatim}
2 16 8 100
2 4 8
\end{verbatim}

Variables of {\em local} and {\em function} scope usually cease to exist
once the statement block with which they are associated exits. (See
\S~\ref{persistent-vars} for some exceptions.) When the function is next
called, the variable is recreated but with no trace of its former value.
All function parameters are {\em local} to the function body.
\index{scope!modifier!local|)}
\index{variables!local scope|)}

\subsubsection{{\em wider} Scope}
\index{scope!modifier!wider|(}
\label{wider}
Dividing variables into either {\em global} variables that  are
accessible from anywhere or {\em local} variables that are only accessible
within a give function is often sufficient.  There are times when access is
needed to variables which are defined in a {\em wider} scope but that are
are still not {\em global}. Here is a simple example:
\begin{verbatim}
    function object(x)
        {
        value := x
        ret := [=]
        ret.get := func ( )  { return value }
        ret.set := func (nv) { wider value
                               value := nv }
        return ret
        }
\end{verbatim}
\label{func-closure}
This example creates as simple function closure; the {\tt object} function
defines two functions which share a common variable. Without the {\em wider}
statement, the {\tt set} function will not be able to modify the shared
variable, i.e. {\tt value}.

The {\em wider} statement allows modification access to a variable defined
in a wider scope. Without this statement, modification of a variable makes
it local to the function by default.
\index{scope!modifier!wider|)}
\index{scope|)}

\subsection{Persistent Local Variables}
\label{persistent-vars}

\index{variables!local!persistent|(}
There are two ways  local variables can survive beyond the
end of the function call that created them.  Here ``survive" does
not mean that subsequent calls to the function see the previous
value, but that the value continues to exist after the initial
function call returns.

The first way is by returning a {\tt reference} to the variable. This
used to be the preferred way of returning big values in 
{\em Glish}, but because
{\em copy-on-write} (See \xref{passing-efficiency}) was added to {\em Glish},
you do not need to worry about this issue.
References, however, can still be returned (See \xref{references}).

\begin{sloppy}
\index{functions!variables surviving due to {\tt whenever}}
The second way local variables survive is if the function body
or a statement block with {\em local} variables executes a {\tt whenever}
statement.  The {\tt whenever} statement specifies actions to be taken at
a future time, asynchronously to the execution of the statements in the
{\em Glish} program (See \xref{whenever-stmt}, and 
particularly \cxref{events}).
For example, the following:
\index{example!function variables which persist}
\begin{verbatim}
    # Waits for x->foo, prints y
    # when it comes
    func announce_upon_foo(x, y)
        {
        whenever x->foo do
            print y
        }
    announce_upon_foo(x, 14)
    work()
    more_work()
    etc()
\end{verbatim}
will print {\tt 14} whenever {\tt x} generates a {\tt foo} event.
The value of {\tt y} (which, being a parameter, is {\em local} to
the function body) is remembered even after the call to {\tt
announce\_upon\_foo} returns.  You can later add another call:
\begin{verbatim}
    announce_upon_foo(x, "hi there")
\end{verbatim}
and when {\tt x} generates {\tt foo} events both {\tt 14} and {\tt "hi there"}
will be printed (in an indeterminate order).
\end{sloppy}

When the function executes a {\tt whenever} {\em all} of its local
variables are preserved and can be accessed within the statements
of the {\tt whenever}'s body.  If those statements modify the variables
then the modifications persist:
\begin{verbatim}
    func announce_upon_foo(x, y)
        {
        whenever x->foo do
            {
            print y
            y +:= 1
            }
        }
    announce_upon_foo(x, 14)
    announce_upon_foo(x, 7)
\end{verbatim}
will print {\tt 14} and {\tt 7} upon {\tt x}'s first {\tt foo} event,
{\tt 15} and {\tt 8} upon the second, and so on.

\begin{sloppy}
Persistent local variables are particularly important for {\em subsequences\/}.
(See \xref{subseq}.)
\end{sloppy}
\index{variables!local!persistent|)}
\index{functions!body|)}
\index{functions|)}
