% $Header: /home/cvs/casa/code/doc/reference/Glish.dir/fail.tex,v 19.0 2003/07/16 04:19:08 aips2adm Exp $

\chapter{Debugging, Logging, and Error Handling}
\label{debugging-logging-fail}

{\em Glish} scripts can be very difficult to debug. This chapter describes the tools
that are available in {\em Glish} to debug scripts, to log
interactions 
with {\em Glish},
and to write scripts which handle errors properly.

\section{{\tt fail} Statement}
\label{fail-stmt-2}

\index{{\tt fail}!statement}
\index{types!{\tt fail}}
The {\tt fail} statement is used for explicitly handling 
errors.  In {\em Glish}, {\em fail} refers to both a value and a statement.
As a statement, {\tt fail}
is an alternate {\tt return} statement and  the only way to generate
values of type {\tt fail}.
As a value type, {\tt fail} is unique because if a value of type {\tt fail} is 
passed as a parameter
to a function, the function is not invoked, but rather the result of the function
is the {\tt fail} value which was passed as a parameter. The same is true
for the evaluation of expressions containing {\tt fail} values. This
property allows {\tt fail} values, once generated, to be propagated.

Here is a simple example of how {\tt fail} values are used:
\begin{verbatim}
    func divide( dividend, divisor )
        {
        if ( divisor == 0 ) fail "division by zero"
        return dividend / divisor
        }
\end{verbatim}
Here the function simply checks for division by zero. So now if this function
is called with a divisor of {\tt 0}:
\begin{verbatim}
    func try1(x,y) divide(x,y)
    func try2(x,y) try1(x,y)
    func output(x) { print x }
    output(try2(34,0))
\end{verbatim}
the output looks something like:
\begin{verbatim}
    <fail>: division by zero
            Stack:  divide()
                    try1()
                    try2()
                    output() 
\end{verbatim}
The result contains the message with which {\tt fail} was called, and the call
stack to where the error occurred. Notice that {\tt output()} does not have to
check to verify that it has not been passed a {\tt fail} value. When a {\tt fail}
value is passed to a function, the function is not invoked, but the {\em return value}
of the function is the {\tt fail} value it was passed, and the print out of this value
is the output shown.

\index{{\tt fail}!automatic propagation}
Additionally, {\tt fail} values which are not {\em handled} are automatically propagated.
{\em Handling} a {\tt fail} value is defined to be checking it's type with one of the
type identification functions (see \xref{predefineds-identification}). With automatic
propagation, if the generated {\tt fail} value is not handled before the current
function returns then the result of the current function is also a {\tt fail} value.
This is true even though the {\tt fail} value was generated {\bf earlier in the
function execution}, perhaps much earlier. Because {\em Glish} does not do control
flow analysis of functions prior to execution, it has no way of determining if
a {\tt fail} value will be handled {\em at the time it is generated}.

\index{{\tt fail}!disable automatic propagation}
\index{interpreter!noaf disable auto fail propagation@{\tt -noaf} disable auto fail propagation}
\index{noaf flag to Glish interpreter@{\tt -noaf} flag to Glish interpreter}
It is possible to suppress this automatic propagation of {\em non-handled} {\tt fail}
values when {\em Glish} is started with the \verb+-noaf+ command line flag. {\em This flag
may be removed in the future.}

\indfunc{is\_fail}
\label{is_fail-func2}
The function {\tt is\_fail()} is available to check to see if a value has type {\em fail}.
If so, this function returns {\tt T}, otherwise it returns {\tt F}. This function can be
used to immediately propagate {\tt fail} values; for example (continuing the above example):
\begin{verbatim}
    func try3(a,b)
        {
        x := try2(a,b)
        if ( is_fail(x) ) {
            print "problem!"
            fail
            }
        return a + b * 2
        }
    print try3(8,2)
    print try3(8,0)
\end{verbatim}
The output produced by this code segment looks like:
\begin{verbatim}
    12
    problem!
    <fail>: division by zero
        Stack:  divide()
                try1()
                try2()
                try3()
\end{verbatim}
In theory, this use of {\tt is\_fail()} should only be necessary when there
are intermediate values produced, here {\tt x}, on which the return value
does not depend. In practice, this may not be the case.

\section{Input/Output Logging}
\label{command-logging}

\index{{\tt system} global variable!{\tt output.log}|(}
\index{{\tt system} global variable!{\tt output.olog}|(}
\index{{\tt system} global variable!{\tt output.ilog}|(}
\index{log!command|(}
\index{log!output|(}
It is useful to have a log of input to {\em Glish} and the output {\em Glish}
generates. There are several ways which this logging information can
be processed. Probably the simplest way is to have logging information
go to a file. Setting \verb+system.output.log+ to a text string, which
represents a valid directory path to a file. If the path is valid, all
input {\em and} all output will be logged to that file. The following
commands:
\begin{verbatim}
    system.output.log := "one.log"
    a := array(1:16,4,4)
    print a
    system.output.log := F
    print a
\end{verbatim}
result in a file called \verb+one.log+ in the current directory with
the contents:
\begin{verbatim}
    a := array(1:16,4,4)
    print a
    #[[1:4,]
    #    1 5 9  13
    #    2 6 10 14
    #    3 7 11 15
    #    4 8 12 16]
    system.output.log := F
\end{verbatim}
The output generated by {\em Glish} is preceeded by a ``{\tt \#}''. If
you want to log only the commands, setting \verb+system.output.ilog+ will
log only the commands, i.e. leaving out the lines commented by ``{\tt \#}''
above. Similarly, setting \verb+system.output.olog+ logs only the output
and not the input.

For more complicated applications, each of these record fields, i.e. {\tt log},
{\tt ilog}, and {\tt olog}, can be set to a function which takes a single parameter.
The function will then be called repeatedly with the line to be logged; the function
then logs the line. Likewise, these fields can be assigned to an agent. In this case,
each line to be logged is sent as an event where the {\em event name} is {\tt append} and
the {\em event value} is the line to be logged.
\index{log!output|)}
\index{log!command|)}
\index{{\tt system} global variable!{\tt output.ilog}|)}
\index{{\tt system} global variable!{\tt output.olog}|)}
\index{{\tt system} global variable!{\tt output.log}|)}

\section{Trace}
\label{command-trace}

\index{{\tt system} global variable!{\tt output.trace}|(}
\index{log!execution|(}
\label{system-output-trace}
Despite careful design a {\em Glish} script sometimes refuses to execute
properly and may appear to be non-determinate. With most languages, this type of
debugging is handled with a symbolic debugger. {\em Glish} doesn't have a symbolic
debugger (yet), but setting \verb+system.output.trace+ to ``{\tt T}''
causes {\em Glish} to
print out a breif summary of each statement just before it executes it. This can be
useful for debugging, and in some cases, it is even better than a symbolic debugger.
It is particularly useful for understanding the sometimes very compilcated interplay
of asynchronious events. This input:
\begin{verbatim}
    func fib(n)
        {
        if ( n <= 1 ) return 1
        return fib(n-1) + fib(n-2)
        }
    system.output.trace := T   
    fib(3)
\end{verbatim}
generates output that looks like:
\begin{verbatim}
            |-> fib(3)
            |-> if (n <= 1)
            |-> return (fib((n - 1)) + fib((n - 2)))
            |-> fib((n - 1))
            |-> if (n <= 1)
            |-> return (fib((n - 1)) + fib((n - 2)))
            |-> fib((n - 1))
            |-> if (n <= 1)
            |-> return 1
            |-> fib((n - 2))
            |-> if (n <= 1)
            |-> return 1
            |-> fib((n - 2))
            |-> if (n <= 1)
            |-> return 1
    3 
\end{verbatim}
\index{log!execution|)}
\index{{\tt system} global variable!{\tt output.trace}|)}
