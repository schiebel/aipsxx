% $Header: /home/cvs/casa/code/doc/reference/Glish.dir/using.tex,v 19.0 2003/07/16 04:18:53 aips2adm Exp $

\chapter{Using Glish}
\label{using-glish}
\begin{sloppy}

\index{Glish!using|(}
This chapter covers the particulars of using the {\em Glish} system, including
the {\em Glish} interpreter and its initialization files, and how to
debug {\em Glish}
programs.

\section{The Glish Interpreter}
\label{interpreter}

\index{interpreter|(}
All {\em Glish} scripts are executed by the {\em Glish} interpreter.  This program
\index{Glish!invoking}
\index{arguments!to Glish scripts}
is invoked as:
\begin{quote}
\begin{list}{}{}
\item[{\tt glish}]
     \mbox{{\tt [ -help ]}} \mbox{{\tt [ -version ]}} \mbox{{\tt [ -info ]}} \mbox{{\tt [ -v[1-9] ]}} \mbox{{\tt [ -vi ]}} \mbox{{\tt [ -vf ]}} \mbox{{\tt [ -w ]}} \mbox{{\tt [} {\em bindings $\ldots$} {\tt ]}} \mbox{{\tt [ -l} {\em file $\ldots$} {\tt ]}} \mbox{{\tt [} {\em file} {\tt ]}} \mbox{{\tt [ -- ] [ {\em args $\ldots$} ] }}
\end{list}
\end{quote}

\index{help flag to Glish interpreter@{\tt -help} flag to Glish interpreter}
\index{interpreter!help flag@{\tt -help} flag}
\index{version flag to Glish interpreter@{\tt -version} flag to Glish interpreter}
\index{interpreter!version flag@{\tt -version} flag}
\index{info flag to Glish interpreter@{\tt -info} flag to Glish interpreter}
\index{interpreter!info flag@{\tt -info} flag}
{\tt -help}, {\tt -version}, and {\tt -info} all provide information about the
{\em Glish} interpreter, and they are all optional. Indeed, using any of these
flags causes the interpreter to print information and then exit. {\tt -help}
prints out all of the standard command line options. {\tt -version} prints
out the {\em Glish} version number. {\tt -info} prints out information
about the {\em Glish} installation including version number and
directory information.

\index{v flag to Glish interpreter@{\tt -v} flag to Glish interpreter}
\index{interpreter!v verbose flag@{\tt -v} verbose flag}
\verb+-v[1-9]+ is an optional {\em verbose} flag indicating that the interpreter
should report on its activity.   The number indicates the
level of verboseness. For example, if {\tt -v1} is specified then it reports the
name and value of each event it receives from a client.  If {\tt -v2} is specified
then in addition to reporting each event as it is received from a client, {\em Glish}
also reports each event as it is queued for ``notification'' (i.e., triggering of {\tt whenever}
statements) and as it removes events from the notification queue to perform the
notification (i.e., running a {\tt whenever} statement).

\index{vi flag to Glish interpreter@{\tt -vi} flag to Glish interpreter}
\index{interpreter!vi verbose flag@{\tt -vi} verbose flag}
\verb+-vi+ is an optional {\em verbose} flag indicating that the interpreter
should list each file that is included by the interpreter. This results in
a list of all of the files involved in running a particular script.

\index{vf flag to Glish interpreter@{\tt -vf} flag to Glish interpreter}
\index{interpreter!vf verbose flag@{\tt -vf} verbose flag}
\verb+-vf+ is an optional {\em verbose} flag indicating that the interpreter
should report each {\em fail} (See \xref{fail-stmt-2}) which is not handled.
A {\tt fail} value is considered handled if {\tt is\_fail()} or {\tt type\_name()}
is called with the {\tt fail} value as a parameter.

\index{w flag to Glish interpreter@{\tt -w} flag to Glish interpreter}
\index{interpreter!w verbose flag@{\tt -w} verbose flag}
\verb+-w+ is an optional {\em verbose} flag indicating that the interpreter
should report the generation of each error string..

\index{l flag to Glish interpreter@{\tt -l} flag to Glish interpreter}
\index{interpreter!l load flag@{\tt -l} load flag}
{\tt -l} is an optional {\em load} flag. It indicates that these files should be
loaded initially before doing anything else. Multiple \verb+-l+ flags can be used
to load several {\em Glish} script files before either going to interactive mode or
running another script.

\index{environment variables!command-line bindings}
\index{interpreter!environment variable bindings}
{\em bindings} is an optional list of environment variable bindings
of the form:
\begin{quote}
    {\em var} {\tt =} {\em value}
\end{quote}

{\em file}, with no {\tt-l} flag, is the optional name of the source file
to compile and execute. When supplied, it is the last parameter {\em Glish}
examines. By convention such files end in a ``{\tt .g}" suffix.  When a {\em file}
is supplied, it is executed. If there are no active clients, {\em Glish}
exits after executing the file. If clients are active, {\em Glish} continues
to run, processing events to and from the clients. When no {\em file} argument
is supplied, {\em Glish} runs {\em interactively} (\xref{interactive-use}).

{\em args} is an optional list of arguments to pass to the {\em Glish}
script; if present, {\em args} may optionally
be delimited from the preceding
\index{\\zz@{\tt --\ } argument to Glish interpreter}
\index{interpreter!{\tt --} argument separator}
{\em file} using the special argument ``{\tt --}".  ``{\tt --}'' may also
be used in lieu of an initial {\em file} to specify that the interpreter
should run {\em interactively} (\xref{interactive-use}).

The {\em Glish} interpreter adds the given {\em bindings} to the environment,
compiles the listed {\em files}, and then executes the result with
the given {\em args\/}.  For example,
\index{example!environment variable bindings}
\begin{verbatim}
    glish host=cruncher myscript.g 10 12.5
\end{verbatim}
compiles the script {\em myscript.g} and runs it with {\tt argv}
equal to {\tt "10 12.5"} (see \xref{argv-var}, for a discussion of
the {\tt argv} global); the record field {\tt environ.host} will equal
{\tt "cruncher"} (see \xref{environ-var}, for a discussion of the
{\tt environ} global).

Prior to compiling the specified files, the interpreter attempts to
\index{interpreter!user-customization}
\index{user-customization file}
\label{glishrc}
evaluate local customization files. The order that it evaluates these
is as follows:

\begin{description}

\item[{\bf \$HOME/.glishrc.pre}]
First {\em Glish} looks for a file called ``{\tt.glishrc.pre}'' in the
user's home directory. If found, this file is executed.

\item[{\bf \$GLISHROOT/.glishrc}]
Next {\em Glish} looks for a local system
wide {\tt .glishrc} file. First it tries to find this file by looking
in the directory specified by {\tt \$GLISHROOT}
\index{GLISHROOT@{\tt \$GLISHROOT} environment variable}.
If this environment variable is not defined or if no {\tt .glishrc} file
is found there, the default location for this file is checked. This location
is established at build time. To see the default location, run \verb+glish -info+.
If this file is found, it is executed.

\item[{\bf \$GLISHRC} {\bf or} {\bf ./.glishrc} {\bf or} {\bf \$HOME/.glishrc}]
Next {\em Glish} attempts to find a user customization file. Three locations
are checked, and when a valid file is found, {\em Glish} does not check the
other possible locations. {\em Glish} first checks the
\index{GLISHRC@{\tt \$GLISHRC} environment variable}
{\tt \$GLISHRC} environment variable. If it is set and contains the path
to a valid file, {\em Glish} uses the file it names as the user's
the customization file.  If the variable is not set or does not
point to a valid file, {\em Glish} looks
\index{glishrc@{\tt .glishrc} initialization file}
\index{interpreter!glishrc file@{\tt .glishrc} file}
for a ``{\tt .glishrc}" file in the current directory. If none is found,
{\em Glish} looks in the user's home directory.

\item[{\bf \$HOME/.glishrc.post}]
finally {\em glish} tries to load a file called ``{\tt.glishrc.post}'' from the
user's home directory.
\end{description}

Next {\em Glish} proceeds with the files specified on the command line.
If you don't specify any arguments or if you give the ``{\tt --}'' argument
instead of a source file name, then {\em Glish} is run {\em interactively\/},
discussed in the next section.  \xref{program-execution} then discusses
execution of a {\em Glish} script more generally.
\index{interpreter|)}

\subsection{Using Glish Interactively}
\label{interactive-use}

\index{interpreter!running interactively}
When run interactively, the {\em Glish} interpreter
\index{interpreter!prompt}
\indoponekey{-}{Glish interpreter prompt}{+2Glish interpreter}
prompts with a dash (``{\tt - }") for input.  At this
prompt you may type any legal {\em Glish} statement (or expression, since
expressions are statements).  This prompt changes
\indoponekey{+}{Glish interpreter continuation prompt}{+1Glish interpreter}
to a plus sign (``{\tt + }")
if you need to type some more input to complete the statement you've
begun.  {\em Glish} then executes the statement and prints the result, continuing
until you type an end-of-file (usually control-D).  For example,
\index{example!using Glish interactively}
\begin{quote}
    {\tt largo 130 \%} {\em glish} \\
    {\tt Glish version 2.1.} \\
    {\tt - } {\em 1:3 * 2:4} \\
    {\tt [2 6 12]} \\
    {\tt - } {\em (end-of-file)} \\
    {\tt largo 131 \%}
\end{quote}
shows using {\em Glish} interactively to evaluate the product of {\tt [1, 2, 3]}
times {\tt [2,~3,~4]} to get {\tt [2, 6, 12]}.

There are no restrictions on interactive use.  In particular, you may
create clients and execute {\tt whenever} statements, and you may execute
scripts stored in files by {\tt include}'ing them (\xref{include-directive}).

\subsection{How Glish Executes a Script}
\label{program-execution}

\index{interpreter!execution of scripts|(}
\index{scripts!how executed|(}
When {\em Glish} executes a script, it first attempts to read and parse the
entire file. If this succeeds, it then executes the parsed file starting
with the first statement and proceeding through all the statements. If
the initial parsing fails, no part of the file is executed.

When a function is encountered, the {\em Glish} interpreter compiles the function
into a parse tree for later use, and then proceeds to the next statement.
In this example:
\index{example!script execution}
\begin{verbatim}
    func increment(x)
        {
        return x + 1
        }

    n := 1
    n := increment(n)
    print n
\end{verbatim}
the first statement is the definition of the {\tt increment} function.
The interpreter processes the function, assigns {\tt 1} to
the global {\tt n}, calls {\tt increment} with this value,
assigns the result to {\tt n}, and then prints the result.

\index{whenever statement!order of execution}
When a {\tt whenever} is executed, {\em Glish} simply makes a note that
in the future if it sees the indicated events it should execute
the body of the {\tt whenever}.  But an important point is that
whenever {\em Glish} is executing a statement block (such as when it initially
\index{events!when read by interpreter}
executes a script), it does {\em not} process any incoming events
until {\em after} it is done executing the entire block.  For example, when
\index{example!order of execution|(}
\begin{verbatim}
    d := client("demo")
    d->init([5, 3])
    whenever d->init_done do
        print "done initializing demo"
    do_some_other_work()
\end{verbatim}
is executed, it creates the client {\em demo} and sends it an {\tt init}
event with a value of {\tt [5, 3]}.  It then sets up a {\tt whenever} for
dealing with~{\tt d}'s {\tt init\_done} event, and finally calls
the function {\tt do\_some\_other\_work}.  Only {\em after} this function
returns will the {\em Glish} interpreter begin reading any events~{\tt d}
may have generated (in particular, a {\tt init\_done} event).  Any
events generated while {\em Glish} is executing a block of statements are
{\em not} lost but merely queued for later processing.

This rule regarding when events are read is particularly important
\index{race conditions!avoiding}
in an example like the one above.  It means that you do {\em not}
have to worry about setting up a {\tt whenever} for dealing with~{\tt d}'s
{\tt init\_done} event prior to sending an {\tt init} event to~{\tt d}
even though perhaps {\tt d} will generate 
this immediately after
receiving the {\tt init} event, which may occur before the interpreter
executes the {\tt whenever} (because~{\tt d}'s client is a process
separate from the interpreter process).

One important effect of this rule, however, is that it may have
unintuitive consequences when dealing with subsequences.  In particular,
the following program:
\begin{verbatim}
    x := 1

    subseq print_x()
        {
        whenever self->print do
            print x
        }

    p := print_x()
    p->print()
    x := 2
    p->print()
\end{verbatim}
prints {\tt 2} followed by {\tt 2}, not {\tt 1} followed by {\tt 2}.
This is because {\tt x} is assigned to {\tt 2} {\em before} the
first {\em Glish} processes the first {\tt print} event sent to {\tt print\_x}.

Changing this sequence to:
\begin{verbatim}
    x := 1

    subseq print_x()
        {
        whenever self->print do
            print $value
        }

    p := print_x()
    p->print(x)
    x := 2
    p->print(x)
\end{verbatim}
produces the expected output of {\tt 1} followed by {\tt 2}.
\index{example!order of execution|)}

The rule of no event processing until {\em Glish} is done executing the
statement block holds also when it is executing the body of a
{\tt whenever} statement.  One exception to this rule is that
\index{await statement!suspending execution}
\index{suspending execution}
executing an {\tt await} statement (\xref{await}) suspends
execution of the block. Therefore {\em Glish} begins processing events again
until the {\tt await} condition is met, at which point {\em Glish} continues
executing the block.

\index{events!order of processing|(}
When the interpreter is processing events it first processes any
pending events (those that have already arrived, or were generated
by event-send's to subsequences during the last statement block's
execution).  If processing one of these events leads to the generation
of additional events (again, those sent to subsequences) then these
events, too, are processed, until all pending events have been
exhausted.

At this point, the interpreter checks to see whether there are any clients
\index{interpreter!exiting automatically}
running.  If not, it exits because now there is no possibility of 
further events being generated.  If, however, there are some clients
running, then the interpreter waits for one or more of them to generate an
event.  When this happens, the events are read and queued in an
undetermined order and the interpreter again processes these pending events
as described in the preceding paragraph.

Because the interpreter cannot tell which clients only generate events
in response to events they've received, it cannot detect a condition
in which it should exit because only these sorts of clients are running
(and therefore no new events can be created).  Usually scripts using
\indtt{exit}{statement}
clients with this property can be modified to use {\tt exit} statements
(\xref{exit-stmt}) when it is clear they are finished doing their work.

One final point regards the ordering of events, to which the following
rules apply:
\begin{itemize}

\index{events!order preserved}
\item Events generated by the same agent are processed by the interpreter
in the same order as generated.

\item Events sent to the same agent are received by it in the same order
as generated.

\item Events generated by different agents or sent to different agents
\index{temporal event ordering}
may lose their temporal ordering; i.e., the one sent first may arrive
(from a clock's point of view) last.

\index{whenever statement!order of execution}
\item If an event matches more than one {\tt whenever} statement, then
the order in which the {\tt whenever} statement bodies are executed
is unspecified.  It is possible that in the future this will change
and an order will be specified.

\end{itemize}
\index{events!order of processing|)}
\index{scripts!how executed|)}
\index{interpreter!execution of scripts|)}

\section{Debugging Glish Scripts and Clients}

\label{debugging-clients}
\index{debugging|(}
{\em Glish} provides some rudimentary tools to aid in debugging {\em Glish} scripts.
These include the tools discussed in Chapter~\ref{debugging-logging-fail},
reporting which events are generated with the \verb+-v+ flag (discussed in this chapter),
the ``event monitor'' (discussed below), the use of the {\tt print} statements\footnote{
Since {\em Glish} is interpreted, you will find that adding debugging {\tt print}
statements to a {\em Glish} script and restarting often gives a very quick
means of debugging.}, and use of the {\tt client} function's {\tt suspend=T}
option (discussed below).

\subsection{Debugging Clients}

Debugging {\em Glish} clients is primarily done using a conventional
\indtt{suspend=}{client argument}
debugger and the {\tt suspend=T} option to the {\tt client}
function (see \xref{client-func-long}).  With this option, when the
client is executed and constructs its {\em Client} object (see
\xref{client-class-overview}), the {\em Client} constructor will
first announce itself, producing a message like:
\index{example!debugging}
\begin{verbatim}
    tester @ myhost, pid 18915: suspending ...
\end{verbatim}
and then suspend itself by entering the following loop:
\begin{verbatim}
    suspend = 1;
    while ( suspend )
        sleep( 1 );
\end{verbatim}
\index{{\em gdb}}
\index{{\em dbx}}
A debugger such as {\em gdb} or {\em dbx} then can be used to attach
to this running process.\footnote{For {\em gdb}, use the {\tt attach}
\index{clients!suspending!attaching to}
command; for {\em dbx}, start it with no parameters and then enter
{\tt debug} followed by the name of the executable and the {\em pid} value.}
Once attached, set the variable
{\tt suspend} to {\tt 0} (or {\tt glish\_false})\footnote{To do this you
may need to change the debugging scope after attaching to the process;
in both {\em gdb} and {\em dbx} this is done using the {\tt up} command
to move up the call stack until arriving in the {\tt Client::Client}
constructor (which may have a more garbled name).}, set any breakpoints
needed for debugging, and continue the process.

In addition to the {\tt suspend=T} argument to {\tt client}, every time
{\em Glish} creates a new client the interpreter inspects the environment
\index{suspend list of clients to suspend@{\tt \$suspend} list of clients to suspend}
variable {\tt \$suspend} to see whether that client's name occurs in
{\tt \$suspend}'s (blank-separated) list of names.  For example,
\begin{verbatim}
    glish suspend="my_demo ./bin/camac" my_script.g
\end{verbatim}
executes the script {\em my\_script.g} and whenever a client
with the name {\tt my\_demo} or {\tt ./bin/camac} is executed,
the client will act as though {\tt suspend=T} had been specified.

Note that the name here refers to the actual name of the executable
and {\tt not} the name of the variable to which the result of the
{\tt client()} call is assigned.  For example, the above {\tt suspend}
list will not suspend a client created by the following:
\begin{verbatim}
    my_demo := client("./my_demo")
\end{verbatim}

\subsection{The Event Monitor}
\label{event-monitor}

\index{event monitor}
\index{glish\_monitor environment variable@{\tt \$glish\_monitor} environment variable}
If the {\tt \$glish\_monitor} environment variable is set when 
the {\em Glish} interpreter starts running, then 
the interpreter takes the value of
the monitor as designating the name of a client to 
serve as an ``event
monitor".

The event monitor is sent an event every time either the interpreter
receives an event from a client, or sends an event to a client or a
subsequence.  The former results in the monitor receiving an {\tt event\_in}
\indtt{event\_in}{monitor event}
\indtt{event\_out}{monitor event}
event, the latter in an {\tt event\_out} event (i.e., ``in" and ``out"
are relative to the interpreter's perspective).  The event's value
is a {\tt record} with three fields: {\tt id}, which identifies the
agent associated with the event; {\tt name}, the name of the event;
and {\tt value}, the value of the event.

\end{sloppy}
\index{debugging|)}
\index{Glish!using|)}
