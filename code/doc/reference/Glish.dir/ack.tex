% $Header: /home/cvs/casa/code/doc/reference/Glish.dir/ack.tex,v 19.0 2003/07/16 04:18:41 aips2adm Exp $

\chapter{Acknowledgments}
\label{ack}

\index{Glish!acknowledgments|(}
The original concept of a language for specifying event connections began
with Chris Saltmarsh and colleagues at the CERN Laboratory for Nuclear
Research.  {\em Glish} itself was originally developed by Vern Paxson, of the
Lawrence Berkeley Laboratory, in consultation with Chris.  Since shortly after
release 2.5 in 1994, Darrell Schiebel, of the National Radio Astronomy
Observatory (NRAO), has maintained and developed {\em Glish}.

{\em Glish} was originally developed as part of the Superconducting Super
Collider, but currently, its primary use is for data reduction and telescope
control in radio astronomy. In particular, it is a cornerstone of
the \htmllink{{\em AIPS++}}{\aipsurl} system, and it is in this context
that development of {\em Glish} has continued since release 2.5.  Indeed,
it has been the {\em AIPS++} developers and users who have uncovered
the majority of problems which have been fixed. Their efforts are much
appreciated.

Finally, Vern Paxson has been a great resource. His input
was instrumental during the initial period of {\em Glish} development
at the NRAO, and he has proven to be a great sounding board
for new designs. His efforts are greatly appreciated.
\index{Glish!acknowledgments|)}

\chapter{Copyright}
\label{copyright}
\index{Glish!copyright|see copyright}

{\em Glish} has benefited greatly from the wide availability of source code on the
{\em Internet}. The core functionality in {\em Glish}\footnote{Any
modifications to other original source code included with Glish are also covered by
{\em Glish}'s copyright.} is distributed under
a {\em UCB}\footnote{University of California, Berkeley} style copyright, and care
has been taken to avoid including other software distributed under incompatible
copyrights, e.g. the {\em GNU Public License}. This section includes the
copyright notices for {\em Glish} and all of the components that have been
knitted into {\em Glish}.

\section{{\em Glish}}
\index{copyright!Glish|(}
The{\em  Glish} software and documentation is covered by the following copyright:

\begin{center}
Copyright \copyright 1993 The Regents of the University of California.\\
All rights reserved.\\
Copyright copyright 1997,1998,1999 Associated Universities Inc.
All rights reserved.
\end{center}

\begin{sloppy}
\begin{quotation}
This code is derived from software contributed to Berkeley by
\mbox{Vern~Paxson} and software contributed to Associated Universities
Inc. by \mbox{Darrell~Schiebel}.

The United States Government has rights in this work pursuant
to contract no. DE-AC03-76SF00098 between the United States
Department of Energy and the University of California, contract
no. DE-AC02-89ER40486 between the United States Department of Energy
and the Universities Research Association, Inc. and Cooperative
Research Agreement \#AST-9223814 between the United States National
Science Foundation and Associated Universities, Inc.

Redistribution and use in source and binary forms are permitted
provided that: (1) source distributions retain this entire
copyright notice and comment, and (2) distributions including
binaries display the following acknowledgment:  ``This product
includes software developed by the University of California,
Berkeley, the National Radio Astronomy Observatory (NRAO), and
their contributors'' in the documentation or other materials
provided with the distribution and in all advertising materials
mentioning features or use of this software.  Neither the names of
the University nor NRAO nor the names of their contributors may be
used to endorse or promote products derived from this software
without specific prior written permission.

THIS SOFTWARE IS PROVIDED ``AS IS'' AND WITHOUT ANY EXPRESS OR
IMPLIED WARRANTIES, INCLUDING, WITHOUT LIMITATION, THE IMPLIED
WARRANTIES OF MERCHANTABILITY AND FITNESS FOR A PARTICULAR
PURPOSE.
\end{quotation}
\end{sloppy}

This basically says ``do whatever you please with this software except
remove this notice or take advantage of the University's, NRAO's, or the
Glish authors' names''.
\index{copyright!Glish|)}

\section{Regular Expressions}
\index{copyright!regular expressions|(}
\index{Spencer, Henry}
\index{Wall, Larry}

The regular expression software in {\em Glish} is based on code from
\htmllink{{\em Perl}}{\perlurl}
(version 5.004\_04). This code was {\bf\sc Modified} to be used as a
library. The original code is covered by \underline{The Artistic License}
and Henry Spencer's copyright.

\subsection{The Artistic License}
\begin{center}
Copyright \copyright 1991-1997, Larry Wall.
\end{center}

\subsubsection{Preamble}

The intent of this document is to state the conditions under which a
Package may be copied, such that the Copyright Holder maintains some
semblance of artistic control over the development of the package,
while giving the users of the package the right to use and distribute
the Package in a more-or-less customary fashion, plus the right to make
reasonable modifications.

\subsubsection{Definitions}

\begin{description}

	\item[Package] refers to the collection of files distributed by the
	Copyright Holder, and derivatives of that collection of files
	created through textual modification.

	\item[Standard Version] refers to such a Package if it has not been
	modified, or has been modified in accordance with the wishes
	of the Copyright Holder as specified below.

	\item[Copyright Holder] is whoever is named in the copyright or
	copyrights for the package.

	\item[You] is you, if you're thinking about copying or distributing
	this Package.

	\item[Reasonable copying fee] is whatever you can justify on the
	basis of media cost, duplication charges, time of people involved,
	and so on.  (You will not be required to justify it to the
	Copyright Holder, but only to the computing community at large
	as a market that must bear the fee.)

	\item[Freely Available] means that no fee is charged for the item
	itself, though there may be fees involved in handling the item.
	It also means that recipients of the item may redistribute it
	under the same conditions they received it.

\end{description}

\subsubsection{Conditions}

\begin{enumerate}

\item You may make and give away verbatim copies of the source form of the
Standard Version of this Package without restriction, provided that you
duplicate all of the original copyright notices and associated disclaimers.

\item You may apply bug fixes, portability fixes and other modifications
derived from the Public Domain or from the Copyright Holder.  A Package
modified in such a way shall still be considered the Standard Version.

\item You may otherwise modify your copy of this Package in any way, provided
that you insert a prominent notice in each changed file stating how and
when you changed that file, and provided that you do at least ONE of the
following:

\begin{enumerate}

    \item place your modifications in the Public Domain or otherwise make them
    Freely Available, such as by posting said modifications to Usenet or
    an equivalent medium, or placing the modifications on a major archive
    site such as uunet.uu.net, or by allowing the Copyright Holder to include
    your modifications in the Standard Version of the Package.

    \item use the modified Package only within your corporation or organization.

    \item rename any non-standard executables so the names do not conflict
    with standard executables, which must also be provided, and provide
    a separate manual page for each non-standard executable that clearly
    documents how it differs from the Standard Version.

    \item make other distribution arrangements with the Copyright Holder.

\end{enumerate}

\item You may distribute the programs of this Package in object code or
executable form, provided that you do at least ONE of the following:

\begin{enumerate}

    \item distribute a Standard Version of the executables and library files,
    together with instructions (in the manual page or equivalent) on where
    to get the Standard Version.

    \item accompany the distribution with the machine-readable source of
    the Package with your modifications.

    \item give non-standard executables non-standard names, and clearly
    document the differences in manual pages (or equivalent), together
    with instructions on where to get the Standard Version.

    \item make other distribution arrangements with the Copyright Holder.

\end{enumerate}

\item You may charge a reasonable copying fee for any distribution of this
Package.  You may charge any fee you choose for support of this
Package.  You may not charge a fee for this Package itself.  However,
you may distribute this Package in aggregate with other (possibly
commercial) programs as part of a larger (possibly commercial) software
distribution provided that you do not advertise this Package as a
product of your own.  You may embed this Package's interpreter within
an executable of yours (by linking); this shall be construed as a mere
form of aggregation, provided that the complete Standard Version of the
interpreter is so embedded.

\item The scripts and library files supplied as input to or produced as
output from the programs of this Package do not automatically fall
under the copyright of this Package, but belong to whoever generated
them, and may be sold commercially, and may be aggregated with this
Package.  If such scripts or library files are aggregated with this
Package via the so-called "undump" or "unexec" methods of producing a
binary executable image, then distribution of such an image shall
neither be construed as a distribution of this Package nor shall it
fall under the restrictions of Paragraphs 3 and 4, provided that you do
not represent such an executable image as a Standard Version of this
Package.

\item C subroutines (or comparably compiled subroutines in other
languages) supplied by you and linked into this Package in order to
emulate subroutines and variables of the language defined by this
Package shall not be considered part of this Package, but are the
equivalent of input as in Paragraph 6, provided these subroutines do
not change the language in any way that would cause it to fail the
regression tests for the language.

\item Aggregation of this Package with a commercial distribution is always
permitted provided that the use of this Package is embedded; that is,
when no overt attempt is made to make this Package's interfaces visible
to the end user of the commercial distribution.  Such use shall not be
construed as a distribution of this Package.

\item The name of the Copyright Holder may not be used to endorse or promote
products derived from this software without specific prior written permission.

\item THIS PACKAGE IS PROVIDED "AS IS" AND WITHOUT ANY EXPRESS OR
IMPLIED WARRANTIES, INCLUDING, WITHOUT LIMITATION, THE IMPLIED
WARRANTIES OF MERCHANTABILITY AND FITNESS FOR A PARTICULAR PURPOSE.

\end{enumerate}

\subsection{Henry Spencer}
\begin{center}
Copyright \copyright 1986, University of Toronto.
\end{center}

     Permission is granted to anyone to use this software for any
     purpose on any computer system, and to redistribute it freely,
     subject to the following restrictions:

\begin{enumerate}

     \item The author is not responsible for the consequences of use of
             this software, no matter how awful, even if they arise
             from defects in it.

     \item The origin of this software must not be misrepresented, either
             by explicit claim or by omission.

     \item Altered versions must be plainly marked as such, and must not
             be misrepresented as being the original software.

\end{enumerate}


\index{copyright!regular expressions|)}

\section{Command-line Editing}
\index{copyright!command-line editing|(}
\index{Turner, Simmule}
\index{Salz, Rich}

The command-line editing software in {\em Glish} is based on the
{\em editline} library. This is a {\bf\sc Modified} version of the
library. It was modified to work in a non-blocking, interrupt driven
environment.

\begin{center}
Copyright \copyright 1992,1993 Simmule Turner and Rich Salz. \\
All rights reserved.
\end{center}

\begin{sloppy}
\begin{quotation}

This software is not subject to any license of the American Telephone
and Telegraph Company or of the Regents of the University of California.

Permission is granted to anyone to use this software for any purpose on
any computer system, and to alter it and redistribute it freely, subject
to the following restrictions:

\begin{enumerate}
\item The authors are not responsible for the consequences of use of this
   software, no matter how awful, even if they arise from flaws in it.
\item The origin of this software must not be misrepresented, either by
   explicit claim or by omission.  Since few users ever read sources,
   credits must appear in the documentation.
\item Altered versions must be plainly marked as such, and must not be
   misrepresented as being the original software.  Since few users
   ever read sources, credits must appear in the documentation.
\item This notice may not be removed or altered.
\end{enumerate}
\end{quotation}
\end{sloppy}
\index{copyright!command-line editing|)}
