% $Header: /home/cvs/casa/code/doc/reference/Glish.dir/bugs.tex,v 19.0 2003/07/16 04:18:42 aips2adm Exp $

\chapter{Bugs} \label{bugs}
\index{bugs!known|(}

We list here the known \emph{Glish} bugs:
%\begin{sloppy}
\begin{enumerate}

\item There is a limit on how much output can be generated by a
synchronous shell command.

\item The \texttt{asin}, \texttt{acos}, and \texttt{atan} math functions don't
yet work for \texttt{complex} arguments.

\item Invoking \texttt{max}, \texttt{min}, or \texttt{range} on an empty
vector indexed by an empty vector, such as \texttt{min([])}, returns
spurious results.

\item The interpreter sometimes gets confused as to whether what
has been typed in so far ends a statement or should be continued.
This is particularly prevalent with entering ``\texttt{if}" statements
in interactive mode.

\item When the \emph{Glish} interpreter dies, sometimes some of the clients it
created continue running.

\item Event values sent to or from clients cannot contain \texttt{function}'s
or \texttt{agent}'s.  \texttt{reference} values are first dereferenced.

\item The current precedence is such that \verb+-5^2+ yields \texttt{25},
while probably \texttt{-25} is more intuitive.

\item Error messages don't always identify well the object they
relate to, or the corresponding file.  Also, those that write an
object's value write then \emph{entire} value, which can prove very
annoying for large objects.

\item \emph{Glish} does not do a very good job converting \texttt{string}'s to
\emph{numeric} values.  In particular, it should mark conversion of
a value like ``\texttt{"1.234foo"}'' as erroneous, while allowing
automatic conversion of a value like ``\texttt{"1.234"}'' (i.e.,
no explicit use of \texttt{as\_double()} required).

\item Printing of values by the \emph{Glish} interpreter is sometimes
messy to the point of being unreadable (particularly printing
function values).

\item If the \emph{Glish} interpreter tries to contact a remote host to
run \emph{glishd} and is unsuccessful, it does not recover gracefully.

\item This manual needs a companion manual documenting
the \emph{Glish} internals.

\end{enumerate}

%\end{sloppy}

\index{bugs!known|)}
