\setlongtables
\begin{longtable}[c]{|l|}
\caption{User Script gaussfit.g}\label{gauss} \\
\cline{1-1}
\endfirsthead
\multicolumn{1}{l}{\hspace{9mm}\footnotesize{\slshape
 continued from previous page}\hfill{User Script gaussfit.g}} \\
\cline{1-1}
\endhead
\cline{1-1}
\multicolumn{1}{r}{\small \slshape
 continued on next page} \\
\endfoot
\cline{1-1}
\endlastfoot
{\slshape\small}
\verb|#| \\
\verb|# gaussfit: function to fit Gaussians and plot the results| \\
\verb|#   | \\
\verb|#   sdrec: SD record (default takes current spectrum)| \\
\verb|#   nfit:  number of Gaussians to fit| \\
\verb|#| \\
\verb|gaussfit := function(sdrec=F, nfit=1)| \\
\verb|{| \\
\verb|    # plot the spectrum if not already in the plotter| \\
\verb|    if(sdrec){gbssa.plotscan(sdrec)}| \\
\verb|| \\
\verb|    # define the initial guess| \\
\verb|    print 'Enter the initial peak, height, and center (e.g., 25 2 185)';| \\
\verb|    dum := readline();| \\
\verb|    line := split(dum);| \\
\verb|    peak := as_double(line[1]);| \\
\verb|    width := as_double(line[2]);| \\
\verb|    center := as_double(line[3]);| \\
\verb|| \\
\verb|    # calculate the Gaussian model| \\
\verb|    gbssa.gauss(peak, width, center, scanrec=sdrec);| \\
\verb|| \\
\verb|    return T;| \\
\verb|}| \\
\end{longtable}
