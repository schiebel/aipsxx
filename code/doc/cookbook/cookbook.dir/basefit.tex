\setlongtables
\begin{longtable}[c]{|l|}
\caption{User Script basefit.g}\label{base} \\
\cline{1-1}
\endfirsthead
\multicolumn{1}{l}{\hspace{9mm}\footnotesize{\slshape
 continued from previous page}\hfill{User Script basefit.g}} \\
\cline{1-1}
\endhead
\cline{1-1}
\multicolumn{1}{r}{\small \slshape
 continued on next page} \\
\endfoot
\cline{1-1}
\endlastfoot
{\slshape\small}
\verb|#| \\
\verb|# basefit: function to fit baselines and display the data| \\
\verb|#   | \\
\verb|#   scan:     scan number| \\
\verb|#   subscan:  subscan number| \\
\verb|#   nfit:     order of polynomial fit| \\
\verb|#| \\
\verb|basefit := function(scan, subscan, nfit=1)| \\
\verb|{| \\
\verb|    # get the spectrum and then send it to the plotter| \\
\verb|    y_data := gbssa.getscan(scan, subscan);| \\
\verb|    gbssa.plotscan(y_data);| \\
\verb|| \\
\verb|    # set the baseline region| \\
\verb|    print 'Set the baseline region in channels (e.g, [x1:x2][x3:x4]....)';| \\
\verb|    nrset := readline();| \\
\verb|| \\
\verb|    # calculate the polynomial model| \\
\verb|    y_model := gbssa.baseline(scanrec=y_data, order=nfit, \| \\
\verb|               action='show', range=nrset);| \\
\verb|    x := readline();| \\
\verb|| \\
\verb|    # subtract the polynomical from the data| \\
\verb|    y_fit := gbssa.baseline(scanrec=y_data, order=nfit, \| \\
\verb|             action='subtract', range=nrset);| \\
\verb|| \\
\verb|    return T;| \\
\verb|}| \\
\end{longtable}
