\setlongtables
\begin{longtable}[c]{|l|}
\caption{User Script calout.g}\label{calout} \\
\cline{1-1}
\endfirsthead
\multicolumn{1}{l}{\hspace{9mm}\footnotesize{\slshape
 continued from previous page}\hfill{User Script calout.g}} \\
\cline{1-1}
\endhead
\cline{1-1}
\multicolumn{1}{r}{\small \slshape
 continued on next page} \\
\endfoot
\cline{1-1}
\endlastfoot
{\slshape\small}
\verb|#| \\
\verb|# calout: function to calibrate the data and then write it out| \\
\verb|#         to a new data set.| \\
\verb|#   | \\
\verb|#   bscan:  first scan number| \\
\verb|#   escan:  last scan number| \\
\verb|#   nss:    number of subscans| \\
\verb|#   nif:    number of IF channels| \\
\verb|#   nphase: number of phases (includes all IFs)| \\
\verb|#   tcal:   noise tube temperature [K]| \\
\verb|#| \\
\verb|calout := function(bscan, escan, nss=10, nif=2, nphase=4, tcal=1.5)| \\
\verb|{| \\
\verb|    for (i in bscan:escan)| \\
\verb|    {| \\
\verb|        for (j in 1:nss)| \\
\verb|        {| \\
\verb|            for (k in 1:nif)| \\
\verb|            {| \\
\verb|                print i,j,k;| \\
\verb|                x := tant(i, j, k, nphase, tcal);| \\
\verb|                gbssa.save(x);| \\
\verb|            }| \\
\verb|        }    | \\
\verb|    }| \\
\verb|| \\
\verb|    return T;| \\
\verb|}| \\
\end{longtable}
