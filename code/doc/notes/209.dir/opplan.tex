\newcommand{\specsURL}{../../specs/specs.html}
\newcommand{\consortiumURL}{../consortium.html}
%
% URL to Help system
%
\externallabels{../../user/Refman}{../../user/Refman/labels.pl}

% Add home page navigation button
%

\htmladdtonavigation{\htmladdnormallink
  {\htmladdimg{../../gif/home.gif}}{{../../html/aips++.html}}}

\section{Purpose}

The purpose of this document is to lay out a vision for operating
AIPS++ in the long-term once the first release is made. This document
should serve as a starting point for discussion in a number of
different forums: the AIPS++ Project, the AIPS++ Executive Committee,
and the AIPS++ Scientific and Technical Advisory Group.

\section{Introduction}

In the latest development plan for AIPS++ (\htmladdnormallink{Note
202}{http://aips2.nrao.edu/aips++/docs/notes/202/202.html}), a goal is
that it be a functional equivalent to the existing consortium packages
by late 1999. At that point, AIPS++ will be a fully functional system,
in use for production at a large number of astronomical (and
non-astronomical) sites. To get some idea of the scale of the
distribution that is plausible, we can sum the existing sites for
consortium packages: AIPS, Unipops, SDE, Miriad, Newstar. This is in
excess of 200 active sites. Presumably the majority of the active
users will be astronomers doing science, but we should not forget that
if AIPS++ is successful, then some fraction of these sites will host
active AIPS++ programmers. AIPS++ will be the mechanism for
observatories to provide support for astronomers using their
telescopes, much in the same way that IRAF is used as a vehicle for
data reduction packages for a number of organizationally distinct
telescopes. This is different from the existing AIPS model, where one
observatory, NRAO, has control. A close, but much smaller scale, model
inside the consortium is the use of MIRIAD by BIMA and the ATNF.  It
is thus important that AIPS++ be clear from the beginning what type of
role it will play in astronomy. We see it mainly as a platform or
vehicle for a number of clients to use in their work.

Let us turn, therefore, to focus on the clients that AIPS++ will have in
this operational phase. These clients split naturally into a number of
categories.

\begin{description}
\item[Observatories] Observatories will use AIPS++ as the vehicle for
providing support for particular telescopes. This will presumably have
a number of components:
\begin{description}
\item[Software] for observing preparation, monitoring of observations,
and reduction of data.
\item[Distribution] of ancillary data for the telescope {\em e.g.}
source lists, antenna coordinates, instrument history, physical 
parameters (such as ionospheric menasurements), {\em etc.}
\item[Support] for astronomers using the telescope.
\end{description}
\item[Astronomers] Astronomers will use AIPS++ for access to
a common telescope reduction package, and for analysis tools that are
not the specific responsibility of any one observatory. One can also
envisage that AIPS++ will become a mechanism for collaboration and for
publication, but we pay less attention to these possibilities here.
\item[Programmers] Programmers (who may be active astronomers) will use AIPS++
as a resource for tools to solve problems and as a distribution mechanism for
their products.
\end{description}

The overall goal of the AIPS++ package must be to aid science. As a
working assumption, we will take it that all of these classes of
clients are equally important in furthering that goal, and thus are
equally deserving of support by AIPS++.  We should thus aim to be as
responsive as possible to all of these clients. Perhaps the most
difficult problem of operations will be to find suitable compromises
between the imperatives of all these different clients. Finding the
right balance between conservatism and innovation will require a firm
hand. We return to this problem below.

In the rest of this document, we describe the operations of AIPS++ in
more detail. We see the operations of AIPS++ as splitting into a
number of components. First, there are {\em core operational
functions} that must be performed in order for the package to be
installable and usable at any consortium site. Second, there is {\em
astronomer and programmer support}, needed to support the use of the
system by astronomers and programmers alike. Third, there must be a
{\em quality assurance program} that ensures and tracks the overall
quality of the system.  Overseeing, coordinating and monitoring all of
these functions is a {\em management and oversight} component. These
are discussed in turn in the following sections.

\section{Core operational functions}

The core operational functions for any package are {\em maintenance,
distribution}, and {\em development}.

\subsection{Maintenance}

Maintenance of the package is needed to discover and eradicate bugs, and
to track environmental changes such as operating systems changes
and compiler changes.
\begin{description}
\item[Bugs] In any system, a continuing effort is needed to uncover,
track, and eradicate bugs. For the mechanics of bug tracking, we
currently use the GNATS package from the Free Software
Foundation. This seems to be adequate currently but we may have to
consider migration to another package if the current approach does not
scale up.  The current system has two potential bottlenecks. First,
all bug reports currently go through the Project Center, and second,
the bug reports are then classified by one person. We have to work to
eliminate both these bottlenecks once the flow of bug reports
diversifies and increases. A quick response to bug reports is vital to
avoid the ``stale bugs'' problem whereby a delay in responding to bug
reports means that many bugs have been fixed or are no longer
relevant.
\item[Tracking environmental changes] The environment under which AIPS++
executes will inevitably change over time as the underlying operating
systems evolve and as improved version of tools such as compilers
become available. An extreme example of an environmental change
that must be monitored is the advent of Java and all it portends
for computing in the future.
\end{description}

\subsection{Distribution}

\begin{description}
\item[System] In the current setup, the core AIPS++ system is maintained 
at the Project Center in Socorro, New Mexico. 

System distribution for developers occurs via two mechanisms. First,
periodically the system is distributed to sites over the Internet,
either incrementally or cumulatively.  This is usually done in a batch
job, and uses ftp for transmission.  Second, changes to the master are
performed via an NFS mount of the master disks onto a local machine.
The first function, downloading, is highly reliable and essentially
presents no significant problems.  The second function, uploading, is
more problematical since the Internet is sufficiently congested that
the necessary NFS mount is hard to acquire and maintain.  This will be
alleviated with the forthcoming modifications to NFS to use TCP/IP in
place of simple UDP datagrams.  Inside NRAO, we rely upon our own
intranet in place of the Internet.  Outside NRAO, we have the option
of bypassing the mounting process by ftp'ing files directly to an NRAO
computer.

For end-users, the system is currently distributed as both binaries
or source by anonymous ftp.  Our beta testing has shown this to be a
simple, effective and reliable approach.  About 90\% of AIPS
distribution is via ftp. However to service some communities, we will
also distribute via two hardcopy forms: CD-ROM and various tapes: 8mm
and DAT initially.

\item[Data] Data in AIPS++ fall into a number of categories. First,
there are data from telescopes.  Second, there are data needed for the
software to execute. Examples would be databases needed for the
Measures system. Third, there are databases that aid the user in some
process. Examples would be source parameters, antenna locations,
scheduling information, {\em etc.}  Fourth, there are data needed for
testing and demonstration purposes.  We are currently determining
requirements for the latter three, and plan to put in place mechanisms
for users to acquire such data either from the AIPS++ Project Center,
or directly from the original source.

\item[Knowledge] Knowledge about astronomical processing and analysis
should be one of the prime commodities that AIPS++ can help provide to
its clients. We should distinguish between knowledge generated by
AIPS++, and knowledge funneled through AIPS++ mechanisms. In the
former category we see conventional documentation such as reference
manuals, cookbooks, tutorials and migration guides ({\em ``AIPS++ for
AIPS users''}).  In the latter category, we think that AIPS++ should
provide mechanisms for users to talk to other users, via exploders,
list-servers, news-letters, {\em etc.}. The major value added by AIPS++
in this latter category would be a common framework in which the
discussions can be couched, and an archive which can be searched (a
database for knowledge). Thus, for example, the NRAO Synthesis
Imaging workshops should be strongly coupled with AIPS++
documentation.

\item[Schedule] The update schedule will be one of the most contentious
aspects of operating AIPS++. We can imagine that a dynamic group in a
phase of rapid development of some sub-system in AIPS++ will push for
a rapid release schedule whereas more established groups with less
rapidly evolving needs will prefer a conservative approach to
releases. For help in understanding this issue, it is instructive to
consider the evolution of AIPS during the development of software for
the VLBA. During this time, the AIPS software was evolving rapidly in
response to a deluge of new data (and concommitant problems to be
solved) from the VLBA correlator. Groups tightly tied to use of the
VLBA therefore wished to update their own working copy of AIPS more
often than allowed by any reasonable formal release schedule. The
answer was for them to be placed on the AIPS Midnight Job whereby the
working version is updated every night. The advantage of this is
obvious. The disadvantages are that the system is then more unstable,
and, more seriously, that it is impossible to determine the state of
the system on any given date.  However, if an actual release is also
available then these bad effects can be reduced. This may form a good
model: the center should release stable, well-tested releases a couple
of times per year, and then allow users to be placed on a periodic
update schedule as dictated by their needs.

\end{description}

\subsection{Development}

From experience with other systems, it is certain that development of
AIPS++ will continue until (or even after) it is supplanted by another
system. Historically, development of an existing production system has
been an extremely delicate proposition. To add new capabilities
without compromising those existing capabilities was difficult, and
the degree of success that AIPS++ has in the area will be the key
measure of the success of the object-oriented design of AIPS++. It is
apparent from our (and industry) experience that object-oriented
methods work well if high-level interfaces do not have to drift over
time.

\begin{description}
\item[Core library] Any significant continuing development of the core library
after the initial programmer's release will need careful
consideration.  Additions to a widely-used interface are often
acceptable but the cost of changing an interface can be very
large. This means that the core at the time of the first developers
release, the core libraries should be up-to-date and
state-of-the-art. In practice, if a core library {\em is} changed, we
think that the resulting cost of changing code has to be borne by the
Project Center since otherwise the cost to end-user programmers of
tracking continuing changes will be a strong disincentive to use the
system.
\item[Applications] Development of applications will continue
throughout the lifetime of the system, and is clearly to be 
encouraged! Here the most difficult problem is deciding which 
pieces of an application can be moved to the core library.
Advice on priorities for this process can come from our clients
but detailed advice on what to migrate when must come from a
technically knowledgable group. Integration of a new package will be
viewed most favorably if the code obeys our coding and documentation
rules, and if the expected lifetime of the package is more than
one year.
\item[System] Changes in the implementation of the system library 
will be necessary over time for a large number of reasons: operating
systems change, compilers improve, new devices come along ({\em e.g.}
better long-term storage devices, improved visualization devices, {\em
etc.}). Determining which configurations will be supported is a
function of the Project Center, but advice is particularly important
here.
\end{description}

Development entails hard choices. These choices have to be made
bearing in mind the needs of potentially all the clients, and making
sensible guesses as to which lines of development should be followed
with which priority. As we said above, this will probably be one of
the most contentious areas in operating AIPS++. We think that a
diverse but tightly knit group would be best suited to advising AIPS++
management on such issues.  We return to this issue below.

\section{Astronomer and Programmer support}

\subsection{Help}

We see help for clients of AIPS++ as being layered. We strongly
suggest that each AIPS++ consortium site have a designated contact
person who can also provide immediate, in-person advice to both
astronomers and programmers on how to use the system.  For the next
layer of help or for queries from non-consortium sites, there are a
couple of possibilities: either set up a special group to provide
detailed help, or adopt something similar to the {\em Designated AIP}
program of the AIPS system whereby such duties are rotated around the
active programmers in the Project, perhaps to three at one time, one
per 8 hour shift? The latter has a number of attractions: it keeps the
programmers in touch with the users, the level of advice is high, and
it spreads a difficult and demanding job amongst a group of
people. The downside is that it takes some fraction of the time of the
best people. Overall, we prefer the latter approach and recommend that
it be adopted.

\subsection{Education and training}

\begin{description}
\item[Users workshops] For the first few years of operation of
AIPS++, it will be important to hold training sessions for
astronomical users. This is probably best considered as a
responsibility of the consortium partners so that, for example, NRAO
would offer a workshop for the users of NRAO telescopes. The role of
the AIPS++ Project might be to send core staff to help with those
workshops.  
\item[Tutorials and demonstrations] Computer-based tutorials and
demonstrations are an effective way to train end-users both in
concepts of astronomical observing and in AIPS++ data reduction.
Computer-based conferencing may also eventually allow effective
training to proceed year round, but this needs some investigation.
\item[Developers workshops] Programmers need training at a number of
levels: programming in Glish, in C++ using our libraries, adding to
our libraries, and finally sub-system development. We envisage that
all of these would be covered in annual developers workshops.
\end{description}

\subsection{Feedback}

We see user feedback as coming via a number of mechanisms. Immediate
feedback comes via email, phone calls, and bug reports. This feedback
and any related replies should be logged in a generally accessible
form. Thus for purposes of logging, we will prefer feedback via
e-mail. Longer term feedback comes via User Group meetings at which
representative clients of AIPS++ can comment on current capabilities
and future development priorities.

\subsection{Library and Contributed code}

The code in AIPS++ is organized into a number of conceptually
different types of repositories. Admittance of code to these repositories
is controlled by a number of different policies. The classes of
repositories are:

\begin{description}
\item[core libraries] {\em aips, synthesis, dish, vlbi, doc} These contain the core
library code and documentation of AIPS++. Admittance of code and
documentation to one of these directories requires passing our code
review process. The AIPS++ Project is responsible for maintaining code
in these libraries.
\item[documentation library] {\em doc} The core documentation source is kept here.
We currently have no policy on admittance to this area. This is a topic that
we will  have to review, and possibly move to package level documentation
{\em package/doc}.
\item[consortium libraries] {\em atnf, bima, nfra, nrao} These are repositories for
consortium site specific code. AIPS++ has no policy on whether these are subject to
code-review, but we do demand that coding, documentation and other standards
be obeyed. The consortium sites are each responsible for maintenance of this
code.
\item[contributed library code] {\em contrib} This is a repository for contributed
code that is either unsupported or supported by an individual author. A minimal
set of code standards probably needs to be enforced {\em e.g.} some programmer
documentation in our standard format, and some user documentation (if appropriate),
optionally in our standard format. We currently have no code in this category.
\item[ad hoc library code] {\em trial} This is a way-station repository for code that
is in development by consortium programmers and is not yet ready for
admittance to a core library. There is no admittance limitation, and
the individual programmers are responsible for maintenance.
\end{description}

It is in the interests of AIPS++ that good quality, useful code
migrate from the peripheral libraries such as {\em contrib} and {\em
atnf, bima, nfra, nrao} into the core libraries {\em aips, synthesis,
dish, vlbi, doc}. 

\section{Quality Assurance}

Traditionally quality assurance for software packages has been either
diffused over an entire organization or, more commonly, done by the
users. In keeping with the large scope of the AIPS++ Project, we have
decided to designate a Quality Assurance Group that has special
responsibility for ensuring the quality of official AIPS++
products. The main task for the QAG is assuring the high quality of
AIPS++ products, principally via testing, including execution of
software, reviewing code and documentation.

\subsection{Testing}
\begin{description}
\item[Unit testing] Currently the code review process requires that
along with submitted code, unit-test software is also submitted.
The complete suite of unit tests is run once a week
as a check of the basic integrity of the system. We have found this 
to be extremely useful, and see no reason not to continue it.
\item[Application testing] Application testing is currently
weak. Some test scripts are present in the system but these are few in
number. We think that two orthogonal approaches are needed. First we
must perform the equivalent of periodic unit testing checking
known inputs against expected outputs. Second, we
should assign a group to perform blind-testing of applications,
to trap unexpected behavior not caught by unit-testing.
\item[Package testing] Does a given package do all that is
required? Is the model used an appropriate one? Are there
better models available? This type of high-level review will
require a collaboration between the QA group and an expert in
the field. One might think of this as being similar to a review
of a technical textbook.
\item[Benchmarking] The AIPS Project found that benchmarking of
new systems using a pre-determined set of applications was
extremely useful both for determining possible computer purchases
and for uncovering systematic performance problems. They use a 
set of Dirty Dozen Tasks to characterize the performance.
\end{description}

The Quality Assurance Group will be initially consist of three people,
with the following division of responsibility:

\begin{description}
\item[Code Cop]
\begin{itemize}
\item Ensuring that code is reviewed in a timely manner
\item Ensuring that module and code documentation is kept up to date 
\item Ensuring that unit testing is performed
\item Updating external libraries when necessary
\item Maintaining the RCS and Change log entries so that they are useful
\item Maintaining the directory hierarchy
\end{itemize}

\item[Rules Boss]
\begin{itemize}
\item Establishing new coding standards and removing outdated ones
\item Maintaining templates used for aips++ development
\item Controlling development of the AIPS++ directory hierarchy
\item Maintaining and policing the file extensions used
\item Controlling inclusion of external libraries
\end{itemize}

\item[Chief Tester]
\begin{itemize}
\item Responsible for user level documentation
\item Ensuring that that applications testing, both automated and
  interactive, is performed
\item Maintaining benchmarks and performing new ones
\item Has final sign-off on AIPS++ releases
\end{itemize}
\end{description}

The head of the QAG will report directly to the Project Manager.

\subsection{Code Reviews}

Our code review process has been in place for some time now and seems to work
well for the current size of the Project. The process is documented in 
\htmladdnormallink{Note 167}
{http://aips2.nrao.edu/aips++/docs/notes/167/167.html}.
A {\em code cop} is currently responsible for overseeing the code review
process. As the Project grows, this role would probably better be served
by a group of people, perhaps in the above-mentioned Quality Assurance 
Group.

We need to augment the code review process with design reviews to catch
problems early on in the design-implement-review-use cycle. We may also
wish to use code walkthroughs or inspections.

\section{Management and Oversight}

It is our view that the current management structure works well, and
strikes a reasonable balance between autonomy for and oversight of the
Project Manager. Thus any change to the structure must be motivated by
the transition to operations. Lines of command and responsibility
must remain clear, both inside the consortium sites and within the
Project. Thus the reporting line of {\em Executive Committee - Project
Manager - Site Managers - Project staff} should be maintained if at
all possible. We note that the lack of a formal line of responsibility
of Site Managers to Project Manager works satisfactorily in most cases.
It will be a challenge to maintain this aspect under the increased
pressure of operations.

Although we hesitate to suggest an overly structured management
scheme, we do think that some formal recognition of the de facto
sub-structuring of the Project is probably overdue. We have group
leaders in various areas: System, Core library, Synthesis, Single
Dish, Visualization, {\em etc.} Currently most activities in these
groups are monitored at a detailed level by the Project Manager. In
the long-term, this is not feasible and we need to move to a situation
where the sub-group leaders take more explicit responsibility for
setting, assigning, and monitoring targets. A common philosophy and
framework for management within the sub-groups can be imposed by the
Project Manager.

In addition, we need to add explicitly an Operations Manager who
oversees the day-to-day operations, freeing the Project Manager to
concentrate more on longer-term issues. This Operations Manager would
report to the Project Manager.

Advisory and oversight roles should be added to this proposed
structure. From the discussion in the previous sections, we believe
that two advisory groups are required:

\begin{description}
\item[AIPS++ User Group] This group of representative users of AIPS++ 
meets annually (or perhaps bi-annually initially) to advise the AIPS++
Project management.  It should be a large group with representatives
of all three classes of clients: Observatories, Astronomers and
Programmers. Moderate and long term issues are discussed in this
group.
\item[AIPS++ Technical Working Group] This is an internal group that
meets often, perhaps monthly or quarterly, to advise the Project
Management on detailed technical issues. It is here that short-term
development issues are discussed. We recommend that this be composed of
no more than a dozen people drawn from the inner circles of the Project
and thus limited to those with deep knowledge of working inside the
system.
\end{description}

These two advisory groups report directly to the Project Manager. Note
that these roles are currently fulfilled by the Scientific and Technical
Advisory Group that reports directly to the Project Manager. We think the
STAG should evolve into these two groups over a few years.

To provide oversight on overall project management issues, we suggest
that the Executive Committee may wish to appoint another group that
reports directly to it on management of the project. We suggest that
this group be small and be composed of more senior people with direct
experience in astronomical software and astronomical software management.

\section{Resources and Timescales}

The current AIPS++ staff is quite large: in aggregate, we have 25
people contributing just over 18 FTEs to the AIPS++ Project. Of these,
11 are employed by NRAO, and contribute 9.1 FTEs. The numbers for the
other partners are: ATNF 4 and 2.45, BIMA/NCSA 5 and 3.25, NFRA 5 and
3.25. We think that some of these staff should participate in the
general shift of the Project from development to operations, but we
hesitate to prescribe how this should be done. The new resources
needed to implement this plan are:

\begin{itemize}
\item Operations Manager: providing the Operations Manager is probably best 
seen as an NRAO responsibility that is part of hosting the Project Center.
Such a person should be appointed as soon as possible. The initial prime
responsibility of the OM will be to implement this operations plan.
The Operations Manager will also handle the mechanics of distributing AIPS++
via the mechanisms described above.
\item Documentation group, headed by an astronomer, to oversee the
production and maintenance of high quality documentation, including
demonstrations and tutorials.
\item Quality Assurance Group: could be located at any location. Location 
at the Project Center would have some minor advantages. Establishment of
a QAG could proceed over the next year or two. 
\item Support people at each site: this requires a migration towards a
different type of expertise, more astronomically based than currently
true. We recommend that the consortium sites review their staffing
in light of this need.
\end{itemize}


\section{Summary}

The development of AIPS++ is ambitious. We have chosen to be similarly
ambitious in our plans for operation. An estimate of the staffing
required is about 6-10 FTEs to operate as we have described.  {\em
This is in addition to the staff needed for any ongoing development of
AIPS++, which we anticipate to be substantial.}  Many of these people
can be drawn from existing operational staff at the various
observatories.  A bump in staffing around the time of the transition
into operations is probably inevitable as we finish development of the
core systems, and start the preparations for operations considered
here. We anticipate that the development staff will drop in number as
the core solidifies. Many of these staff can be moved onto either
Operations or to development for specialized telescopes such as the
NRAO MMA or the SKAI.

The Quality Assurance Group will be set up immediately (November
97). The initial assignments will be Code Cop: Ralph Marson, Rules
Boss: Wim Brouw, Chief Tester: Jan Noordam. Ralph Marson will
direct activities of the QAG, reporting directly to the Project
Manager.
