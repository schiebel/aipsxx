\def\eps@scaling{.95}
\def\epsscale#1{\gdef\eps@scaling{#1}}
\def\plotfiddle#1#2#3#4#5#6#7{\centering \leavevmode
    \vbox to#2{\rule{0pt}{#2}}
    \special{psfile=#1 voffset=#7 hoffset=#6 vscale=#5 hscale=#4 angle=#3}}

%
% jmcmulli
% gbtlogview.tex,gbtlogview.ps
% psiz=6000KB
%
% 1997 APR 07
% Guide to gbtlogview
% gbtlogview: veni, vidi, perscrutatus sum
% Joseph P. McMullin
% Green Bank
% West Virginia
%
\newcommand{\thisdocURL}{http://aips2.nrao.edu/aips++/docs/notes/202/202.html}
\newcommand{\searchURL}{../../html/aips2search.html}
\newcommand{\aipsviewURL}{../../user/aipsview/aipsview.html}
\newcommand{\GlishmanualURL}{../../reference/Glish/Glish.html}
\newcommand{\GlishtutorialURL}{../../notes/195/195.html}
\newcommand{\synthesisURL}{../../user/synthesis/synthesis.html}
\newcommand{\gsURL}{../../user/gettingstarted/gettingstarted.html}
\newcommand{\userrefmanualURL}{../../user/Refman/Refman.html}
\newcommand{\specsURL}{../../specs/specs.html}
\newcommand{\betaURL}{../beta/beta.html}
\newcommand{\consortiumURL}{../consortium.html}
\newcommand{\gbthomelink}{http://www.gb.nrao.edu/GBT/GBT.html}
\newcommand{\vlahomelink}{http://www.nrao.edu/doc/vla/html/VLAhome.shtml}
\newcommand{\cvhomelink}{http://www.cv.nrao.edu/cv-home.html}

%
% URL to Help system
%
\externallabels{../../user/Refman}{../../user/Refman/labels.pl}

% Add home page navigation button
%

\htmladdtonavigation{\htmladdnormallink
  {\htmladdimg{../../gif/home.gif}}{{../../html/aips++.html}}}

\section{Purpose}
gbtlogview is a valuable resource to the engineering and technical staff at
Green Bank. This note is indended to introduce and demonstrate the utility
of the program along with various scripts available for use.
\bigskip
\section{Introduction}
The program, gbtlogview, is a tool for listing and plotting AIPS++ table
format data. 

\section{Running gbtlogview}

gbtlogview is currently in a state of flux, as AIPS++ versions are evolving.
Most of these changes will be transparent to the user in terms of invoking
the program. Starting gbtlogview, including setting the various paths, etc,
is done by:

\begin{verbatim}
scylla<27>% gbtlogview2                         (new version of AIPS++)
\end{verbatim}

or

\begin{verbatim}
scylla<27>% gbtlogview2                         (test version of AIPS++)
\end{verbatim}

For those interested in the details, the gbtlogview2 script is the following:

\begin{verbatim}
/opt/local/bin/gbtlogview2

#!/opt/local/bin/bash
 
# Starts test version of aips++ GUI for engr logs.  RDN 6/20/96
 
source /aips++/new/aipsinit.sh
# aipsinit _
gbtlogview
\end{verbatim}

The gbtlogview script called by gbtlogview2 is:

\begin{verbatim}
gbtlogview : /aips++/new/sun4sol\_gnu/bin/gbtlogview

#!/bin/sh
#
.
.    Copyright comment text has been cut out.
.
#
#   $Id: gbtlogview.tex,v 19.0 2003/07/16 03:48:03 aips2adm Exp $
#-----------------------------------------------------------------------------

# in cv
#INSTALL_PATH=./
# in gb
INSTALL_PATH=/aips++/local/bin

xrdb -merge $INSTALL_PATH/xres
glish -l aips++.g -l gbtlogview.g
\end{verbatim}

This sets up an addition to the X-server resource database and then executes
the glish scripts aips++.g and  gbtlogview.g 
(in /aips++/version/sun4sol\_gnu/libexec).

%Currently, this is the only version of gbtlogview available (gbtlogview1 is
%no longer functioning). This version
%retains the xrt plotter favored by the GBT staff. In the future, there will be
%a gbtlogview3, which will call the PGPLOT plotter.

After executing 'gbtlogview2', the user is then in the glish environment in 
the executing window, while
a "GBT Log Data" GUI appears (Figure ~\ref{fig:gbtlogdata}). The GUI is
intended as the main interface. The frame labeled 'Data Source(s)' has
a list of data {\it clients} for examination, including various receivers
and the local weather station. Each or all of these sources may be
selected (by clicking on them with the mouse). The next frame over features
several toggle buttons which control the range of selected data. When
one of these buttons is selected, for example the "Last Hour" button,
information on the start and end time and output table name are automatically
obtained. To actually construct a table of values from the selected 
quantities, the 'Fill' button takes the data from the selected {\it clients}
and produces a flat table of data records with all of the relevant 
information. Figure ~\ref{fig:fillerstatus} displays the window that opens
to show the filler activity.

Two Data Sources of special note are "Specify Device" and "Specify ASCII 
table". "Specify Device" allows the user to specify the directory where
a non-standard device's data files are stored. Pressing the "Fill" button
in this case brings up a frame where you can type the name of the 
directory (e.g. one can access a normal Rx this way as well, 
Directory: /GBTlogs/Receivers/RC04\_6). It will then grab standard data
files according to the time range specified.

"Specify ASCII table" allows the user to specify an ascii table on
disk. Pressing the "Fill" button, brings up a frame where you specify
the directory and table name (e.g. Directory: /scylla/jmcmulli/ [note
the necessary trailing slash], Filename: test.inp). The format of the table
must be in the form:

\begin{verbatim}
Col1	Col2	Col3	Col4	Col5	Col6	# Column titles
I       I       I       R       R       R       # The format of the column:
                                                #I=int,R=real,A=string
                                                #D=double,X=complex(R,I)
                                                #Z=complex(Amp,phase)
1       4       9       3.45    5.46    7.89    # These are the actual data
2       2       7       2.1     0.1     4       # values.
\end{verbatim}

These two special devices are mutually exclusive.

\begin{figure}
\begin{flushleft}
   \plotfiddle{gbtlogdata_2.ps}{5.5in}{0}{50}{85}{-250}{-150}
\end{flushleft}
\end{figure}

\begin{figure}
\begin{flushleft}
   \plotfiddle{fillerstatus.ps}{5.5in}{0}{85}{85}{-250}{-150}
\end{flushleft}
\end{figure}

The 'Browse' button pops up a window with all of the selected data sources'
information in table format. The table may be scrolled through with the
scroll bars but plotting isn't currently enabled from this window.

The 'Plot Columns' window (Figure ~\ref{fig:plotlogdata}) has all of the
data quantity names in three columns (X, Y, and Y2). Clicking on 
a quantity selects it for that axis. The 'Plot' button then displays
it on the plotter window (Figure ~\ref{fig:gplot1d}). Several Y-axis
items may be selected for comparison, along with an additional Y2-axis
item which will have its axis labeled on the right hand side of the
plot.

\begin{figure}
% \epsfbox{plotlogdata.ps}
\begin{flushleft}
   \plotfiddle{plotlogdata.ps}{5.5in}{0}{65}{85}{-250}{-150}
\end{flushleft}
\end{figure}

\begin{figure}
% \epsfbox{gplot1d_a.ps}
\begin{flushleft}
   \plotfiddle{gplot1d_a.ps}{5.5in}{0}{85}{85}{-250}{-150}
\end{flushleft}
\end{figure}

\begin{figure}
% \epsfbox{gplot1d_b.ps}
\begin{flushleft}
   \plotfiddle{gplot1d_b.ps}{5.5in}{0}{65}{95}{-250}{-150}
\end{flushleft}
\end{figure}

%
% Point to the table module and make examples with
%engineering data. Use Roger's examples to give examples
%of working with gbtlogview. 
%
\section{Tables}

AIPS++ manipulates and stores all data in the form of tables. These tables 
consist of an unlimited number of columns of data with optional column
keywords and table keywords. Manipulating these tables is the key to
fully utilizing gbtlogview. Within AIPS++, the table module is equiped
with various methods and functions for this purpose.
Please refer to the 
\htmladdnormallink{table - Module}{http://aips2.nrao.edu/aips++/docs/user/Refman/node13.html#SECTION002110000000000000000} in the Reference Manual.

Initially, after hitting the 'Fill' button, the table "logdata" will be
created on disk in the current directory. To verify this type:

\begin{verbatim}
- tableopentables()
logtable
- table("logtable").colnames()
Time ROW_FLAG RC08_10_DMJD RC08_10_PLATE15K RC08_10_PLATE50K RC08_10_AMBIENT RC0
8_10_DEWARVAC RC08_10_PUMPVAC RC08_10_CRYOSTATECTL RC08_10_XFERSWCTL RC08_10_AMP
PWRCTL RC08_10_IFFILTER RC08_10_CRYOSTATEMON RC08_10_XFER_NOISEMON RC12_18_DMJD
RC12_18_PLATE15K RC12_18_PLATE50K RC12_18_AMBIENT RC12_18_DEWARVAC RC12_18_PUMPV
AC RC12_18_CRYOSTATECTL RC12_18_XFERSWCTL RC12_18_AMPPWRCTL RC12_18_IFFILTER RC1
2_18_CRYOSTATEMON RC12_18_XFER_NOISEMON RC18_26_DMJD RC18_26_PLATE15K RC18_26_PL
ATE50K RC18_26_AMBIENT RC18_26_DEWARVAC RC18_26_PUMPVAC RC18_26_CRYOSTATECTL RC1
8_26_XFERSWCTL RC18_26_AMPPWRCTL RC18_26_IFFILTER RC18_26_CRYOSTATEMON RC18_26_X
FER_NOISEMON Weather1_DMJD Weather1_WINDVEL Weather1_WINDDIR Weather1_AMB_TEMP W
eather1_PRESSURE Weather1_DEWP Gps_DMJD Gps_ID Gps_DAY Gps_DOY Gps_DATE Gps_TIME
 Gps_SV Gps_EL Gps_AZM Gps_SN Gps_AGE Gps_ION Gps_INTRNL Gps_DF_F Gps_AVG_DF_F G
ps_NO1 Gps_TI Gps_TI_FIT Gps_TI_RATE Gps_NO2
\end{verbatim}

The second command, 'table("logtable").colnames()' lists all of the column
names within the table. The list displayed is if all of the receiver, weather
and gps data had been selected; typically, this list will be shorter.

The structure of the table as constructed by the filler is as follows:
\begin{verbatim}
Time    Device1_DMJD  Device1_Quantity_X... Device2_DMJD Device2_quantity_Y...
t1          t1          x @ t1                     t4      y @ t4
t2          t2          x @ t2                     t4      y @ t4
t3          t3          x @ t3                     t4      y @ t4
t4          t3          x @ t3                     t4      y @ t4
t5          t5          x @ t5                     t4      y @ t4
t6          t6          x @ t6                     t4      y @ t4
t7          t6          x @ t6                     t7      y @ t7
t8          t8          x @ t8                     t7      y @ t7
\end{verbatim}

Where the Time column is the running time stamp which will be equal to the
most rapidly sampled variable and t1<t2<tn. 

The filler essentially pads
those columns which are more coarsely sampled with duplicate records until
the monitoring program updates that variable. At that time, the device's
time column "Devicen\_DMJD" will be equal to the "Time" column.

At this point, you can construct a table object within glish (data) and then
manipulate the table according to your needs.

\begin{verbatim}
- data:=table("logtable")
\end{verbatim}

 
\subsection{Examples of using the table-module}

\subsubsection{A simple regridding routine}

Regrid a rapidly sampled variable to the same sampling rate of a more
coarsely sampled variable. Use monitoring of 15 K stage on 8-10 GHz
receiver versus ambient temperatures from the weather station.

\begin{verbatim}
- data:=table("logtable")                 # Build a table object within glish.
- sub:=data.query("Time == RC08_10_DMJD") # Select a subtable of that data
                                          # based on the condition that the
					  # 8-10 GHz time stamp is equal to
					  # the current time (that is, a 
					  # sample has just been taken of
					  # the 8-10 GHz receiver).
- sub.nrows()                             # The number of unique samples of
1432   					  # 8-10 GHz receiver data.
- data.nrows()                            # The original number of samples
86118                                     # based on the most finely sampled
                                          # variable.
- xaxis:=sub.getcol("RC08_10_DMJD")       # Get the time field for the 8-10 
                                          # GHz data.
- xtimeinseconds:=(xaxis-as_integer(xaxis[1]))*86400. # convert from days->sec
- y1axis:=sub.getcol("RC08_10_PLATE15K")  # Get 15 K Plate Temperature.
- y2axis:=sub.getcol("Weather1_AMB_TEMP") # Get Ambient temp from Weather Stat.
- clear() 				  # Clear plot
T 
- timeY(xtimeinseconds,y1axis,"")         # Plot time vs. 15 K plate temp on Y1
0 
- timeY2(xtimeinseconds,y2axis,"")        # Plot time vs. Amb temp on Y2
1 
\end{verbatim}

Clicking on the far right button will blow up the plot screen to full
size. Using the cursor, one can select subregions and see that the
ambient temperature data (which is sampled at approximately a 1.5 second
rate) is now on the 1 minute grid of the 15 K Plate data.

\subsubsection{Count cases greater than a certain value}

If one wanted to know, how many times the 8-10 GHz receiver's 15 K plate
had gone above 17 K.

\begin{verbatim}
- bigtemp:=data.query("RC08_10_PLATE15K > 17.0")
- bigtemp.nrows()
49224 
\end{verbatim}

\subsubsection{Maximum and minimum within a certain time range}

Continuing with the example above, find the maximum and minimum temperature
in the last 3 hours of the data.

\begin{verbatim}
- xstart:=dm.quantity(xtimeinseconds[1],'s') # get the beginning time and 
					     # convert it into a quantity with
					     # units of seconds.
- xstart
[unit=s, value=13098.2014] 
- xstop:=dm.add(xstart,'3.h')		     # set the end time to the start
					     # time plus 3 hours.
- xstop
[unit=s, value=23898.2014]
- platetemp:=data.getcol("RC08_10_PLATE15K") # make an array of all the 15 K
					     # plate temperatures.
- mask:=xtimeinseconds<xstop[value]          # define a boolean mask that is
					     # true only within the time range.
- maxinfirst3:=max(platetemp[mask])          # get the max in this range
- mininfirst3:=min(platetemp[mask])          # get the min in this range
- print mininfirst3, maxinfirst3
15.1367188 16.1132812
\end{verbatim}

\section{rtools4.g}
%rtools4.g has fixed the time to seconds for the gplot1d-pgplot and
%eliminated the setLegendGeometry which is not implemented with this plotter
%

A number of utility plotting functions are available through the glish
script, rtools4.g. The functions allow easy plotting and comparison
of variables of interest.

The utilities currently available are:

\begin{description}
\item{$\bullet$}plotXdewvac(flg): Plots the X-band dewar vaccum reading.
\item{$\bullet$}plotKudewvac(flg): Plots the Ku-band dewar vacuum reading.
\item{$\bullet$}plotKdewvac(flg): Plots the K-band dewar vacuum reading.
\item{$\bullet$}plotXpumpvac(flg): Plots the X-band pump vacuum reading.
\item{$\bullet$}plotKupumpvac(flg): Plots the Ku-band pump vacuum reading.
\item{$\bullet$}plotX15K(flg): Plots the X-band 15K temperature.
\item{$\bullet$}plotX50K(flg): Plots the X-band 50K temperature.
\item{$\bullet$}plotXamb(flg): Plots the X-band ambient temperature.
\item{$\bullet$}plotKu15K(flg): Plots the Ku-band 15K temperature.
\item{$\bullet$}plotKuamb(flg): Plots the Ku-band ambient temperature.
\item{$\bullet$}plotK15K(flg): Plots the K-band 15K temperature.
\item{$\bullet$}plotK50K(flg): Plots the K-band 50K temperature.
\item{$\bullet$}plotKamb(flg): Plots the K-band ambient temperature.
\item{$\bullet$}plotTemp(flg,scale): Plots the Air Temperature from WT1.
\item{$\bullet$}plotDewpt(flg,scale): Plots the Dew Point from WT1.
\item{$\bullet$}plotRH(flg): Plots the relative humidity calculated from temperature.
\item{$\bullet$}plotPressure(flg,scale): Plots the pressure data from WT1.
\item{$\bullet$}plotWindvel(flg,scale): Plots the wind velocity from WT1.
\item{$\bullet$}plotWinddir(flg): Plots the wind direction from WT1.
\item{$\bullet$}plotPpsRackTemp(flg,scale): Plots the rack temperature from OnePpsStatus.
\item{$\bullet$}plotRtpm140Delay(flg):Plots the delay column from Rtpm140.
\item{$\bullet$}plotRtpmOvlbiDelay(flg): Plots the delay column from RtpmOvlbi.
\item{$\bullet$}plotDeltas(ticka,tickb): Plots the OnePps Deltas.
\end{description}
\bigskip

In the new version of AIPS++, 
any of these functions can be accessed after data has been 'filled', as long
as the command 'include rtools4.g' has been executed either interactively
after the glish prompt in gbtlogview, or by setting it up in one's 
.glishrc file.

The syntax for their use is simply:

\begin{verbatim}
- function(flg)

or

- function(flg,scale)
\end{verbatim}

where flg is either blank, (), in which case the plot goes to the Y1 axis,
or flg is (y2), in which case the plot uses the Y2 axis; currently, one
must always plot something to the Y1 axis before the Y2 axis can be used.
If the function takes a 'scale' parameter, this can be either 'f' for
farenheit (functions: plotTemp, plotDewpt, plotPpsRackTemp, and
plotPpsRoomTemp), 'mph' for miles/hour (function:plotWindvel), or
'mmHg' and 'inHg' for millimeters/inches of mercury (function: 
plotPressure); the defaults (either left blank or any other combination
of charactes) are degrees celsius, meters/sec, and mbars, respectively.

The plot will be similar to that provided normally by gbtlogview.

In the 'test' version of AIPS++, these utilities have been integrated 
into a GUI. It can be invoked after data has been filled by:

\begin{verbatim}
- include 'rxtools.g'
X-band data found
Weather1 data found
OnePpsDeltas data found
T 
- call_rx_gui()
F 
\end{verbatim}

Each of the buttons shown in the figure are menus for calling up different
functions, while the entry boxes (Options) allow specification of the 
function inputs. For example, plottemp has two optional specifications
(as shown on the status line), Y2[T,F] and scale(C,F); these indicate
that you can specify whether the temps should be plotted on the Y2 axis
(true or false), and which temperature scale (celsius or fahrenheit) to
use for the plots.

\begin{figure}
% \epsfbox{rxtools.ps}
\begin{flushleft}
   \plotfiddle{rxtools.ps}{5.5in}{0}{85}{85}{-250}{-150}
\end{flushleft}
\end{figure}

\begin{figure}
% \epsfbox{windvel.ps}
\begin{flushleft}
   \plotfiddle{windvel.ps}{6.5in}{0}{85}{85}{-250}{50}
\end{flushleft}
\end{figure}
%

\end{document}
