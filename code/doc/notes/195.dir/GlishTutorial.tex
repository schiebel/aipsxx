
\externallabels{../../user/Refman}{../../user/Refman/labels.pl}

\section{Introduction} 

This document is an AIPS++ applications-specific supplement to
\htmladdnormallink{``The Glish 2.7 User
Manual.''}{../../reference/Glish/Glish.html}.  Glish is both a
programming language and an environment for data acquisition and
analysis.  It contains many of the mathematical features of other
languages such as Fortran and C, and it is particularly strong on
array arithmetic.  To the user, glish becomes AIPS++ or a telescope
control system with the addition of functions and ``clients'' for those
applications.  In fact, both applications may be run from the same
invocation of glish, if so desired.

Here we shall cover the main language features with simple examples
drawn from test data acquired by the GBT hardware systems now under
development.  The emphasis will be on the tools for writing your own
data and control functions rather than on a review of AIPS++ library
functions.  Also, only a few of the built-in glish functions will be
mentioned.  See section \htmladdnormallink{10.12}
{../../reference/Glish/node86.html} of the Glish User Manual for a
complete function summary, and the \htmladdnormallink{AIPS++ User
Reference Manual}{../../user/Refman/Refman.html} for information on the
AIPS++ library. A few elements of the AIPS++ library are used
below; when discussed the relevant section of the AIPS++ User
Reference Manual is available via the link.

I shall assume that you have access to a working AIPS++ installation.
Details about setting up your AIPS++ environment are given in 
\htmladdnormallink{Getting Started in AIPS++}
{../../user/gettingstarted/gettingstarted.html}.

\section{Starting AIPS++ and Glish.}

    In a new directory you can establish an AIPS++ and glish environment
with the following UNIX command:

\begin{verbatim}
For csh-like shells:
$ source /aips++/stable/aipsinit.csh

For bash-like shells:
% . /aips++/stable/aipsinit.sh
\end{verbatim}

Note that the directory name, \verb!/aips++/stable!, may vary depending upon
your installation; see \htmladdnormallink{Getting Started in
AIPS++}{../../user/gettingstarted/gettingstarted.html} for more details.

Then start AIPS++, open the example files, and load the example glish
scripts with

\begin{verbatim}
$ aips++
- include 'glishtutorial.g'
\end{verbatim}
The script 'glishtutorial.g' converts the example FITS data files used
in this tutorial into tables which can be read directly in aips++.  When the 
conversion process has completed the glish prompt will return.

AIPS++ without the example files and scripts can be started with simply

\begin{verbatim}
$ aips++
\end{verbatim}

The glish prompt is either '-' or '+'.  To test glish and plot window
operation run the commands:

\begin{verbatim}
- mp := pgplotter()
- mp.plotxy([1:100]/5.0 ,sin([1:100] / 5.0), title="sine wave")
\end{verbatim}

You should see about three cycles of a sine wave. This is plotted
using the AIPS++ utility pgplotter.

\section{Variable Assignment and Automatic Typing}

    Glish variables are automatically assigned a variable type when they
are created.  In other words, there is no variable declaration statement.
After the following assignments:

\begin{verbatim}
- x := 2
- z := 23.786
- check := F
- status := T
- amp := 1.66 - 0.994i
\end{verbatim}

'x' is an integer, 'z' is a double, 'check' and 'status' are booleans (true
or false), and 'amp' is double complex.  The available data types are
boolean, byte, short, integer, float, double, complex, and double complex.
Floating point constants are assumed to be double, and integer constants
are assumed to be integer.  In most cases the data type is self-evident,
but, if your code needs to know a variable's type, there are built-in test
functions such as is\_boolean(), is\_float(), is\_dcomplex(), etc., which 
return a boolean value T or F.  Also, the function type\_name() will return
a character string describing the type.  For example,

\begin{verbatim}
- is_float(z)
F

- type_name(z)
double 
\end{verbatim}

    Automatic variable typing is quite convenient, but it provides no
warning when you inadvertently reassign a variable.  The assignment

\begin{verbatim}
- status := 1.65e7
\end{verbatim}

is perfectly legal resulting in 'status' now being a double.  You can
protect a variable from reassignment with the 'const' keyword as in

\begin{verbatim}
- const Pi := 3.14159
\end{verbatim}

    Values of variables and constants in an arithmetic expression are all
converted to the type of highest numeric precision in the expression before
performing the mathematical operations.  In an expression using the
variables above

\begin{verbatim}
a := x * z * check
\end{verbatim}

'a' is a double.  Variable typing may be forced for purposes such as
integer truncation or storage economy with the built-in functions
as\_integer(), as\_float(), etc.  Keep in mind, however, that a variable is
retyped every time it is assigned, so, on a few occasions, your expressions
may need to be very explicit about type.

    The arithmetic operators are

\begin{verbatim}
+  -  *  /  %  ^
\end{verbatim}

where \verb!%! is the modulus operator that converts its operand values to integer
and returns an integer result, and \verb!^! is the exponentiation operator with
floating point arguments and result.

\begin{verbatim}
- 26.8 % 4.2
2
- 3.01^1.99
8.960811
\end{verbatim}

    The value of a variable may be printed with the print statement or, in
interactive mode, by simply typing the variable name without the ':='
assignment operator.  The following two statements are equivalent:

\begin{verbatim}
- print x
- x
\end{verbatim}

Be careful with array names!  You can get a real screen full.

\section{Arrays}

    Array operations are one of glish's strong points for both code economy
and computing efficiency.  The notation takes a bit of getting used to.

    An array variable may be created either by assigning to it an array
constant or the result of an array expression,

\begin{verbatim}
- indices := 1:512
- ar3 := ar2 * ar1
\end{verbatim}

or by using the array initialization function,

\begin{verbatim}
- image1 := array(0.0, 256, 256)
\end{verbatim}

The array variable, 'image1', becomes the type of the first argument of
'array()', which is the array initialization value.  The first argument
itself may be a scalar or an array.  The rest of the arguments are array
dimensions of any length and number that will fit in memory.

    Most arrays are created by assignment.  The primary use for the
'array()' function is to create an empty array to be filled piecemeal
later.  Again, remember that an assignment can change the size of an array
or even turn it into a scalar.

    Users of IDL, PV-Wave, and, to some extent, IRAF will find glish's
array index notation familiar.  Fortran and C programmers will be tempted
to attack arrays with integer indices and iterative loops.  Resist the
temptation when possible because the intrinsic array operations are much
faster.

    To avoid an abstraction of arrays let's load a continuum receiver data
column from one of the sample data tables.  A small portion of this table
looks like

\begin{verbatim}
	RECEIVER_ID  PHASE  CAL  SUBSCN    DATA
              0         0     0      1     600885
              1         0     0      1     572540
              0         1     1      1     649341
              1         1     1      1     617254
              0         0     0      2     601413
              1         0     0      2     573158
              0         1     1      2     649827
              1         1     1      2     617841
              0         0     0      3     602472
              1         0     0      3     574106
\end{verbatim}

There are many more rows and columns to this table, but these include the
interesting columns for the moment.

Now load the 'DATA' column from table t1 into a glish array.  This table has
already been opened using the AIPS++ table system if
you included 'glishtutorial.g'.

\begin{verbatim}
- data := t1.getcol('DATA')
\end{verbatim}

If you want a complete picture of this table, browse the table with

\begin{verbatim}
- t1.browse()
\end{verbatim}

The vector 'data' (1-D array) contains interleaved data from two receiver 
channels with the noise calibration signal turned on and off.  To look at a 
single element just use an integer index.

\begin{verbatim}
- data[3]
649341 
\end{verbatim}

Notice that indices start at 1.  To access a contiguous range of values in
the vector use an index range.

\begin{verbatim}
- data[1:5]
[600885 572540 649341 617254 601413]
\end{verbatim}

Any subset of vector contents may be acquired with an index array containing
the indices of the values you want.

\begin{verbatim}
- indx := [2,5,9]
- data[indx]
[572540 601413 602472]
\end{verbatim}

or just simply

\begin{verbatim}
- data[[2,5,9]]
[572540 601413 602472]
\end{verbatim}

Notice the second set of square brackets which define the index list as
members of an integer vector rather than indices of different dimensions of
a three-dimensional array.  To digress a moment, a multidimensional array
is addressed as follows:

\begin{verbatim}
- mda := array(0.0, 16, 16, 16)
- mda[5, 3, 14]			# column 5, row 3, slice 14
- mda[ , 6, 5]			# all values in row 6, slice 5
- mda[11, , ]			# values in the column 11 plane
- mda[ , ,3:5]			# 3 planes of slices 3, 4, and 5
\end{verbatim}

The \# character is the comment delimiter.  Everything from it to the end of
the line is ignored by glish.

    If you like, you can load an example of a two-dimensional array of data
from a different table that contains a column of spectra.

\begin{verbatim}
- spectra := t7.getcol('DATA')
\end{verbatim}

    Glish variables have attributes, a few of which are predefined.  The
dimensions of an array are contained in its 'shape' attribute.  The syntax
for an attribute variable is

\begin{verbatim}
- spectra::shape
[1024 33]
\end{verbatim}

In this case, there are 33 spectra of 1024 channels each.

    Back to our 'data' vector, rather than generate an index vector it is
more efficient to create a boolean vector with the same number of elements
as the 'data' vector and use it as an element mask.  For example, to get
all of the 'data' values for receiver 1 with the calibration noise ``on'' get
the RECEIVER\_ID and CAL columns.

\begin{verbatim}
- rcvr := t1.getcol('RECEIVER_ID');
- cal := t1.getcol('CAL')
\end{verbatim}

Then make a boolean vector of the same length and use it as an index mask
to the 'data' vector.

\begin{verbatim}
- mask := (rcvr == 1) & (cal == 1)
- mask[1:10]
[F F F T F F F T F F]  

- data_cal_1 := data[mask]
- data_cal_1[1:5]
[617254 617841 618768 619804 620325]

- print length(data), length(data_cal_1)
960 240
\end{verbatim}

From the values and length of 'data\_cal\_1' we can see that it contains the
values from the 'data' vector where the boolean vector, 'mask' , is true.
The function length() returns the number of elements in the array, not the
number of bytes.  The equivalent subarray access could be written without
the intermediate 'mask' variable.

\begin{verbatim}
- data_cal_1 := data[rcvr == 1 & cal == 1]
\end{verbatim}

We have also used the operator precedence rules to eliminate the
parentheses, but, when in doubt, use parentheses. The comparison operators
are

\begin{verbatim}
<  >  <=  >=  ==  !=
\end{verbatim}

which are less than, greater than, less than or equal to, greater than or
equal to, equal to, and not equal to, respectively.  The boolean operators
are ``and'', ``or'', and ``not''.

\begin{verbatim}
&  |  !
\end{verbatim}

All of the arithmetic, comparison, and boolean operators operate on arrays
element by element.  Therefore, all arrays in an expression must be of the
same size in each dimension.  For example,

\begin{verbatim}
- [1:3] * [1:3]
[1 4 9]
\end{verbatim}

It is not a vector dot product.

    For a graphical illustration of what we have done with the extracted
data column get one more column, 'Time', from the table and plot the two
data vectors against it.

\begin{verbatim}
- times := t1.getcol('Time')
- times -:= as_integer(times)		# remove day numbers
- mp := pgplotter()
- mp.env(.96294,.96304,5.6e5,6.6e5,0,0)
- mp.line(times[1:32], data[1:32])
- sometimes := (times[1:32])[mask[1:32]]
- mp.eras()
- mp.env(.96294,.96304,6.1e5,6.3e5,0,0)
- mp.line(sometimes, data_cal_1[1:length(sometimes)])
\end{verbatim}

The second line above illustrates one of the combined arithmetic and
assignment operators ( +:= -:= *:= /:= \%:= \^:= \&:= |:= ) that are also
found in C.  The assignment of 'sometimes' is a combined use of vector
ranges and a mask.

    If we assume that the receiver 1 noise calibration intensity is 1.4
Kelvins, we can plot a calibrated scan with the additional commands

\begin{verbatim}
- data_cal_off_1 := data[rcvr == 1 & cal == 0]
- dataK := 1.4 * ((data_cal_1 + data_cal_off_1) / 2.0) /
+ (data_cal_1 - data_cal_off_1)
- mp.eras()
- mp.plotxy([1:len(dataK)],dataK)
\end{verbatim}

Note that a different (and simpler) plotting function, mp.plotxy(), replaces
the mp.env() and mp.line() commands used previously.  The latter functions
were used at first because of the unusual requirements on the axes ranges.

You should see four scans through a radio source placed end to end.  These
are actually four separate scans in the same table that were taken with a
cross pointing scan procedure on the 140-foot.  Note that the \verb!+!
on the third line in the previous example is the glish prompt
and not an indication that you should type the plus character.

    All of the other column names in this table may be listed with

\begin{verbatim}
- t1.colnames()
\end{verbatim}

    Here are some additional indexing ideas.

Rotate a vector five places.

\begin{verbatim}
- rot_vector := data[[5:data::shape,1:4]]
\end{verbatim}

Delete out-of-range values.

\begin{verbatim}
- purged := data[data < 700000]
\end{verbatim}

Make a new time vector to match the 'purged' vector.

\begin{verbatim}
- ptime := times[data < 700000]
\end{verbatim}

Reverse a vector.

\begin{verbatim}
- flipped := data[data::shape:1]
\end{verbatim}

Splice two vectors together.

\begin{verbatim}
- spliced := [v1, v2]
\end{verbatim}

\section{Scripts and Functions}

    Glish code may be written in a file with an editor and executed with
the 'include' command.  These scripts contain glish code that is identical
to the text that you type at the glish prompt.  Script files are
conventionally given the name extension '.g'.  The glish command syntax is

\begin{verbatim}
- include 'myscript.g'
\end{verbatim}

The most frequent use of script files is to edit and load function
definitions.  The syntax may be either

\begin{verbatim}
function <fn name> ( <arguments> ) {
    statements
}
\end{verbatim}
or
\begin{verbatim}
<fn name> := function ( <arguments> ) {
    statements
}
\end{verbatim}

The 'function' keyword may be abbreviated 'func'.  The curly brackets are
to group all of the statements together.  Without the brackets, only the
one statement following the function arguments would be part of the
function.  Glish is pretty liberal about whether brackets and statements
appear on same or different lines and about how you use indentation, as
long as the meaning is unambiguous.  A function may return any variable
type including whole arrays.

    Here is a simple function for plotting data from one receiver of one
scan in a table.

\begin{verbatim}
plot_scan := function (tab, scan_no, receiver=1)
{
    mp := pgplotter()
    data := tab.getcol('DATA');
    scan := tab.getcol('SCAN');
    rcvr := tab.getcol('RECEIVER_ID');

    mask := scan == scan_no & receiver == (rcvr + 1);
    scan_data := data[mask];
    if (length(scan_data) == 0) {
        fail paste('No data found for scan', scan_no, 
                   'receiver', receiver);
    }
    mp.plotxy([1:len(data[mask])], data[mask], 
              title=paste('Scan', scan_no, 'Receiver', receiver));
           return
}
\end{verbatim}

Type this function into a script file, say, 'test.g'.  Load it into glish
with

\begin{verbatim}
- include 'test.g'
\end{verbatim}

and execute the function with

\begin{verbatim}
- plot_scan(t1, 48)
\end{verbatim}

    There are a few new items in the function definition above.  The lines
are terminated with semicolons.  This is optional both in a script file and
at the glish prompt.  If you put two lines of code on the same physical
line, they must be separated by semicolons.  The function contains an if()
statement whose argument is normally a boolean expression, but it can be a
numeric variable.  Zero is false, and everything else is true.  'pgplotter()'
and 'mp.plotxy()'
are AIPS++ functions.  'paste()' is a glish built-in string concatenation
function. 'fail' is a glish built-in that causes an error return; use
it like this when you want to signal an error.

    Function arguments may be defaulted by assigning them a value in the
function definition, as we did with the 'receiver' argument.  When a
function is called, if no value is given for a defaulted argument, the
argument assumes the value given in the function definition.  Undefaulted
arguments must be given values in a function call.  Normally, arguments are
given values according to their position in the argument list, but they can
be given in any order, if they are named.  For example, the plot\_scan
function call above could also have been done with

\begin{verbatim}
- plot_scan(scan_no=48, receiver=1, tab=t1)
\end{verbatim}

The first arguments may be specified by position, but after an argument is
named all the rest must be named.

    Variables defined by assignment within functions are local to that
function.  In other words, when the function execution completes the
variable goes away.  A variable which doesn't disappear may be created by
declaring it to be ``global''.  Likewise, an externally defined variable may
be assigned a value within a function by declaring that variable to be
global within the function.  For example, look at the results of the
execution of two functions defined below.

\begin{verbatim}
x := 10
bill := function () {
    x := 5
}
fred := function () {
    global x := 6
}

- x
10
- bill()
F
- x
10
- fred()
F
- x
6
\end{verbatim}

The \verb!F! that appears after you execute the functions is the glish
boolean, False.  This is the default return value for functions
which do not explicitly return a value.
 
An unusual feature of glish is that global variables can be implicitly
referenced from within a function. However, by default, they become local
to the function upon assignment, for example,

\begin{verbatim}
- gvar := 0.841471
- function x() {
+    print gvar
+    gvar := 90
warning, scope of "gvar" goes from global to local
+    print gvar
+ }
- x()   
0.841471
90
- print gvar
0.841471
\end{verbatim}

Glish provides a warning whenever the scope of a variable goes from global
to local.  If you want the global variable to be affected, you must
explicitly declare it with the 'global' keyword, as in

\begin{verbatim}
- gvar := 0.841471
- function x() {
+    global gvar := 90
+ }
- x()
- print gvar
90
\end{verbatim}

    Finally, if you don't like the name of a function, or it's too long to
type conveniently, you can give it an alias by assigning the function name
to another name, just like assigning one variable to another.

\begin{verbatim}
- ps := plot_scan	# note no parenthesis
- ps(t1, 48)
\end{verbatim}

Be careful, however!  You can wipe out an entire function definition by
reassigning it.  If you make the assignment,

\begin{verbatim}
- plot_scan := 5
\end{verbatim}

the function plot\_scan() is gone until you redefine it by including its
script file again.  However, the variable ps still contains the copy
of plot\_scan that you assigned to it earlier.

As with variables, you can protect a function name and
its arguments with the 'const' keyword, for example,

\begin{verbatim}
- const myplot := function (const ary) { ... }
\end{verbatim}

After you have finished debugging a function, it is usually a good idea to
protect it with 'const'.

\section{Loops and Conditionals}

    Three familiar program flow control statements are available in glish,
``if/else'', ``while'', and ``for'' loop statements.  The general syntax for each
is

\begin{verbatim}
if (<boolean condition>) {
    statements
} else {			# there is no 'else if'
    other statements
}

while (<boolean condition>) {
    execute all of this
}

for ( <variable> in <vector> ) {
    execute all of this
}
\end{verbatim}

The ``for'' loop is a bit different to its equivalents in Fortran and C in
the sense that the running variable doesn't have to make equal steps.  It
just takes on the values of the elements of the vector in succession,
whatever those values may be.  A garden variety loop of i from 1 to 10 in
steps of 1 is

\begin{verbatim}
for (i in 1:10) {
}
\end{verbatim}

Every method you can think of for generating a vector can be used to create
the vector argument in the ``for'' statement.  For example, you can use the
'seq()' built-in function to step from 3.66 to 7.35 in steps of 0.0123.

\begin{verbatim}
for (x in seq(3.66, 7.35, 0.0123)) {
}
\end{verbatim}

Even a boolean vector is OK.

\begin{verbatim}
for (i in [F, T, T, F, T]) {
    print i;
}
\end{verbatim}

\section{Records and Attributes}

    Scalars and arrays of the standard data types are not the only
variables allowed by glish.  Heterogeneous data structures, called records,
can be treated as a single variable.  These can be passed as arguments to a
function and returned as a unit in a function return value.  A record may be
created all at once or added to piece by piece.

\begin{verbatim}
- header := [name='3C273', ra=12.12345, dec=2.3318]
- header.flux := 27.6
- header.type := 'quasar'
- print header.ra
12.12345
\end{verbatim}

    Record members may be strings, any of the numeric data types
including arrays, as well as other records.

    Any glish variable may be assigned attributes.  For example,

\begin{verbatim}
- velocity::definition := 'relativistic'
- header.flux::units := 'Jy'
\end{verbatim}

The distinction between record members and variable attributes is a subtle
one, and it is seldom clear which is best suited to a particular
application.  Both record members and attributes may be accessed by name or
by number.  All of the following are acceptable:

\begin{verbatim}
- header.ra
- header[2]
- header['type']
- velocity::[1]
- velocity::definition
- velocity::['definition']
\end{verbatim}

Numeric access lends itself to stepping through the fields or attributes
with a loop index.  Record field names may be retrieved with the built-in
function 'field\_names()', which returns a string.

\begin{verbatim}
- field_names(header)
name ra dec flux type
\end{verbatim}

The empty attribute construct returns the attributes of a variable as a
record, so you can use the 'field\_names()' function on attributes, too.

\begin{verbatim}
- velocity::
[definition=relativistic]
- field_names(velocity::)
definition
\end{verbatim}


\section{Strings}

    There are two types of string constants in glish.  Single quoted
strings are treated as a single entity and are most useful in print
statements where some control of placement of characters on the screen is
important.  Double quoted strings are treated as arrays of strings with
white space separating the array members.  The output of 'field\_names()' in
the example above is an instance of a string array.  The difference is best
illustrated with some examples.

\begin{verbatim}
- s1 := 'This is a single quoted      string'
- s2 := "This is a double quoted      string"
- s1
This is a single quoted      string 
- s2
This is a double quoted string 
- s1[1]
This is a single quoted      string 
- s2[1]
This 
- s1[2]
<fail>: index (= 2) out of range, array length = 1 
- s2[2]
is 
\end{verbatim}

Notice that all white spaces have been reduced to single spaces when the
double quoted string array is printed.

    String handling is fairly primitive in glish.  There are no string
operators, except possibly for the cautious use of the comparison
operators.

\begin{verbatim}
- twenty := '20'
- three := '3'
- twenty < three
T
\end{verbatim}

If one of the variables, twenty or three, had not been a string, an
error would have been reported.  When in doubt, use the is\_string() test
first.

    There are a few string manipulation functions, including paste(), 
spaste(), and
split().  Paste() and spaste() act like print statements except that the
resulting string is returned as a single string instead of being printed.

\begin{verbatim}
- r := 53.6
- sx := paste('R =', r)
- sx
R = 53.6
\end{verbatim}

The paste() function can take a named argument, sep, to change the
separator character(s) from the space default.

\begin{verbatim}
- paste('R', r, sep=' = ')
R = 53.6
\end{verbatim}

The function spaste() is the same as paste() with no separator.
The string function split() simply turns a single string into a
string array or breaks a string array differently.  It takes two arguments,
the string to be split and a list of characters to be used as separators.
The default separator is a space.

\begin{verbatim}
- str1 := 'This is a single quoted string'
- str2 := "This is a double quoted string"
- sx := split(str1)
- sx[1]
This
- sx := split(str2, "u")
- sx
This is a do ble q oted string
- sx[3]
oted string
\end{verbatim}

\section{Printing}

    The print statement is quite primitive so output formatting in glish is
minimal.  The output of print is simply the concatenation of the quantities
listed converted to strings and separated by spaces.  You can add spaces by
printing spaces in single quoted strings, and you can control the number of
decimal places shown for floating point numbers with one of two print
precision variables.

\begin{verbatim}
- r := 3; s := 5.6; j := "spa   ces"
- print '     show', r, s, j
     show 3 5.6 spa ces
\end{verbatim}

    The printed precision of all floating point variables is set by the
value of a member of the system record variable.

\begin{verbatim}
- old_prec := system.print.precision
- system.print.precision := 3
- x := 234.12345
- print x
234
- system.print.precision := old_prec
\end{verbatim}

    The printed precision of one variable may be set by the member of the
variable's print attribute.

\begin{verbatim}
- x::print.precision := 4
- print x
234.1
\end{verbatim}

    To avoid the possibility of inadvertently printing a huge array to the
screen, you can set a limit on the number of printed array elements with

\begin{verbatim}
- system.print.limit := 50
or
- any_array::print.limit := 50
\end{verbatim}

The effects of print.precision or print.limit may be cancelled by setting
them to a value less than zero.  You may want to set a system print limit
in your .glishrc file.  See
Appendix A and \htmladdnormallink{Getting Started in AIPS++}{../../user/gettingstarted/gettingstarted.html}
for more details on the role and uses of your .glishrc file.


\section{Clients, Events, and Agents}

    From a software designer's point of view, the important part of glish is
its ability to link to any number of independent processes running anywhere
on a computer network.  To glish these independent processes are seen as
``clients'', but the client-server distinction is ambiguous since services can
flow either way.  Communication between glish and its clients is by way of
``events'', rather analogous to hardware interrupts but with the added feature
that events can carry any amount of data to or from a client.

    As an AIPS++ user you may seldom encounter the use of events in your
glish programming.  Events are considerably different in concept from usual
linear scientific programming.  Hence, they are usually wrapped in more
familiar function calls to avoid the need to learn events to use the
services of AIPS++ clients.  In fact, in the examples above we have used
a number of clients, including the table browser and the plot client,
without necessarily being aware of events passing back and forth.  It's not
that clients and events are all that obscure.  It is just that they are one
more concept to learn that might not be of any direct interest.

    Clients can do anything that a normal computer program can do from
performing FFT's to driving a telescope to position.  Most of the
application-specific facilities of AIPS++ are in the clients that glish
initiates.

    To see how the events work try the following commands associated with
the Tk widgets attached to glish.

\begin{verbatim}
- mp := pgplotter()
- fm := frame()
- b1 := button(fm, 'plot')
- whenever b1->press do {
-  mp.eras()
-  mp.plotxy([1:100],sin(as_float(1:100) / 20.0), title='Button plot')
- }
\end{verbatim}

The first statement puts a small window frame on the screen.  The second
puts a button labeled 'plot' in the window.  In the whenever statement,
``press'' is the name of an event sent by the Tk widgets client when the
button is pressed with the mouse.  The statements within the whenever block
are executed when the 'press' event is received.  Try pressing the button.

    Clients are started from glish with the client() function.  A more
complex example than the button client might be a hypothetical FFT engine.
Let's say that we have an executable file on disk, called ``fft\_engine'',
that performs Fourier transforms and knows how to send and receive glish
events.  Start it up with

\begin{verbatim}
- fft := client('fft_engine', host='arcturus')
\end{verbatim}

The host argument is not needed if ``fft\_engine'' is on the same machine that
is running glish.  The variable, fft, is of a special type called an agent.

    Suppose that the documentation for fft\_engine says that it recognizes
an event called ``do\_fft'', and , when the FFT is done, fft\_engine will send
an event called ``fft\_done'' with a new array of transformed data.  First,
set up to receive the ``fft\_done'' event and do something with it.  Then send
the fft request with the data array.  Probably in a script called 'fft.g'
we'd have

\begin{verbatim}
fft := client('fft_engine')
whenever fft->fft_done do {
   mp.eras()
   mp.plotxy([1:len($value)], $value, title='FFT Result')
}
\end{verbatim}

Then from the command line do the following:

\begin{verbatim}
- include 'fft.g'
- data := sin(as_float(1:128) / 3.0)
- send fft->do_fft(data)
- await fft->fft_done
\end{verbatim}

    In this case we have elected to wait for the ``fft\_done'' event to return
and be processed, but we could have continued to issue glish statements for
other purposes.  The FFT result would be returned and plotted whenever the
fft\_engine client was finished with it.

    The parameter, \$value, is a special one associated with events.  It
contains the event's data and can be anything from a single number or
string to a very complicated record with arrays and other records as
members.  Normally, glish and the client know what to expect in \$value, but
some flexibility of contents and data types can be supported with the
is\_xxxx() test functions.

    A fully implemented glish client is written at least partly in C++
using the classes that support events and glish data structures.  However,
existing programs may be turned into simple clients using standard input
and output ASCII data as event values.  Also, clients may be written in the
glish language, but there are fewer reasons for doing it this way.

    The standard-in, standard-out client is created by the shell() function
rather than the client() function.  A one-time communication with a client
program is simply the UNIX command as a string argument to shell().

\begin{verbatim}
- file_list := shell('ls *.g')
\end{verbatim}

The variable, file\_list, is then an array of strings, one string for each
file name that ends with ``.g'' in the directory.

    If the program can handle more than one exchange of data by way of UNIX
pipes, it may be started and communicated to with events.  The startup
glish code looks something like

\begin{verbatim}
a := shell('my_prog', async=T)
whenever a->stdout do {
    print $value
}
whenever a->stderr do {
    print 'Error !', $value
}
\end{verbatim}

The async=T parameter assignment causes the shell() function to return an
agent type quantity instead of a string.  The data events that can be sent
by the client started with the shell command are called ``stdout'', 
and ``stderr'' and their \$value is always a string.  The two 
events recognized by a shell client are
``stdin'', whose argument must always be a string, and ``EOF''.  Hence, an
event sent to ``my\_prog'' is, for example,

\begin{verbatim}
- send a->stdin('23.876 14.999')
\end{verbatim}

    To avoid having an event arrive from a new client before its associated
whenever statements are set up, put the client() or shell() function calls
and the whenever statement in the same script file, and run them together in
the script.  Glish guarantees that events are held until the full script
file is read.

\section{What now?}

To learn more about Glish, you might wish to read the
\htmladdnormallink{``Glish 2.7 User
Manual''}{../../reference/Glish/Glish.html}. For information on AIPS++,
see \htmladdnormallink{Getting Started in
AIPS++}{../../user/gettingstarted/gettingstarted.html} and the
\htmladdnormallink{AIPS++ User Reference
Manual}{../../user/Refman/Refman.html}.

\section{Appendix A - Glish Environment}

Glish can be most easily run by setting up the AIPS++ environment.
For information on how to do this, see \htmladdnormallink{Getting
Started in AIPS++}{../../user/gettingstarted/gettingstarted.html}.

    Glish, itself, reads an environment file, called .glishrc, which may be
used to set a number of system variables.  The most important one is
system.path.include, which specifies the directories to be searched for
``.g'' files that are loaded with the include statement.  Any ``.g'' file may
be loaded by giving its full path name, but the system.path.include
assignment saves a lot of typing.  The contents of a .glishrc file might
look like

\begin{verbatim}
print 'reading /hyades1/rfisher/Pulsars/P1823/.glishrc'
 
system.path.include := 
    ". /gbt3/aips++/sun4sol/libexec /aips2/glishScripts-2509a /";
 
print 'system.path.include: '
for (i in 1:len(system.path.include))
    print '    path', i, system.path.include [i];
\end{verbatim}

The print statements are just embellishments to remind you of where the
environment variables are coming from when glish starts up.  Notice the '.'
path name as well as the system glish directories.

    When glish starts up it reads the .glishrc file in the system glish
directory.  It then looks for a .glishrc file in your current directory.
If none is found there, it tries your home directory.  Note that the
system.path.include assignment shown above will override the one in the
system .glishrc file.  If you just want to add additional paths, you can
use the following construct to expand the string array:

\begin{verbatim}
system.path.include[len(system.path.include) + 1] := '/home/rfisher'
\end{verbatim}

Arrays are automatically expanded in an assignment statement when the
index of the assigned array is greater than its current size.  ``len()'' is
another name for the length() function.

    Finally, it's worth a look at the UNIX script that starts the
glish/AIPS++ combination.  This script is the command given at the
beginning of this document, for example, ``aips++''.  To find out where this
script is, use the UNIX 'which' command.  Then list the file using the full
path name.

\begin{verbatim}
$ which aips++
/aips++/stable/sun4sol_gnu/bin/aips++
$ more /aips++/stable/sun4sol_gnu/bin/aips++
[ lines setting up the aips++ environment variables deleted ]
exec $BINDIR/glish -l $LIBDIR/aips++path.g -l $LIBDIR/aips++.g
\end{verbatim}

The last line executed in this script is the command that starts 
glish.  It automatically includes two files, \verb!aips++path.g! and
\verb!aips++.g!.
Both files contains glish
code which loads the aips++ clients.  The ``-l'' (minus ell) flag tells glish
to keep running in interactive mode.  Without this flag the script \verb!aips++.g!
would be executed and then glish would terminate.

    If you want to start glish without any of the AIPS++ functions or
clients, just issue the UNIX command

\begin{verbatim}
$ glish
\end{verbatim}

\section{Appendix B - Example Data Files}

    The two AIPS++ tables, t1 and t7, used in the examples in this document
were created from some GBT system test data from the 140-ft telescope.
Other instruments, such as the VLA, will have a different file and
directory structure for the original data from the one outlined below, but
many properties of the created AIPS++ tables will be the same.

    The original data are in several FITS Binary Table files written by
different system modules, e.g., the weather monitor, telescope position
control, and one of the data backends.  The t1 table contains data from the
continuum backend, and t7 is a bit of spectral processor data in spectral
line mode.  There will be an automated process to convert the raw FITS data
into tables readable by AIPS++.  For now, the data must be filled by hand.
The utility to fill data from the GBT backends, for example, is called
gbtmsfiller.

    A step to opening tables within glish was hidden by the glishtutorial.g
script.  Namely, to open a table use:

\begin{verbatim}
- t7 := table('t7')
\end{verbatim}
where the argument to the table() function is the directory produced by the
filler.
