\section {Introduction}
These standards and guidelines have been chosen to promote a consistent
style, clarity and organization in AIPS++ code.  {\em Standards} are coding
practices which everyone must follow; {\em guidelines} are recommended
practices, and are identified below by ($\ast$).
\bmar Use of standards and guidelines will in general reduce the number of
initial defects by about 25\%, and make maintenance of code easier.\bmar

{\it Effective C++} and {\it More Effective C++} \bmar 
by Scott Meyers is a good guide to C++ programming.  
Most of his guidelines apply to AIPS++. 
%%---------------------------------------------------------------------------
\section {Code Organization}
%%---------------------------------------------------------------------------
\subsection {Classes} Every class will have a header file,\footnote{See {\em
code/install/codedevl/template-class-h}} an implementation file,\footnote
{See {\em code/install/codedevl/template-class-cc}. Neither totally inlined
classes, nor abstract base classes may need an implementation file.} a test
program \footnote {See {\em code/install/codedevl/template-main-cc}} 
and (optionally) a demo program.  Very closely related classes\footnote {A
good example is a read-only class, and a read-write class derived from it.}
may sometimes be combined into single header and implementation
files.\footnote {If you do this, make sure that the documentation sections --
{\em synopsis, etymology, motivation}, etc. -- are completed for {\em every}
class contained in the file.}

\bmar All files shall follow the latest template file as a guide for the
standard layout.

It is sometimes advisable to separate the implementation into different
files. Examples are to separate templated and non-templated methods, and when
a sub-set of the methods is often used. Such implementation files have in
general names like {\it name2.cc} etc. See {\it
code/aips/implement/Arrays/Array2.cc} for an example.

The test program(s) shall exercise every member function, every exception, and 
cover at least 90\% of the lines of code in the implementation file.\footnote
{Special language tools can measure coverage.  See AIPS++ Note 170,
``The AIPS++ Code Review Process'' for suggestions on writing test programs.}
%%---------------------------------------------------------------------------
\subsection {Include files} \bmar Whenever possible the header files should
use forward declarations\footnote {Explicitly or implicitly like in {\it
iosfwd} or {\it Complexfwd}} in stead of include files. Include files should
be reserved for the implementation files.
%%---------------------------------------------------------------------------
\subsection {Data Types} \bmar Only the standard AIPS++ data types 
\footnote {Char, uChar, Int, uInt, Float, Double, lDouble, Complex, DComplex,
String. In special circumstances the use of Long and uLong is allowed.} 
shall be used, coded in the AIPS++ way. 
%%---------------------------------------------------------------------------
\subsection {Modules} Closely related classes should be organized into a
module -- which requires an appropriately named, separate directory.\footnote
{See {\em code/install/codedevl/template-module-h}. The Tables module,
{\em code/aips/implement/Tables} is a good example to study.} Modules \bmar
are assigned by the AIPS++ system manager. The module
will have a ``module header file'' -- whose primary purpose is to simplify
things for the application programmer by \bmar providing sufficient help
information for the general use of the classes in the module.
\bmar  Whenever possible, application
programmers should not be required to learn the intricacies of a complicated
module, nor to understand specific implementation classes within the
module. In these cases they 
should only have to include a single header file into their application
program. In cases where the role of the module is more a conglomerate of
related, but independent, classes, individual header files \bmar should be
used.\footnote {As a rule, it is advisable for application programmers to
include in their code the include files for the individual classes being
used.}

The module header file \footnote{See {\em
code/install/codedevl/template-module-h}} will live in the module's parent
directory: {\em code/aips/implement/Tables.h}, for example, is the module
header file for, and describes the classes in 
{\em code/aips/implement/Tables/}.
%%---------------------------------------------------------------------------
\subsection {Global Functions} It is sometimes appropriate to write
global functions to act on class objects, rather than creating
class member functions.  As a general rule, you should do this if
there is no class object state preserved from one function call to the next,
and all of the information about the class object/s can be obtained
from the public interface.  \bmar However, for reasons of namespace
management it
is often advised if there exists a strong binding between the class and the
global function, to make the global function a static member of the class; or
use the namespace keyword.
There are two places where global functions
may be declared and defined:  as part of the {\em .cc} and {\em .h} files
of the class they are most closely associated with, or in their own, separate 
{\em .cc} and {\em .h} files.  Use these criteria to decide:
\begin{enumerate}
\item If there is a large conceptual distance between the functions and
the class, then it is probably best to use separate files.
\item If there are a large number of related global functions, use
separate files.
\item If the functions are templated, using separate files (at least separate
implementation \bmar files) will reduce
template instantiation dependencies.
\end{enumerate}
%%---------------------------------------------------------------------------
\subsection {Templates} \bmar Template instantiations are done explicitly
within AIPS++. Make sure templates needed by your code are defined in the
package {\it templates} file in the {\it \_Repository directory}. Templates
needed only in your test or demo programs should exist in your {\it
test/templates} file. See the {\em System Manual} for details about template
management.\footnote {reident, unused, duplicates programs}
%%---------------------------------------------------------------------------
\subsection {File Size ($\ast$)}
Files should rarely -- if ever -- exceed 2000 lines:  a 1000 line
maximum will usually apply.  Well-designed classes often have short header
and, certainly, short implementation files, usually less than 500 lines
including documentation.
%%---------------------------------------------------------------------------
\subsection {Function Length ($\ast$)}
Functions should rarely exceed 100 lines in length; shorter, well-focused
functions should dominate, and will usually be less than 50 lines.
%%---------------------------------------------------------------------------
\section {Documentation and Naming}
\subsection {Documentation in Header Files}
Header files shall contain clear and complete documentation for classes and
modules.  Templates (or ``boilerplate'') for these files are 
{\em code/install/codedevl/template-class-h} and {\em template-module-h}.
That directory also has templates for other standard kinds of files.
%%---------------------------------------------------------------------------
\subsection {Documentation in Implementation Files}
Background, usage, and overall design is presented in class header files,
and implementation files do not need to repeat them.  But non-obvious
sections of code do need to be accompanied by explanation, including 
references to manuals or texts as appropriate. ``Obvious'' is to be
judged from the perspective of a competent programmer, generally 
knowledgeable -- but not necessarily expert -- in the program domain.  The
guiding principle is:  do not force those who read your code to spend 
unnecessary time deciphering it.  It will {\em always} be a net savings
of time if you document what you have done, and why you did it.
%%---------------------------------------------------------------------------
\subsection {Names for Classes, Functions and Variables}
All class, function and variable names will reflect a balance between
clarity and convenience, with clarity having priorty.  Idiomatic 
abbreviations, acronyms and contractions are discouraged.
Names for classes, structures, and enumerated types shall begin with
an uppercase letter.  Identifers which consist of more than one word
(a recommended practice) shall use uppercase at the beginning of each
new word (``{\em GoodClassName}'').  Local variables, class data members,
class member functions, and global functions names shall begin with
lowercase letters (``{\em goodVarName, goodFunctionName}''). \bmar Class data
members shall have names prepended by {\it its}, or postpended with {\it \_p}
(``{\em itsCount, blockPointer\_p}''). 

\bmar {\em Note:} Glish function names follow a different policy: they shall
use lower case letters exclusively.
%%---------------------------------------------------------------------------
\subsection {Names for Files} The only restrictions here are
commonsense, and the Unix library archive utility ``{\em ar}''. \bmar It is
strongly suggested that the file name equals the name of the one or dominant
class in the file. The
above suggests that file names should be long enough to convey meaning
to the reader, but not so long as to be a burden to type. This
translates to: file names should be from about 8 to about 30
characters long.

``{\em ar}'' requires that all object file names be unique in the first 14
characters.  

Underscores are not allowed in file names.

Any file (a shell script, or a C++ program, for example) which can be
invoked at the command line as an executable program, must be named
with lower case chararacters only.  (This policy follows the glish function
naming policy of the previous section; both are designed to present the 
end user with only lower case commands to type, but note that underscores
are prohibited in file names.)

Use the standard extensions ({\em .c, .cc, .h, .f, .g}).
%%---------------------------------------------------------------------------
\subsection {Format for Dates}
Programmers (and code reviewers) must enter dates when documenting code.
These dates indicate, for example, when code has been reviewed, and when a list 
of ``to do'' tasks was last updated.\footnote{See, for instance, the 
documentation tags {\em reviewed} and {\em todo} in {\em template-class-h}.}
AIPS++ uses a single mandatory format for expressing dates.  It has the virtue
of being unambiguous, it is comprehensible to readers around the world, and
it is also used by the project's version control system {\em RCS}.  This 
format is {\em yyyy/mm/dd}. An example is {\em 1995/02/27}.
%%===========================================================================
\section {Coding}
\subsection {Forward Declarations}
{\em \#include} only those header files your class absolutely requires.  Use
\bmar forward declarations when that will suffice.\footnote {This may sometimes
involve the use of special forwarding include files like iosfwd.}
%%---------------------------------------------------------------------------
\subsection {Protected Data Members ($\ast$)}
Avoid protected data members, using protected member functions for access
to private data instead.
%%---------------------------------------------------------------------------
\subsection {Access to Private Data}
Do not return pointers or references to private data, except to accomodate
Fortran or C libraries, or if efficiency absolutely requires it.\footnote {Or
declare the references and pointers const -- but be aware of threading
problems in that case}
%%---------------------------------------------------------------------------
\subsection {Label All Virtual Functions Explicitly}
Virtual functions in a derived class shall be explicitly labelled with
the {\em virtual} keyword in the class declaration, even though this is
 not required by the language.
%%---------------------------------------------------------------------------
\subsection {Document Loose Ends}
Indicate unfinished or questionable code with the documentation extractor's
tag {\em $<$todo$>$}.
%%---------------------------------------------------------------------------
\subsection {Standard Class Member Functions}
Provide definitions for all of the standard member functions which the
compiler would (otherwise) generate automatically:  default
constructor\footnote {Note that this is only created when
no user constructor at all present},
destructor\footnote {always supply a virtual destructor in a base class},
 copy constructor, and the assignment operator.  For any of 
these which you really wish to disallow, declare them private, and create
minimal implementations (if any at all).  For example:
\begin{verbatim}
   ...
   private:
      SomeClass () {;}   // disable the default constructor
   ...
\end{verbatim}
%%---------------------------------------------------------------------------
\subsection {The Order of Function Arguments }
Functions shall be declared so that their arguments (parameters) appear in 
this order: output parameters first; input/output parameters next; 
input parameters last.\footnote {This order may seem unnatural, but it is a 
direct consequence of the C++ rule that default parameters appear
{\em last} in the parameter list.}  Functions which return only one value 
should usually use the function's return-value to return that value, rather
than add an additional output parameter to the parameter list.  If
a function modifies its single argument {\em in place}, then that
argument is best understood as an input/output argument.\footnote
{But also consider {\em strcpy} from the 
standard {\em C} library, which both modifies an argument, and returns 
that same value as the function return value.}
%%---------------------------------------------------------------------------
\subsection {Formal Arguments and Default Parameters}
Specify formal arguments and default parameters in the function declaration;
use the same names in the function \bmar
definition.\footnote {In the case of a
``placeholder'' parameter that is not yet used in the definition, omit the
name {\em in the definition} to stop warnings.}
%%---------------------------------------------------------------------------
\subsection {One argument constructors} \bmar Constructors with one argument should
be made explicit, or have an explanation why automatic conversion is wanted
(as an example: {\it Unit(String)} is an obvious candidate for automatic
conversion, to be able to write e.g. ``km/s'' rather than Unit(``km/s'')).
%%---------------------------------------------------------------------------
\subsection {Return Types} 
Specify explicit return types for all functions.  Return an appropriate
integer value from {\em main}.
%%---------------------------------------------------------------------------
\subsection {const Functions} \bmar
All functions that do not change the state of
the class should be declared {\it const}. If only the ``hidden'' state (like
a cache) changes, the function should be declared {\it const}, but with a
{\it mutable} cache variable. {\it mutable} should be used carefully.
%%---------------------------------------------------------------------------
\subsection {Compound Statements} Multi-line compound statements should have
braces. \bmar E.g.
\begin{verbatim}
if (condition) {
   statement1;
} else {
   statement2;
};
\end{verbatim}
%%---------------------------------------------------------------------------
\subsection {Template arguments} \bmar 
Minimize the number of template arguments by
   utilising the {\it typename} construct for user classes, and the {\it
   traits} methodology for fundamental types.\footnote { A good example can
   be found in the MeasConvert class}
%%---------------------------------------------------------------------------
\subsection {Casting} \bmar
Only C++ style casting is allowed. The use of
   {\em reinterpret\_cast} is not allowed. Allowed are: {\em static\_cast},
   {\em const\_cast} (but be aware of what you are doing) and {\em
   dynamic\_cast}. 
%%---------------------------------------------------------------------------
\subsection {Macros} ($\ast$) \bmar
Macros (especially {\it \#if} type, should be limited
   to the {\it include guard}. Other cases should be checked with the QAG,
   who often can point out other solutions.
%%---------------------------------------------------------------------------
\subsection {Exceptions}
Exceptions should be reserved for truly exceptional circumstances, and
not used to replace status flags or condition tests. Exceptions should be
documented with the $<$thrown$>$ tag (and maybe with the thrown declaration
specifier). 
%%---------------------------------------------------------------------------
\subsection {Compiler Warnings}
All code should compile without producing warnings from the 
\bmar compiler. Most compilers do not produce many spurious warnings any more.
%%---------------------------------------------------------------------------

